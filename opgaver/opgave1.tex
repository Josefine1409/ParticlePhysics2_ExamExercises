\documentclass[../main.tex]{subfiles}

\begin{document}

%%%%%%%%%%%%%%%%%%%%%%%%%%%%%%%%%%%%%%%%%%%%%%%%%%%%%%%%%%%%%%%%%%%%%%%%%%%%%%%%%%%%%

\section{Spin-$1$ partikler med masse}

En fri massiv partikel med spin-$1$ er repræsenteret ved et vektorfelt i rumtiden på en mærkværdig måde. Befinder vi os i partiklens hvilesystem (hvilket vi kan transformere til, siden partiklen har en masse $m \ne 0$) må vektorfeltet have samme antal frihedsgrader som spin-$1$-systemet, som vi har i ikkerelativistisk kvantemekanik, altså tre polarisationstilstande givet ved projektionen af spin langs en fikseret akse (typisk $\zhat$-aksen), $S_z = 0, \, \pm 1$. For at beskrive egenskaberne for en fri spin-$1$ partikel til fulde skal vi derfor give massen, impulsen og polarisationen.


%%%%%%%%%%%%%%%%%%%%%%%%%%%%%%%%%%%%%%%%%%%%%%%%%%%%%%%%%%%%%%%%%%%%%%%%%%%%%%%%

\paragraph*{\textbf{1)}}

Naivt set kunne man tro, at man kunne benytte en trevektor, $\vv{e}$, som en generel polarisationsvektor for et massiv spin-$1$-felt, men dette passer ikke. Argumentér for at siden vi arbejder med relativistisk kvantemekanik og derfor skal bruge Lorentztransformationerne, da må den mest generelle form af polarisationsvektoren være en firvektor, $e_\mu$. Argumentér også for at der findes tre uafhængige sådanne firvektor polarisationsvektorer.


%%%%%%%%%%%%%%%%%%%%%%%%%

\paragraph*{\textbf{2)}}

Antag at vi insisterer på at $e_\mu p^\mu = 0$ og $e_\mu e^\mu = -1$, hvor $p^\mu$ er firimpulsen for partiklen. Bevis at disse relationer vil holde i ethvert referencesystem.


%%%%%%%%%%%%%%%%%%%%%%%%%

\paragraph*{\textbf{3)}}

Betragt hvilesystemet for spin-$1$-partiklen og lad os som basis for de tre polarisationstilstande vælge de standard kartesiske basisvektorer for et tredimensionelt vektorrum, blot skrevet som firvektorer, altså
\begin{align}
    e^\mu (\xhat) = \FourRowMat{0}{1}{0}{0} \: , \quad
    e^\mu (\yhat) = \FourRowMat{0}{0}{1}{0} \: , \quad \text{og} \quad
    e^\mu (\zhat) = \FourRowMat{0}{0}{0}{1} \: .
\end{align}

Antag at partiklens treimpuls, $\vv{p}$, er langs $\zhat$-retningen. Vis at de generelle polarisationsvektorer i dette tilfælde bliver $(0,\, 1,\, 0,\, 0)$, $(0,\, 0,\, 1,\, 0)$ og $(\abs{\vv{p}}/m,\, 0,\, 0,\, E/m)$.


%%%%%%%%%%%%%%%%%%%%%%%%%

\paragraph*{\textbf{4)}}

Ofte bruges en relateret men en smule anderledes basis for poolarisationstilstandene for et massiv spin-$1$-felt, de såkaldte helicitetstilstande, hvilke indiceres med heltallet $\lambda$. Disse har formen
\begin{align}
    e^\mu(\lambda = \pm 1) = \invsqrtTo \FourRowMat{0}{\mp 1}{-i}{0} \: , \quad \text{og} \quad
    e^\mu(\lambda = 0) = \invsqrtTo \FourRowMat{0}{0}{0}{1} \: .
\end{align}

Vis at hvis vi roterer disse polarisationstilstande omkring $\zhat$-aksen med en vinkel på $\theta$, da bliver de multipliceret med en faktor $\pexp{-i\lambda\theta}$. Argumentér for at denne rotation medfører, at $\lambda$ er egenværdien for den kvantemekaniske rotationsoperator, $J_z$, for hver af tilstandene.


%%%%%%%%%%%%%%%%%%%%%%%%%

\paragraph*{\textbf{5)}}

Heliciteten for en partikel er defineret som projektionen af partiklens spin på dens impuls, altså $\vv{J}\cdot\vv{p}$, hvor $J$ er spinoperatoren og $\vv{p}$ er treimpulsen. Argumentér for at hvis vi booster fra hvilesystemet til et system, som bevæger sig med partiklen med hastighed $\vv{p}/E$, så bliver $J_z$ til helicitetsoperatoren.


%%%%%%%%%%%%%%%%%%%%%%%%%

\paragraph*{\textbf{6)}}

Vis at polarisationstilstandene med veldefineret helicitet for en spin-$1$-partikel, som bevæger sig langs $\zhat$-retningen, er
\begin{align}
    e^\mu(\lambda = \pm 1) = \mp \left(0,\, \frac{\vv{e}_x \pm i\vv{e}_y}{\sqrt{2}}\right) \: , \quad \text{og} \quad
    e^\mu(\lambda = 0) = \mp \left(\abs{p},\, 0,\, 0,\, E\right) \: ,
\end{align}
hvor $\vv{e}_x$ og $\vv{e}_y$ er de standard trekomponent basisvektorer.


%%%%%%%%%%%%%%%%%%%%%%%%%

\paragraph*{\textbf{7)}}

Antag at nogen beder dig opstille en teori for en partikel med et generelt spin $S > 1$. Ved brug af det, som du har lært i de foregående delopgaver, hvordan vil du da generalisere proceduren for at konstruere polarisationstilstande for et sådan felt? Skitser blot proceduren; \emph{lad være med} gå i gang med de detaljerede beregninger.


%%%%%%%%%%%%%%%%%%%%%%%%%%%%%%%%%%%%%%%%%%%%%%%%%%%%%%%%%%%%%%%%%%%%%%%%%%%%%%%%

\subsection{Besvarelse}

%%%%%%%%%%%%%%%%%%%%%%%%%

\paragraph[1) Polarisationsvektor for massivt spin-$1$.fekt]{\textbf{1)}}




%%%%%%%%%%%%%%%%%%%%%%%%%

\paragraph[2) $e_\mu p^\mu = 0$ og $e_\mu e^\mu = 1$]{\textbf{2)}}




%%%%%%%%%%%%%%%%%%%%%%%%%

\paragraph[3) Generelle polarisationsvektorer]{\textbf{3)}}




%%%%%%%%%%%%%%%%%%%%%%%%%

\paragraph[4) $\lambda$ egenværdi for rotationsoperator $J_z$ for helicitetstilstande]{\textbf{4)}}




%%%%%%%%%%%%%%%%%%%%%%%%%

\paragraph[5) $J_z$ bliver helicitetsoperator]{\textbf{5)}}




%%%%%%%%%%%%%%%%%%%%%%%%%

\paragraph[6) Polarisationstilstande for spin-$1$-partikel i bevægelse langs $\zhat$]{\textbf{6)}}




%%%%%%%%%%%%%%%%%%%%%%%%%

\paragraph[7) Teori for spin-$S$ partikel med $S > 1$]{\textbf{7)}}




%%%%%%%%%%%%%%%%%%%%%%%%%%%%%%%%%%%%%%%%%%%%%%%%%%%%%%%%%%%%%%%%%%%%%%%%%%%%%%%%%%%%%

\end{document}