\documentclass[../main.tex]{subfiles}

\begin{document}

%%%%%%%%%%%%%%%%%%%%%%%%%%%%%%%%%%%%%%%%%%%%%%%%%%%%%%%%%%%%%%%%%%%%%%%%%%%%%%%%%%%%%

\section{Spin-$1$ partikler med masse}

En fri massiv partikel med spin-$1$ er repræsenteret ved et vektorfelt i rumtiden på en mærkværdig måde. Befinder vi os i partiklens hvilesystem (hvilket vi kan transformere til, siden partiklen har en masse $m \ne 0$) må vektorfeltet have samme antal frihedsgrader som spin-$1$-systemet, som vi har i ikkerelativistisk kvantemekanik, altså tre polarisationstilstande givet ved projektionen af spin langs en fikseret akse (typisk $\zhat$-aksen), $S_z = 0, \, \pm 1$. For at beskrive egenskaberne for en fri spin-$1$ partikel til fulde skal vi derfor give massen, impulsen og polarisationen.


%%%%%%%%%%%%%%%%%%%%%%%%%%%%%%%%%%%%%%%%%%%%%%%%%%%%%%%%%%%%%%%%%%%%%%%%%%%%%%%%

\paragraph*{\textbf{1)}}

Naivt set kunne man tro, at man kunne benytte en trevektor, $\vv{e}$, som en generel polarisationsvektor for et massiv spin-$1$-felt, men dette passer ikke. Argumentér for at siden vi arbejder med relativistisk kvantemekanik og derfor skal bruge Lorentztransformationerne, da må den mest generelle form af polarisationsvektoren være en firvektor, $e_\mu$. Argumentér også for at der findes tre uafhængige sådanne firvektor polarisationsvektorer.


%%%%%%%%%%%%%%%%%%%%%%%%%

\paragraph*{\textbf{2)}} \label{sec:Opg1_Q2}

Antag at vi insisterer på at $e_\mu p^\mu = 0$ og $e_\mu e^\mu = -1$, hvor $p^\mu$ er firimpulsen for partiklen. Bevis at disse relationer vil holde i ethvert referencesystem.


%%%%%%%%%%%%%%%%%%%%%%%%%

\paragraph*{\textbf{3)}}

Betragt hvilesystemet for spin-$1$-partiklen og lad os som basis for de tre polarisationstilstande vælge de standard kartesiske basisvektorer for et tredimensionelt vektorrum, blot skrevet som firvektorer, altså
\begin{align} \label{eq:Opg1_Q3_PolarisationStatesInRestFrame}
    e^\mu (\xhat) = \FourRowMat{0}{1}{0}{0} \: , \quad
    e^\mu (\yhat) = \FourRowMat{0}{0}{1}{0} \: , \quad \text{og} \quad
    e^\mu (\zhat) = \FourRowMat{0}{0}{0}{1} \: .
\end{align}

Antag at partiklens treimpuls, $\vv{p}$, er langs $\zhat$-retningen. Vis at de generelle polarisationsvektorer i dette tilfælde bliver $(0,\, 1,\, 0,\, 0)$, $(0,\, 0,\, 1,\, 0)$ og $(\abs{\vv{p}}/m,\, 0,\, 0,\, E/m)$.


%%%%%%%%%%%%%%%%%%%%%%%%%

\paragraph*{\textbf{4)}}

Ofte bruges en relateret men en smule anderledes basis for poolarisationstilstandene for et massiv spin-$1$-felt, de såkaldte helicitetstilstande, hvilke indekseres med heltallet $\lambda$. Disse har formen
\begin{align} \label{eq:Opg1_Q4_HelicityStates}
    e^\mu(\lambda = \pm 1) &= \invsqrtTo \FourRowMat{0}{\mp 1}{-i}{0} \: , \quad \text{og} \quad
    e^\mu(\lambda = 0) = \FourRowMat{0}{0}{0}{1} \: .
\end{align}

Vis at hvis vi roterer disse polarisationstilstande omkring $\zhat$-aksen med en vinkel på $\theta$, da bliver de multipliceret med en faktor $\pexp{-i\lambda\theta}$. Argumentér for at denne rotation medfører, at $\lambda$ er egenværdien for den kvantemekaniske rotationsoperator, $J_z$, for hver af tilstandene.


%%%%%%%%%%%%%%%%%%%%%%%%%

\paragraph*{\textbf{5)}}

Heliciteten for en partikel er defineret som projektionen af partiklens spin på dens impuls, altså $\vv{J}\cdot\vv{p} / \abs{\vv{p}}$, hvor $\vv{J}$ er spinoperatoren og $\vv{p}$ er treimpulsen. Argumentér for at hvis vi booster fra hvilesystemet til et system, som bevæger sig med partiklen med hastighed $\vv{p}/E$, så bliver $J_z$ til helicitetsoperatoren.


%%%%%%%%%%%%%%%%%%%%%%%%%

\paragraph*{\textbf{6)}}

Vis at polarisationstilstandene med veldefineret helicitet for en spin-$1$-partikel, som bevæger sig langs $\zhat$-retningen, er
\begin{align} \label{eq:Opg1_Q6_StatesToShow}
    e^\mu(\lambda = \pm 1) = \mp \left(0,\, \frac{\vv{e}_x \pm i\vv{e}_y}{\sqrt{2}}\right) \: , \quad \text{og} \quad
    e^\mu(\lambda = 0) = \inv{m} \left(\abs{\vv{p}},\, 0,\, 0,\, E\right) \: ,
\end{align}
hvor $\vv{e}_x$ og $\vv{e}_y$ er de standard trekomponent basisvektorer.


%%%%%%%%%%%%%%%%%%%%%%%%%

\paragraph*{\textbf{7)}}

Antag at nogen beder dig opstille en teori for en partikel med et generelt spin $S > 1$. Ved brug af det, som du har lært i de foregående delopgaver, hvordan vil du da generalisere proceduren for at konstruere polarisationstilstande for et sådan felt? Skitser blot proceduren; \emph{lad være med} gå i gang med de detaljerede beregninger.


%%%%%%%%%%%%%%%%%%%%%%%%%%%%%%%%%%%%%%%%%%%%%%%%%%%%%%%%%%%%%%%%%%%%%%%%%%%%%%%%

\subsection{Besvarelse}

%%%%%%%%%%%%%%%%%%%%%%%%%

\paragraph[1) Polarisationsvektor for massivt spin-$1$-felt]{\textbf{1)}}

Idet at man udfører en Lorentztransformation vil denne påvirke både påvirke rum og tid, da disse bliver blandet under transformationen. Derfor er det ikke nok kun at beskrive polarisationen med en trevektor i rummet. Den mest generelle måde at beskrive både de rummelige og det tidslige koordinat i relativitetsteori er med en firvektor, hvorfor en firvektor også skal benyttes til at beskrive polarisationen, $e^\mu$. Ydermere giver det mening at benytte en firvektor til at beskrive polarisationen, da vi normalt ønsker vores teorier til at være Lorentzinvariant, hvilket en sammentrækning (eng: contraction) af firvektorer er.

Som beskrevet i opgavebeskrivelsen, så har partiklen tre frihedsgrader for kvantepolarisationstilstandene. Derfor må der findes tre ortogonale og dermed uafhængige polarisationsfirvektorer.


%%%%%%%%%%%%%%%%%%%%%%%%%

\paragraph[2) $e_\mu p^\mu = 0$ og $e_\mu e^\mu = 1$]{\textbf{2)}}

Lad $\Lambda_\nu^\mu$ betegne en general Lorentztransformation således, at en arbitrær firvektor $x^\nu$ ved transformation til det nye system ved $x'^\mu = \Lambda_\nu^\mu x^\nu$. Sammentrækningen mellem en arbitrær kontravariant firvektor $x^\mu$ og en arbitrær kovariant firvektor $y_\mu$ er i det transformerede referencesystem derved
\begin{align} \label{eq:Opg1_A2_ContractionsInvariantUnderLorentzTransformation}
\begin{split}
    y'_\mu x'^\mu &= \eta_{\mu\nu} y'^\nu x'^\mu \\
        &= \eta_{\mu\nu} \Lambda_\alpha^\nu y^\alpha \Lambda_\beta^\mu x^\beta \\
        &= \eta_{\mu\nu} \Lambda_\alpha^\nu \eta^{\alpha\gamma} y_\gamma \Lambda_\beta^\mu x^\beta \\
        &= \eta_{\mu\nu} \Lambda_\alpha^\nu \Lambda_\beta^\mu \eta^{\alpha\gamma} y_\gamma x^\beta \\
        &= \eta_{\beta\alpha} \eta^{\alpha\gamma} y_\gamma x^\beta \\
        &= \delta_\beta^\gamma y_\gamma x^\beta \\
        &= y_\beta x^\beta \: ,
\end{split}
\end{align}
hvor $\eta_{\alpha\beta}$ er den metriske tensor, hvilken for Minkowskirum er invariant under Lorentztransformation, $\eta_{\mu\nu} \Lambda_\beta^\mu \Lambda_\alpha^\nu = \eta_{\beta\alpha}$, og det er i det ovenstående blevet benyttet at $\eta_{\beta\alpha} \eta^{\alpha\gamma} = \delta_\beta^\gamma$, hvilket igen gør sig gældende i fladt Minkownskirum.

Af \cref{eq:Opg1_A2_ContractionsInvariantUnderLorentzTransformation} kan det ses, at en sammentrækning af to arbitrære firvektorer er Lorentzinvariant. Dermed, hvis vi insisterer på at $e_\mu p^\mu = 0$ og $e_\mu e^\mu = -1$ i ét referencesystem, da må de gøre sig gældende i ethvert referencesystem.


%%%%%%%%%%%%%%%%%%%%%%%%%

\paragraph[3) Generelle polarisationsvektorer]{\textbf{3)}}

Jeg vil starte med at notere, at der beregnes i naturlige enheder, altså er $\hbar = c = 1$.

Idet vi kender polarisationstilstandene i hvilesystemet, da kan vi benytte en Lorentztransformation til at finde de generelle polarisationstilstande. Vi benytter dermed en Lorentztransformation, som transformerer fra hvilesystemet til et system, hvori partiklen har treimpuls $\vv{p}$ i $\zhat$-retningen, altså $\abs{\vv{p}} = p_z$. Dette kan gøres på en af to måder, som begge giver samme resultat: 1) Vi benytter en Lorentztransformation fra hvilesystemet til referencesystemet med impuls $-\vv{p}$ med hensyn til partiklen, eller 2) vi benytter en invers Lorentztransformation fra partiklens hvilesystem til laboratoriesystemet, hvor partiklen har impuls $\vv{p}$. For begge tilfælde er Lorentztransformationen
\begin{align} \label{eq:Opg1_A3_LorentzTransformation}
    \Lambda_\mu^\nu &= \FourRowMat{\gamma & 0 & 0 & \beta\gamma}{0 & 1 & 0 & 0}{0 & 0 & 1 & 0}{\beta\gamma & 0 & 0 & \gamma}
        = \FourRowMat{\gamma & 0 & 0 & \abs{\vv{v}}\gamma}{0 & 1 & 0 & 0}{0 & 0 & 1 & 0}{\abs{\vv{v}}\gamma & 0 & 0 & \gamma}
        = \FourRowMat{E/m & 0 & 0 & \abs{\vv{p}}/m}{0 & 1 & 0 & 0}{0 & 0 & 1 & 0}{\abs{\vv{p}}/m & 0 & 0 & E/m}
\end{align}
hvor vi har benyttet, at $\vv{p} = \gamma m \vv{v}$ og at energien er givet ved $E = \gamma m$. Benytter vi nu ovenstående Lorentztransformation, \cref{eq:Opg1_A3_LorentzTransformation}, på polarisationstilstandene i hvilesystemet ($\mu$-systemet), \cref{eq:Opg1_Q3_PolarisationStatesInRestFrame}, får vi de generelle polarisationstilstande i det nye system ($\nu$-systemet) til
\begin{subequations} \label{eq:Opg1_A3_GeneralPolarisationVectors}
\begin{align}
    e^\nu (\xhat) &= \Lambda_\mu^\nu e^\mu (\xhat)
        = \FourRowMat{E/m & 0 & 0 & \abs{\vv{p}}/m}{0 & 1 & 0 & 0}{0 & 0 & 1 & 0}{\abs{\vv{p}}/m & 0 & 0 & E/m} \FourRowMat{0}{1}{0}{0}
        = \FourRowMat{0}{1}{0}{0}
        = e^\mu (\xhat) \: , \\
    e^\nu (\yhat) &= \Lambda_\mu^\nu e^\mu (\yhat)
        = \FourRowMat{E/m & 0 & 0 & \abs{\vv{p}}/m}{0 & 1 & 0 & 0}{0 & 0 & 1 & 0}{\abs{\vv{p}}/m & 0 & 0 & E/m} \FourRowMat{0}{0}{1}{0}
        = \FourRowMat{0}{0}{1}{0}
        = e^\mu (\yhat) \: , \quad \text{og} \\
    e^\nu (\zhat) &= \Lambda_\mu^\nu e^\mu (\zhat)
        = \FourRowMat{E/m & 0 & 0 & \abs{\vv{p}}/m}{0 & 1 & 0 & 0}{0 & 0 & 1 & 0}{\abs{\vv{p}}/m & 0 & 0 & E/m} \FourRowMat{0}{0}{0}{1}
        = \FourRowMat{\abs{\vv{p}}/m}{0}{0}{E/m} \: ,
\end{align}
\end{subequations}
hvilket var det, som skulle vises.


%%%%%%%%%%%%%%%%%%%%%%%%%

\paragraph[4) $\lambda$ egenværdi for rotationsoperator $J_z$ for helicitetstilstande]{\textbf{4)}}

% \begin{align}
%     e^\mu(\lambda = \pm 1) = \invsqrtTo \FourRowMat{0}{\mp 1}{-i}{0} \: , \quad \text{og} \quad
%     e^\mu(\lambda = 0) = \invsqrtTo \FourRowMat{0}{0}{0}{1} \: .
% \end{align}

% Vis at hvis vi roterer disse polarisationstilstande omkring $\zhat$-aksen med en vinkel på $\theta$, da bliver de multipliceret med en faktor $\pexp{-i\lambda\theta}$. Argumentér for at denne rotation medfører, at $\lambda$ er egenværdien for den kvantemekaniske rotationsoperator, $J_z$, for hver af tilstandene.

For en rotation med en vinkel $\theta$ omkring $\zhat$-aksen er rotationsmatricen
\begin{align}
    R_z(\theta) &= \FourRowMat{1 & 0 & 0 & 0}{0 & \cos(\theta) & - \sin(\theta) & 0}{0 & \sin(\theta) & \cos(\theta) & 0}{0 & 0 & 0 & 1} \: .
\end{align}
Benytter vi denne på helicitetstilstandene, \cref{eq:Opg1_Q4_HelicityStates}, får vi
\begin{subequations}
\begin{align}
    \begin{split}
        R_z(\theta) e^\mu(\lambda = \pm 1) &= \FourRowMat{1 & 0 & 0 & 0}{0 & \cos(\theta) & - \sin(\theta) & 0}{0 & \sin(\theta) & \cos(\theta) & 0}{0 & 0 & 0 & 1} \invsqrtTo \FourRowMat{0}{\mp 1}{-i}{0} \\
            &= \invsqrtTo \FourRowMat{0}{\mp \cos(\theta) + i \sin(\theta)}{\mp \sin(\theta) - i \cos(\theta)}{0} \\
            &= \invsqrtTo \FourRowMat{0}{-1 \big[\cos(\theta) \mp i \sin(\theta)\big]}{-i \big[\cos(\theta) \mp i\sin(\theta)\big]}{0} \\
            &= \invsqrtTo \FourRowMat{0}{\mp 1}{-i}{0} \big[ \cos(\theta) \mp \sin(\theta) \big] \\
            &= e^\mu(\lambda = \pm 1) \pexp{\mp i\theta} \\
            &= e^\mu(\lambda = \pm 1) \pexp{-i[\pm 1]\theta} \\
            &= e^\mu(\lambda = \pm 1) \pexp{-i\lambda\theta} \: ,
    \end{split}
\end{align}
idet at $\pexp{\pm i \theta} = \cos(\theta) \pm i \sin(\theta)$, og
\begin{align}
    \begin{split}
        R_z(\theta) e^\mu(\lambda = 0) &= \FourRowMat{1 & 0 & 0 & 0}{0 & \cos(\theta) & - \sin(\theta) & 0}{0 & \sin(\theta) & \cos(\theta) & 0}{0 & 0 & 0 & 1} \FourRowMat{0}{0}{0}{1} \\
            &= \FourRowMat{0}{0}{0}{1} \\
            &= e^\mu(\lambda = 0) \\
            &= e^\mu(\lambda = 0) \id \\
            &= e^\mu(\lambda = 0) \pexp{0} \\
            &= e^\mu(\lambda = 0) \pexp{-i \lambda \theta} \: .
    \end{split}
\end{align}
\end{subequations}

Ergo et det vist, at under rotation omkring $\zhat$-aksen med en vinkel $\theta$, så bliver
\begin{align} \label{eq:Opg1_A4_RotationOfPolarisationStates}
    e^\mu(\lambda) \xrightarrow{R_z(\theta)} e^\mu(\lambda) \pexp{-i\lambda\theta} \: .
\end{align}

Siden rotationen omkring $\zhat$-aksen af de tre helicitetstilstande ikke ændrer retningen af deres stråler, da er tilstandene egentilstande for rotation omkring $\zhat$-aksen.

Den kvantemekaniske rotationsoperator for en rotation omkring $\zhat$-aksen er givet som $\mathcal{D}_\zhat (\theta) = \pexp{-iJ_z\theta}$ i naturlige enheder ($\hbar = c = 1$). Dermed må vi fra \cref{eq:Opg1_A4_RotationOfPolarisationStates} have, at
\begin{align}
    \mathcal{D}_\zhat (\theta) e^\mu(\lambda) &= \pexp{-iJ_z\theta} e^\mu(\lambda)
        = \pexp{-i\lambda\theta} e^\mu(\lambda) \: ,
\end{align}
for $\lambda = 0,\pm 1$ og dermed at $J_z e^\mu(\lambda) = \lambda e^\mu(\lambda)$, hvilket tydeligere kan ses fra Taylorudvidelsen af eksponentialfunktionerne
\begin{align}
    \left( \sum_{n=0}^\infty \inv{n!}(-iJ_z\theta) \right) e^\mu(\lambda) &= \left( \sum_{n=0}^\infty \inv{n!}(-i\lambda\theta) \right) e^\mu(\lambda) \: ,
\end{align}
hvilket skal være sandt for en et vilkårligt $\theta$.


%%%%%%%%%%%%%%%%%%%%%%%%%

\paragraph[5) $J_z$ bliver helicitetsoperator]{\textbf{5)}}

I hvile kun z-retningen => Ligning

% Vi ved at heliciteten for en partikel er defineret som projektionen af partiklens spin på dens impuls, altså $h = \vv{J}\cdot\vv{p} / \abs{\vv{p}}$, hvor $\vv{J}$ er spinoperatoren og $\vv{p}$ er treimpulsen. Vi kan nu blot vælge at rotere vores system således, at partikel bevæger sig langs $\zhat$-aksen. Når vi så booster (langs $\zhat$-aksen), da vil vi få at partiklen blot får 

\begin{align}
    h &= \vv{J} \cdot \frac{\vv{p}}{\abs{\vv{p}}}
        = \vv{J} \cdot \frac{\vv{p}_z}{\abs{\vv{p}_z}}
        = \vv{J} \cdot \zhat
        = J_z \: .
\end{align}

% Argumentér for at hvis vi booster fra hvilesystemet til et system, som bevæger sig med partiklen med hastighed $\vv{p}/E$, så bliver $J_z$ til helicitetsoperatoren.


%%%%%%%%%%%%%%%%%%%%%%%%%

\paragraph[6) Polarisationstilstande for spin-$1$-partikel i bevægelse langs $\zhat$]{\textbf{6)}}

Polarisationstilstandene med veldefineret helicitet for en spin-1-partikel er givet i \cref{eq:Opg1_Q4_HelicityStates}, og for at finde ud af, hvordan disse ser ud, når partiklen bevæger sig langs $\zhat$-retningen med treimpuls $\vv{p}$, da benytter vi det inverse Lorentzboost fra \cref{eq:Opg1_A3_LorentzTransformation},
\begin{align}
    \Lambda_\mu^\nu &= \FourRowMat{E/m & 0 & 0 & \abs{\vv{p}}/m}{0 & 1 & 0 & 0}{0 & 0 & 1 & 0}{\abs{\vv{p}}/m & 0 & 0 & E/m} \: ,
\end{align}
hvorved vi får
\begin{subequations}
\begin{align}
    e^\nu(\lambda = \pm 1) &= \Lambda_\mu^\nu e^\mu(\lambda = \pm 1)
        = \FourRowMat{E/m & 0 & 0 & \abs{\vv{p}}/m}{0 & 1 & 0 & 0}{0 & 0 & 1 & 0}{\abs{\vv{p}}/m & 0 & 0 & E/m} \invsqrtTo \FourRowMat{0}{\mp 1}{-i}{0}
        = \invsqrtTo \FourRowMat{0}{\mp 1}{-i}{0} \\
    e^\nu(\lambda = 0) &= \Lambda_\mu^\nu e^\mu(\lambda = 0)
        = \FourRowMat{E/m & 0 & 0 & \abs{\vv{p}}/m}{0 & 1 & 0 & 0}{0 & 0 & 1 & 0}{\abs{\vv{p}}/m & 0 & 0 & E/m} \FourRowMat{0}{0}{0}{1}
        = \FourRowMat{\abs{\vv{p}}/m}{0}{0}{E/m} \: .
\end{align}
\end{subequations}
Dermed er det vist, at når vi benytter et Lorentzboost således, at spin-$1$-partikel nu bevæger sig med impuls $\vv{p}$ i $\zhat$-retningen, da bliver helicitetsegentilstandene fra \cref{eq:Opg1_Q4_HelicityStates}
\begin{align}
    e^\mu(\lambda = \pm 1) = \mp \left(0,\, \frac{\vv{e}_x \pm i\vv{e}_y}{\sqrt{2}}\right) \: , \quad \text{og} \quad
    e^\mu(\lambda = 0) = \inv{m} \left(\abs{\vv{p}},\, 0,\, 0,\, E\right) \: ,
\end{align}
hvor $\vv{e}_x$ og $\vv{e}_y$ er de standard trekomponent basisvektorer, hvilket skulle vises. At disse stadig er helicitetsegentilstanden kan ses ved, at et boost langs $\zhat$-retningen commuterer med $J_z$, så når tilstandene før boostet var egentilstande for helicitet vil de efterfølgende også være det. Dette kommer af \textbf{4)} og \textbf{5)}.


%%%%%%%%%%%%%%%%%%%%%%%%%

\paragraph[7) Teori for spin-$S$ partikel med $S > 1$]{\textbf{7)}}

\ldots

% For a general spin S > 1 particle, we still need an object to describe the internal degrees of freedom. This object has be well defined under Lorenz transformations and there needs to be 2S+1 linearly independent objects like this. For S = 2 a four vector does not have sufficient degrees of freedom, so a rank 2 tensor is needed gµν, and in general one would expect that a rank S tensor is needed to describe a spin S field. This gives too many degrees of freedom, so some assumptions would have to be made, like a form of normalization or symmetry of the tensor. Next, to a basis might be chosen due to some nice rotation properties in the rest frame, and then the general can be found via a Lorentz boost.

% Antag at nogen beder dig opstille en teori for en partikel med et generelt spin $S > 1$. Ved brug af det, som du har lært i de foregående delopgaver, hvordan vil du da generalisere proceduren for at konstruere polarisationstilstande for et sådan felt? Skitser blot proceduren; \emph{lad være med} gå i gang med de detaljerede beregninger.


%%%%%%%%%%%%%%%%%%%%%%%%%%%%%%%%%%%%%%%%%%%%%%%%%%%%%%%%%%%%%%%%%%%%%%%%%%%%%%%%%%%%%

\end{document}