\documentclass[../main.tex]{subfiles}

\begin{document}

%%%%%%%%%%%%%%%%%%%%%%%%%%%%%%%%%%%%%%%%%%%%%%%%%%%%%%%%%%%%%%%%%%%%%%%%%%%%%%%%%%%%%

\section{Kvanteelektrodynamik med elektroner}

Betragt Lagrangedensiteten for elektroner, hvilken kaldes Dirac Lagrangedensiteten,
\begin{align} \label{eq:Opg4_DiracLagrangian}
    \L &= \psibar (i\slashed{\partial} - m) \psi \: ,
\end{align}
hvor $\slashed{\partial} = \gamma^\mu\partial_\mu$ og $\gamma_\mu$ er Diracmatricerne af dimension $4 \times 4$.


%%%%%%%%%%%%%%%%%%%%%%%%%%%%%%%%%%%%%%%%%%%%%%%%%%%%%%%%%%%%%%%%%%%%%%%%%%%%%%%%

\paragraph*{\textbf{1)}}

Betragt transformationen $\psi \rightarrow \bexp{-ie\theta(x)}\psi$, hvor $e$ er elektronens ladning og $\theta(x)$ er en skalarfunktion, som afhænger af rum og tid (betegnet samlet ved koordinatet $x = (\vv{x},\, t)$). Benyt minimal substitution for et felt med ladning $q = -e$,
\begin{align} \label{eq:Opg4_Q1_MinimalSubstitution}
    \partial^\mu \rightarrow D^\mu = \partial^\mu - ieA^\mu  \: ,
\end{align}
i Diracs Lagrangedensitet og vis at denne nu er gaugeinvariant. Hvad er den nødvendige transformation for $A^\mu$?


%%%%%%%%%%%%%%%%%%%%%%%%%

\paragraph*{\textbf{2)}}

I matematik betegner $\U(1)$ (den $1 \times 1$ unitære matrixgruppe) gruppen af de komplekse tal på enhedscirklen (modulo $1$). Forklar hvorfor det giver mening at betegne gauge transformationen i \textbf{1)} og den tilhørende teori, kvanteelektrodynamikken (QED), som en $\U(1)$-gaugeteori.


%%%%%%%%%%%%%%%%%%%%%%%%%

\paragraph*{\textbf{3)}}

Vis at Lagrangedensiteten for vekselvirkningen (eng: interaction Lagrangian density), hvilken vi opnår ved brug af gaugeinvariansprincippet (eng: principle of gauge invariance), er på strømvektorfeltsformen
\begin{align} \label{eq:Opg4_Q3_interactionLagrangian}
    \L_I &= e \psibar \gamma_\mu \psi A^\mu = - J_\mu A^\mu \: .
\end{align}

Vis ydermere at $J_\mu(x) = - e \psibar(x) \gamma_\mu \psi(x)$ er en bevaret strøm i den forstand at $\partial^\mu J_\mu(x) = 0$. Hvordan transformerer Lagrangedensiteten for vekselvirkningen under en gaugetransformation? Tillader denne transformation teorien at forblive gaugeinvarient?

(Hint: Betragt virkningen (eng: action) genereret af Lagrangedensiteten for vekselvirkningen.)


%%%%%%%%%%%%%%%%%%%%%%%%%

\paragraph*{\textbf{4)}}

Diracfeltet kan ekspanderes i normal modes ifølge
\begin{align} \label{eq:Opg4_Q4_FieldExpansionNormalModes}
    \psi_\alpha (x) &= \sum_\lambda \int \frac{\dd^3\vv{p}}{(2\pi)^3} \invsqrt{2\omega_{\vv{p}}} \left( b_{\vv{p},\lambda} u(\vv{p},\lambda)_\alpha \bexp{-ipx} + d_{\vv{p},\lambda}\dagger v(\vv{p},\lambda)_\alpha \bexp{ipx} \right) \: ,
\end{align}
hvor $\lambda$ er heliciteten og $px = p_\mu x^\mu$. $b$ og $d$ er operatorer som kreerer hhv. fermioniske partikler og antipartikler med en given impuls $\vv{p}$ og helicitet $\lambda$. Vis at
\begin{align} \label{eq:Opg4_Q4_FourPartsOfTheCurrent}
    -e \psibar \gamma_\mu \psi &= \sum_{\vv{p},\vv{p}',\lambda,\lambda'} \sum_{n=1}^4 j_\mu^{(n)}(p,\lambda,p',\lambda',x)
        = \sum_{n=1}^4 j_\mu^{(n)}(x) \: ,
\end{align}
og find de fire led $j_\mu^{(n)}$ eksplicit.


%%%%%%%%%%%%%%%%%%%%%%%%%

\paragraph*{\textbf{5)}}

Udled de følgende matrixelementer ved brug af strømmen fra \textbf{4)}:
\begin{subequations} \label{eq:Opg4_Q5_TransitionCurrents}
\begin{align}
    \mel**{\mathrm{e}^-,\, \vv{p}',\, \lambda'}{j_\mu^{(1)}}{\mathrm{e}^-,\, \vv{p},\, \lambda}
        &= -e\bar{u}(\vv{p}',\lambda') \gamma_\mu u(\vv{p},\lambda) \pexp{i[p' - p]x} \: , 
        \label{eq:Opg4_Q5_TransitionCurrent1} \\
    \mel**{\mathrm{e}^+,\, \vv{p}',\, \lambda'}{j_\mu^{(2)}}{\mathrm{e}^+,\, \vv{p},\, \lambda}
        &= e\bar{v}(\vv{p},\lambda) \gamma_\mu v(\vv{p}',\lambda') \pexp{i[p' - p]x} \: ,
        \label{eq:Opg4_Q5_TransitionCurrent2} \\
    \mel**{0}{j_\mu^{(3)}}{\mathrm{e}^-,\, \vv{p},\, \lambda;\, \mathrm{e}^+,\, \vv{p}',\, \lambda'}
        &= -e\bar{v}(\vv{p}',\lambda') \gamma_\mu u(\vv{p},\lambda) \pexp{-i[p' + p]x} \: , 
        \label{eq:Opg4_Q5_TransitionCurrent3} \\
    \mel**{\mathrm{e}^-,\, \vv{p}',\, \lambda';\, \mathrm{e}^+,\, \vv{p},\, \lambda}{j_\mu^{(4)}}{0}
        &= -e\bar{u}(\vv{p}',\lambda') \gamma_\mu v(\vv{p},\lambda) \pexp{i[p' + p]x} \: .
        \label{eq:Opg4_Q5_TransitionCurrent4}
\end{align}
\end{subequations}

Giv en fysisk fortolkning af de fire udtryk (det kan være en idé at tegne hver af udtrykkene som en del af et Feynmandiagram).


%%%%%%%%%%%%%%%%%%%%%%%%%

\paragraph*{\textbf{6)}}

Matrixelementerne i \textbf{5)} kaldes overgangsstrømme (eng: transition currents), $J_\mu^{fi}(x)$. Vis eksplicit at de er bevarede strømme ved at benytte gradientoperatoren $\partial^\mu$ på hver af dem.


%%%%%%%%%%%%%%%%%%%%%%%%%

\paragraph*{\textbf{7)}}

Argumentér for at når vi beregner fysiske processer i QED med vekselvirkningen fra \cref{eq:Opg4_Q3_interactionLagrangian}, så vil vi altid få led på formen $J_\mu^{fi}(0)\epsilon^\mu(\sigma)$, hvor $\epsilon^\mu(\sigma)$ er en fotons polarisationstilstand indiceret ved $\sigma$. Argumentér yderligere for at når vi kvadrerer amplituden for en given proces, så får vi udtryk på formen
\begin{align}
    \abs{J_\mu^{fi}(0)\epsilon^\mu(\sigma)}^2 &= \epsilon^\mu(\sigma)^* \epsilon^\nu(\sigma) J_\mu^{fi}(0)^* J_\nu^{fi}(0) \: .
\end{align}


%%%%%%%%%%%%%%%%%%%%%%%%%

\paragraph*{\textbf{8)}}

Ofte bliver vi nødt til at summere over uobserverede fotonpolarisationstilstande; vi summerer altså over $\sigma$,
\begin{align}
    \left[ \sum_\sigma \epsilon^\mu(\sigma)^* \epsilon^\nu(\sigma) \right] J_\mu^{fi}(0)^* J_\nu^{fi}(0) \: .
\end{align}

Antag at impulsoverførslen (eng: momentum transfer) for strømmen $J_\mu^{fi}$ er langs $\zhat$-retningen, altså $q = (q_0,\, 0,\, 0,\, q_0)$. Vis at vi for en reel foton har
\begin{align} \label{eq:Opg4_Q8_IdentityWithSumOfPolarisation}
    \sum_\sigma \epsilon^\mu(\sigma)^* \epsilon^\nu(\sigma) &= \delta_1^\mu \delta_1^\nu + \delta_2^\mu \delta_2^\nu \: ,
\end{align}
hvor vi har benyttet den relativistiske konvention om at $\mu = 0$ betegner tidsretningen og $\mu = 1,2,3$ betegner de rummelige retninger hhv. $\xhat$-, $\yhat$- og $\zhat$-retningen.


%%%%%%%%%%%%%%%%%%%%%%%%%

\paragraph*{\textbf{9)}}

Vis at for en reel foton gør det sig gældende, at
\begin{align} \label{eq:Opg4_Q9_Result}
    \left[ \sum_\sigma \epsilon^\mu(\sigma)^* \epsilon^\nu(\sigma) \right] J_\mu^{fi}(0)^* J_\nu^{fi}(0) &= - g^{\mu\nu} J_\mu^{fi}(0)^* J_\nu^{fi}(0) \: .
\end{align}


%%%%%%%%%%%%%%%%%%%%%%%%%

\paragraph*{\textbf{10)}}

Under hvilke omstændigheder medfører \cref{eq:Opg4_Q9_Result} følgende udtryk
\begin{align} \label{eq:Opg4_Q10_ExpressionToBeTrue}
    \sum_\sigma \epsilon^\mu(\sigma)^* \epsilon^\nu(\sigma) &= -g^{\mu\nu} \: ?
\end{align}
Hvilke yderligere led kunne opstå for en reel foton, hvor $q^2 = 0$? Hvilke yderligere led kunne opstå for en virtuel foton, hvor $q^2 \ne 0$?


%%%%%%%%%%%%%%%%%%%%%%%%%

\paragraph*{\textbf{11)}}

Betragt spredningen af elektroner på myoner, $\mathrm{e}^- + \mu^- \rightarrow \mathrm{e}^- + \mu^-$. Tegn Feynmandiagrammet til anden orden for denne proces.


%%%%%%%%%%%%%%%%%%%%%%%%%

\paragraph*{\textbf{12)}}

Vis at S-matricen til anden orden for elektron-myon-spredningsprocessen kan skrives som
\begin{align} \label{eq:Opg4_Q12_SMatrixSecondOrder}
    S_{fi}^{(2)} &= (-i)^2 \int \dd^4 x_1\, \dd^4 x_2\, J_\mu^{\mathrm{e}^-}(x_1) J_\nu^{\mu^-}(x_2) \mel**{0}{T\left[A^\mu(x_1)A^\nu(x_2)\right]}{0} \: .
\end{align}
Opskriv eksplicit elektron- og myonstrømmen (eng: electron and muon current) ved brug af overgangsstrømmene fra \textbf{5)}. Sørg for at mærke alle dele nødvendige for at specificere start- og sluttilstandene korrekt.


%%%%%%%%%%%%%%%%%%%%%%%%%

\paragraph*{\textbf{13)}}

Ved at benytte Klein-Gordon-propagatoren som analogi, argumentér for at vi kan benytte den følgende form for fotonpropagatoren
\begin{align} \label{eq:Opg4_Q13_PhotonPropagator}
    G^{\mu\nu}(q) &= \int \dd^4 x \pexp{iqx} \mel**{0}{T\left[A^\mu(x)A^\nu(0)\right]}{0}
        = \frac{-ig^{\mu\nu}}{q^2}
\end{align}
i udtrykket for $S_{fi}^{(2)}$.


%%%%%%%%%%%%%%%%%%%%%%%%%

\paragraph*{\textbf{14)}}

Definer som sædvanligt $S_{fi}^{(2)} = - i \M_{fi} (2\pi)^4 \delta^4(p_f - p_i)$. Vis at
\begin{align} \label{eq:Opg4_Q14_AmplitudeMfi}
    - i \M_{fi} &= \left( -i J_\mu^{\mathrm{e}^-}(0) \right) \frac{-ig^{\mu\nu}}{q^2} \left( -i J_\nu^{\mu^-}(0) \right) \: .
\end{align}
Udtryk impulsoverførslen $q$ som funktion af start- og sluttilstandsimpulsen.


%%%%%%%%%%%%%%%%%%%%%%%%%

\paragraph*{\textbf{15)}}

Når vi kvadrerer amplituden, summerer over sluttilstande og tager gennemsnittet af starttilstandene får vi
\begin{align} \label{eq:Opg4_Q15_AmplitudeOfMByAllMomenta}
    \inv{4} \sum_{\mathrm{spins}} \abs{\M_{fi}}^2 &= \frac{8e^4}{q^4} \big( \left[ p_1 p_2 \right] \left[ p_3 p_4 \right] + \left[ p_1 p_4 \right] \left[ p_2 p_3 \right] \big)
\end{align}
i grænsen hvor alle impulser er meget større end masserne af både elektronen og myonen. Her er skalarproduktet af firvektorerne betegnet som $p p' = p_\mu p'^\mu = p^\mu p'_\mu$. Vis at i denne grænse kan vi skrive
\begin{align} \label{eq:Opg4_Q15_AmplitudeOfMByMomentaSquared}
    \inv{4} \sum_{\mathrm{spins}} \abs{\M_{fi}}^2 &= \frac{8e^4}{q^4} \left( \left[ p_1 p_2 \right]^2 + \left[ p_1 p_4 \right]^2 \right) \: .
\end{align}

Vi betegner nu vinklen mellem den indkomne og den udadgående elektron i massemidtpunktssystemet (eng: center-of-mass frame) $\theta$. Vis at
\begin{align} \label{eq:Opg4_Q15_AmplitudeOfMByAngleInCMFrame}
    \inv{4} \sum_{\mathrm{spins}} \abs{\M_{fi}}^2 &= \frac{2e^4}{\sin^4\left(\frac{\theta}{2}\right)} \left[ 1 + \cos^4\pfrac{\theta}{2} \right] \: .
\end{align}


%%%%%%%%%%%%%%%%%%%%%%%%%

\paragraph*{\textbf{16)}}

Vis at tværnittet (eng: cross section) i massemidtpunktssystemet er givet ved
\begin{align} \label{eq:Opg4_Q16_CrossSection}
    \left(\dif{\sigma}{\Omega}\right)_\mathrm{CM} &= \frac{\alpha^2}{2E_\mathrm{CM}^2} \frac{1 + \cos^4\pfrac{\theta}{2}}{\sin^4\pfrac{\theta}{2}} \: ,
\end{align}
hvor $E_\mathrm{CM}$ er den totale energi i massemidtpunktssystemet og $\alpha$ er finstrukturkonstanten. Slå ''Tværsnit for Rutherfordspredning'' op et sted (f.eks. i lærebog eller online). Sammenlign det fundne resultat (\cref{eq:Opg4_Q16_CrossSection}) med udtrykket for Rutherfordspredningen. Vis at udtrykkene har samme funktionsopførsel (eng: functional behavior) i grænsen $\theta \rightarrow 0$.


%%%%%%%%%%%%%%%%%%%%%%%%%

\paragraph*{\textbf{17)}}

Brug alle informationerne, som du har fået fra foregående delopgaver, til at nedskrive Feynmanreglerne for QED for træiveau-Feynmandiagrammer (aka. \emph{uden} loops), altså skal du finde ud af hvilke faktorer, som er forbundet med start- og sluttilstande for partikler og antipartikler, hvilke faktorer, som er forbundet med knudepunkter (eng: vertices), og hvilke faktorer, som er forbundet med propagatorer.


%%%%%%%%%%%%%%%%%%%%%%%%%%%%%%%%%%%%%%%%%%%%%%%%%%%%%%%%%%%%%%%%%%%%%%%%%%%%%%%%

\subsection{Besvarelse}

%%%%%%%%%%%%%%%%%%%%%%%%%

\paragraph[1) Gaugeinvariant Lagrangedensitet for elektroner]{\textbf{1)}}

Vi betragter Diracs Lagrangedensitet fra \cref{eq:Opg4_DiracLagrangian}, hvor princippet for minimal substitution med $q = -e$ er blevet benyttet idet Lagrangedensiteten ellers ikke ville være invariant under transformationen $\psi \rightarrow \bexp{-ie\theta(x)}\psi$ for $x = (\Vec{x},\, t)$ værende en firvektor (der vil være et ekstra led fra differentialoperatoren):
\begin{align} \label{eq:Opg4_A1_LagrangeDensityWithMinimalSubstitution}
    \L &= \psibar (i \slashed{\partial} - m)\psi
        \rightarrow \psibar (i \slashed{D} - m)\psi
        = \psibar \big( i \gamma_\mu [\partial^\mu - ieA^\mu] - m \big)\psi \: ,
\end{align}
hvor $D^\mu$ er den gaugekovariante differentialoperator, $\slashed{\partial} = \gamma_\mu \partial^\mu$ og $\psibar = \psi\dagger \gamma^0$ er den Diracadjungerede (Dirac adjunct).
Idet minimal substitutionen indføres, indføres også et nyt felt, $A^\mu$, hvilket også transformerer under førnævnte transformation, $A^\mu \rightarrow A'^\mu$.

Opgaven kan nu løses på en af følgende måder: Antag at transformationen af $A^\mu$ er kendt og vis at denne gør Lagrangedensiteten invariant, eller antag at Lagrangedensiteten er invariant og find transformationen af $A^\mu$. Her vælges den sidste af de to metoder.

Vi betragter transformationen
\begin{align} \label{eq:Opg4_A1_U1TransformationOfPsi}
    \psi &\rightarrow \psi' = \U \psi = \pexp{-ie\theta[x]}\psi
\end{align}
for $x = (\Vec{x},\, t)$ værende en firvektor. Det konjugerede felt transformerer dermed som\footnote{Hermitisk konjugering af et produkt af operatorer er produktet af hver operator Hermitisk konjugeret og i omvendt orden, altså $(XY)\dagger = Y\dagger X\dagger$.}
\begin{align}
    \psibar &\rightarrow \overline{\psi'}
        = (\psi')\dagger \gamma^0
        = (\U \psi)\dagger \gamma^0
        = \psi\dagger \U\dagger \gamma^0
        = \psi\dagger \gamma^0 \U\dagger
        = \psibar\pexp{ie\theta[x]} \: .
\end{align}
Benyttes disse transformationer fås Lagrangedensitetens transformation til (hvor $\theta(x)$ betegnes $\theta$ for overskuelighed)
\begin{align}
\begin{split}
    \L \rightarrow \L' &= \overline{\psi'} \big(i \gamma_\mu [\partial^\mu - ieA'^\mu] - m \big) \psi' \\
        &= \psibar \pexp{ie\theta} \big(i \gamma_\mu [\partial^\mu - ieA'^\mu] - m \big) \pexp{-ie\theta} \psi \\
        &= \psibar \pexp{ie\theta} \bigg(i \gamma_\mu \Big[ \big\{ -ie\pexp{-ie\theta}\left(\partial^\mu \theta\right) + \pexp{-ie\theta}\partial^\mu \big\} \\
            &\qquad\qquad\qquad\qquad\quad - ieA'^\mu \pexp{-ie\theta} \Big] - \pexp{-ie\theta} m \bigg) \psi \\
        &= \psibar \id \big(\gamma_\mu e\left( \partial^\mu \theta \right) + i \gamma_\mu \partial^\mu - m \big) \psi + \psibar \pexp{ie\theta} \gamma_\mu eA'^\mu \pexp{-ie\theta} \psi
\end{split}
\end{align}

Lagrangedensiteten er invariant hvis $\L' = \L$, hvorfor
\begin{align}
\begin{split}
    0 &= \L' - \L \\
        &= \psibar \big(\gamma_\mu e\left( \partial^\mu \theta \right) + i \gamma_\mu \partial^\mu - m \big) \psi + \psibar \pexp{ie\theta} \gamma_\mu eA'^\mu \pexp{-ie\theta} \psi \\
            &\qquad - \psibar \big( i \gamma_\mu [\partial^\mu - ieA^\mu] - m \big)\psi \\
        &= \psibar e \gamma_\mu \big[\left( \partial^\mu \theta \right) - A^\mu + \pexp{ie\theta} A'^\mu \pexp{-ie\theta} \big] \psi \: .
\end{split}
\end{align}
Dette medfører, at
\begin{align}
    A^\mu - \left( \partial^\mu \theta \right) &= \pexp{ie\theta} A'^\mu \pexp{-ie\theta} \: ,
\end{align}
idet $\psi$ er en arbitrær bølgefunktion og ladningen $e$ kan være forskellig fra $0$.
Vi kan dermed opskrive $A'^\mu$ som
\begin{align}
\begin{split} \label{eq:Opg4_A1_transformationOfA^muCalculation}
    A'^\mu &= \id A'^\mu \id \\
        &= \pexp{-ie\theta}\pexp{ie\theta} A'^\mu \pexp{-ie\theta}\pexp{ie\theta} \\
        &= \pexp{-ie\theta} \big[ A^\mu - \left( \partial^\mu \theta \right) \big] \pexp{ie\theta} \\
        &= \pexp{-ie\theta} \pexp{ie\theta} \big[ A^\mu - \left( \partial^\mu \theta \right) \big] \\
        &= \id \big[ A^\mu - \left( \partial^\mu \theta \right) \big] \\
        &= A^\mu - \left( \partial^\mu \theta \right) \: .
\end{split}
\end{align}
For at Lagrangedensiteten er invariant under transformationen $\psi \rightarrow \pexp{-ie\theta[x]} \psi$ skal transformationen af $A^\mu$ altså være (\cref{eq:Opg4_A1_transformationOfA^muCalculation})
\begin{align} \label{eq:Opg4_A1_transformationOfA^mu}
    A^\mu &\rightarrow A'^\mu = A^\mu - \left[ \partial^\mu \theta(x) \right] \: .
\end{align}


%%%%%%%%%%%%%%%%%%%%%%%%%

\paragraph[2) QED som $\U(1)$-gaugeteori]{\textbf{2)}}

Transformationen 
\begin{align}
    \psi &\rightarrow \psi' = \U \psi = \pexp{-ie\theta[x]}\psi
\end{align}
for $x = (\Vec{x},\, t)$ værende en firvektor, er en $\U(1)$-gaugeteori, da transformationen har modulo $1$
\begin{align}
    \abs{\U} &= \U\U^* = \pexp{-ie\theta[x]} \pexp{ie\theta[x]} = \id \: .
\end{align}
Idet transformationen er en $\U(1)$-transformation giver det mening, at teorien dannet ud fra denne transformation kaldes ved transformationens symmetri, altså her en $\U(1)$-teori.


%%%%%%%%%%%%%%%%%%%%%%%%%

\paragraph[3) Lagrangedensiteten for vekselvirkningen]{\textbf{3)}}

Vekselvirkningeledet i Lagrangedensiteten er det led, hvor de to felter interagerer, så med Lagrangedensiteten fra \cref{eq:Opg4_DiracLagrangian} med minimal substitution \cref{eq:Opg4_Q1_MinimalSubstitution} fås \cref{eq:Opg4_A1_LagrangeDensityWithMinimalSubstitution}
\begin{align} \label{eq:Opg4_A3_LagrangeDensityWithMinimalSubstitutionCalculated}
\begin{split}
    \L &= \psibar \big( i \gamma_\mu [\partial^\mu - ieA^\mu] - m \big)\psi \\
        &= i \psibar \gamma_\mu \partial^\mu \psi - m \psibar \psi + e \psibar \gamma_\mu \psi A^\mu \\
        &= \L_0 + \L_I \: ,
\end{split}
\end{align}
hvor $\L_0$ er den frie Lagrangedensitet og $\L_I$ er Lagrangedensiteten for vekselvirkningen, hvilken er
\begin{align} \label{eq:Opg4_A3_interactionLagrangian}
    \L_I &= e \psibar \gamma_\mu \psi A^\mu = - J_\mu A^\mu \: ,
\end{align}
når $J_\mu(x) = - e \psibar(x) \gamma_\mu \psi(x)$.
\\

Som det næste skal det vises, at $J_\mu(x)$ er en bevaret strøm, hvilket vil sige, at $\partial^\mu J_\mu = 0$. Dette vises ved først at differentiere strømmen og benytte Diracligningen $0 = (i \gamma^\mu \partial_\mu - m)\psi$ og dens Hermitiskkonjugerede.

For at beregne den Hermitiskkonjugerede af Diracligningen, da starter vi med at beregne den Hermitiskkonjugerede af $\gamma^0 \gamma^\mu$, hvilket gøres ud fra kendskab til udledningen af Diracmatricerne, nemlig at
\begin{align}
    \gamma^0 &= \beta
        \qquad \text{og} \qquad
    \gamma^i = \beta \alpha_i \quad \forall i \in \{1,2,3\} \: ,
\end{align}
hvor $\alpha_i$ og $\beta$ er Hermitiske matricer. Dermed får vi at
\begin{align} \label{eq:Opg4_A3_gamma0*(gamma^mu)^dagger=gamma^mu*gamma0}
    \gamma^0 \left(\gamma^\mu\right)\dagger &= \TwoRowMat{\gamma^0 \left(\gamma^0\right)\dagger}{\gamma^0 \left(\gamma^i\right)\dagger}
        = \TwoRowMat{\beta \left(\beta\right)\dagger}{\beta \left(\beta \alpha_i\right)\dagger}
        = \TwoRowMat{\beta \beta}{\beta \left(\alpha_i\right)\dagger \left(\beta\right)\dagger}
        = \TwoRowMat{\beta \beta}{\beta \alpha_i \beta}
        = \gamma^\mu \gamma^0 \: ,
\end{align}
og betragter man $\gamma^0 = \textrm{diag}(\id,-\id)$ kan det ses, at $(\gamma^0)^2 = \id$. Benytter vi nu dette til at beregne den Hermitiskkonjugerede af Diracligning fås
\begin{align} \label{eq:Opg4_A3_FromHermitianOfDiracEquation}
\begin{split}
    0 &= \Big[ \left( i \gamma^\mu \partial_\mu - m \right) \psi \Big]\dagger \\
        &= \left( i \gamma^\mu \big[\partial_\mu \psi\big] - m \psi \right)\dagger \\
        &= - i \big[ \partial_\mu \psi\dagger \big] \left(\gamma^\mu\right)\dagger - m \psi\dagger \\
        &= - i \big[ \partial_\mu \psi\dagger \big] \id \left(\gamma^\mu\right)\dagger - m \psi\dagger \id \\
        &= - i \big[ \partial_\mu \psi\dagger \big] \gamma^0 \gamma^0 \left(\gamma^\mu\right)\dagger - m \psi\dagger \gamma^0 \gamma^0 \\
        &= \Big\{ - i \big[ \partial_\mu \psibar \big] \gamma^\mu - m \psibar \Big\} \gamma^0 \\
        &= -i \Big\{ \big[ \partial_\mu \psibar \big] \gamma^\mu - i m \psibar \Big\} \gamma^0 \\
    \Rightarrow \big( \partial_\mu \psibar \big) \gamma^\mu &= i m \psibar \: .
\end{split}
\end{align}
Differentierer vi nu strømmen og benytter vi \cref{eq:Opg4_A3_FromHermitianOfDiracEquation} samt at vi fra Diracligningen har at $\gamma^\mu \partial_\mu \psi = - i m \psi$, så får vi
\begin{align} \label{eq:Opg4_A3_CurrentIsConserved}
\begin{split}
    \partial^\mu J_\mu &= \partial^\mu \big[ - e \psibar(x) \gamma_\mu \psi(x) \big] \\
        &= \partial_\mu \big[ - e \psibar \gamma^\mu \psi \big] \\
        &= - e \Big[ \big( \partial_\mu \psibar \big) \gamma^\mu \psi + \psibar \gamma^\mu \big( \partial_\mu \psi \big) \Big] \\
        &= - e \Big[ \big( i m \psibar \big) \psi + \psibar \big( - i m \psi \big) \Big] \\
        &= - e \big[ i m \psibar \psi + - i m \psibar \psi \big] \\
        &= 0 \: ,
\end{split}
\end{align}
hvorfra det kan ses, at strømmen er bevaret.
\\

Sidst kigger vi på transformationen af vekselvirkningen. Transformerer vi ved brug af de givne transformationer fra \textbf{1)} får vi
\begin{align} \label{eq:Opg4_A3_InteractionLagrangianTransformed}
\begin{split}
    \L_I' &= e \overline{\psi'} \gamma_\mu \psi' A'^\mu \\
        &= e \psibar \pexp{ie\theta} \gamma_\mu \pexp{-ie\theta} \psi \big( A^\mu - [\partial^\mu \theta] \big) \\
        &= e \psibar \id \gamma_\mu \psi \big( A^\mu - [\partial^\mu \theta] \big) \\
        &= e \psibar \gamma_\mu \psi A^\mu - e \psibar \gamma_\mu \psi [\partial^\mu \theta] \\
        &= \L_I + J_\mu [\partial^\mu \theta] \: ,
\end{split}
\end{align}
hvor $\commutator{\gamma^\mu}{\, \U} = 0$, siden $\U$ er et tal (egenskab af $\U(1)$-gruppen).

For nu at besvare, om denne transformationen af vekselvirkningen tillader, at teorien er lokal gaugeinvariant, da afhænger forklaringen af, hvad der helt præcis menes med spørgsmålet. Betragter vi først den transformerede vekselvirkning i \cref{eq:Opg4_A3_InteractionLagrangianTransformed}, så ses det tydeligt, at $\L_I \ne \L_I'$, men ved sammensætningen med den frie Lagrangedensitet, som heller ikke i sig selv er lokal gaugeinvariant, da bliver summen af disse lokal gaugeinvariant ($\L = \L'$), hvilket er vist i \textbf{1)} ($\L_I$ var defineret således, at $\L$ blev lokal gaugeinvariant).

Skal det i stedet besvares, om teorien forbliver gaugeinvariant i den forstand, at $S_I = S_I'$, hvor $S_I$ er virkningen, så er argumentationen lidt anderledes. Vi starter stadig fra \cref{eq:Opg4_A3_InteractionLagrangianTransformed}, og husker på, at vi kan skrive $\partial^\mu (A_\mu B_\mu) = (\partial^\mu A_\mu) B_\mu + A_\mu (\partial^\mu B_\mu)$, så
\begin{align}
\begin{split}
    \L_I' &= \L_I + J_\mu \big( \partial^\mu \theta \big) \\
        &= \L_I + \big( \partial^\mu J_\mu \theta \big) + \big( \partial^\mu J_\mu \big) \theta \\
        &= \L_I + \big( \partial^\mu J_\mu \theta \big) \: ,
\end{split}
\end{align}
idet $\partial^\mu J_\mu = 0$ fra \cref{eq:Opg4_A3_CurrentIsConserved}. Når vi nu skal finde virkningen, da integreres denne vekselvirkning over hele rummet, hvilket for første led blot giver vekselvirkningen for den originale Lagrangedensitet for vekselvirkningen, mens det andet led er hel rummelig integration over $\Div{J_\mu \theta(x)}$. Dette kan omskrives ved brug af divergenssætningen\footnote{
    Divergenssætningen er giver som \cite[lign. 1.56]{griffiths_introToEldyn_2017}
    \begin{align}
        \int_V \Div{\vv{v}} \dd \tau &= \oint_S \vv{v} \cdot \dd \vv{a} \: .
    \end{align}
} (eng: Divergence theorem), så selvom $\theta(x)$ kan vælges som en arbitrær fase selv ved uendelig, så har strømmen en grænsebetingelse på $0$, hvorved dette andet led forsvinder således, at virkningen $S_I' = S_I$ og teorien derved er lokal gaugeinvariant.

Uanset metoden, så må svaret være, at \textbf{ja} teorien er lokal gaugeinvariant, enten med begrundelsen $\L = \L'$ eller at $S_I = S_I'$.


%%%%%%%%%%%%%%%%%%%%%%%%%

\paragraph[4) Beregn $-e\psibar\gamma_\mu\psi$ og find $j_\mu^{(n)}$]{\textbf{4)}}

De fire strømme $j_\mu^{(n)}$ findes ved at indsætte feltudvidelsen fra \cref{eq:Opg4_Q4_FieldExpansionNormalModes} i udtrykket for strømmen
\begin{align}
\begin{split}
    - e \psibar \gamma_\mu \psi
        &= - e \bigg\{ \sum_\lambda \int \frac{\dd^3\vv{p}}{(2\pi)^3} \invsqrt{2\omega_{\vv{p}}} \Big[ b_{\vv{p},\lambda} u(\vv{p},\lambda) \pexp{-ipx} \\
            &\qquad\qquad\qquad\qquad\qquad\qquad + d_{\vv{p},\lambda}\dagger v(\vv{p},\lambda) \pexp{ipx} \Big] \bigg\}\dagger \gamma^0 \gamma_\mu \\
            &\qquad \times \bigg\{ \sum_{\lambda'} \int \frac{\dd^3\vv{p}'}{(2\pi)^3} \invsqrt{2\omega_{\vv{p}'}} \Big[ b_{\vv{p}',\lambda'} u(\vv{p}',\lambda') \pexp{-ip'x} \\
            &\qquad\qquad\qquad\qquad\qquad\qquad\quad + d_{\vv{p}',\lambda'}\dagger v(\vv{p}',\lambda') \pexp{ip'x} \Big] \bigg\} \\
        &= - e \sum_{\lambda,\lambda'} \int \frac{\dd^3\vv{p}\, \dd^3\vv{p}'}{(2\pi)^6} \invsqrt{2\omega_{\vv{p}}\omega_{\vv{p}'}} \\
            &\qquad \times \Big[ \pexp{ipx} u\dagger(\vv{p},\lambda) b_{\vv{p},\lambda}\dagger
            + \pexp{-ipx} v\dagger(\vv{p},\lambda) d_{\vv{p},\lambda} \Big] \gamma^0 \gamma_\mu \\
            &\qquad \times \Big[ b_{\vv{p}',\lambda'} u(\vv{p}',\lambda') \pexp{-ip'x}
            + d_{\vv{p}',\lambda'}\dagger v(\vv{p}',\lambda') \pexp{ip'x} \Big] \\
        &= - e \sum_{\lambda,\lambda'} \int \frac{\dd^3\vv{p}\, \dd^3\vv{p}'}{(2\pi)^6} \invsqrt{2\omega_{\vv{p}}\omega_{\vv{p}'}} \\
            &\qquad \times \Big[ \pexp{ipx} u\dagger(\vv{p},\lambda) b_{\vv{p},\lambda}\dagger \gamma^0 \gamma_\mu b_{\vv{p}',\lambda'} u(\vv{p}',\lambda') \pexp{-ip'x} \\
            &\qquad\qquad + \pexp{-ipx} v\dagger(\vv{p},\lambda) d_{\vv{p},\lambda} \gamma^0 \gamma_\mu b_{\vv{p}',\lambda'} u(\vv{p}',\lambda') \pexp{-ip'x} \\
            &\qquad\qquad + \pexp{ipx} u\dagger(\vv{p},\lambda) b_{\vv{p},\lambda}\dagger \gamma^0 \gamma_\mu d_{\vv{p}',\lambda'}\dagger v(\vv{p}',\lambda') \pexp{ip'x} \\
            &\qquad\qquad + \pexp{-ipx} v\dagger(\vv{p},\lambda) d_{\vv{p},\lambda} \gamma^0 \gamma_\mu d_{\vv{p}',\lambda'}\dagger v(\vv{p}',\lambda') \pexp{ip'x} \Big] \\
        &= - e \sum_{\lambda,\lambda'} \int \frac{\dd^3\vv{p}\, \dd^3\vv{p}'}{(2\pi)^6} \invsqrt{2\omega_{\vv{p}}\omega_{\vv{p}'}} \\
            &\qquad \times \Big[ \pexp{i[p-p']x} b_{\vv{p},\lambda}\dagger b_{\vv{p}',\lambda'} \overline{u}(\vv{p},\lambda) \gamma_\mu u(\vv{p}',\lambda') \\
            &\qquad\qquad + \pexp{-i[p+p']x} d_{\vv{p},\lambda} b_{\vv{p}',\lambda'} \overline{v}(\vv{p},\lambda) \gamma_\mu u(\vv{p}',\lambda') \\
            &\qquad\qquad + \pexp{i[p+p']x} b_{\vv{p},\lambda}\dagger d_{\vv{p}',\lambda'}\dagger \overline{u}(\vv{p},\lambda) \gamma_\mu v(\vv{p}',\lambda') \\
            &\qquad\qquad + \pexp{-i[p-p']x} d_{\vv{p},\lambda} d_{\vv{p}',\lambda'}\dagger \overline{v}(\vv{p},\lambda) \gamma_\mu v(\vv{p}',\lambda') \Big] \: .
\end{split}
\end{align}

Dermed bliver de fire strømme
\begin{subequations} \label{eq:Opg4_A4_FourPartsOfTheCurrectExplicit}
\begin{align}
    j_\mu^{(1)} &= - e \sum_{\lambda,\lambda'} \int \frac{\dd^3\vv{p}\, \dd^3\vv{p}'}{(2\pi)^6} \frac{1}{2\sqrt{\omega_{\vv{p}}\omega_{\vv{p}'}}} \pexp{i[p-p']x} b_{\vv{p},\lambda}\dagger b_{\vv{p}',\lambda'} \overline{u}(\vv{p},\lambda) \gamma_\mu u(\vv{p}',\lambda')
    \label{eq:Opg4_A4_FourPartsOfTheCurrectExplicitJ1} \\
    %
    j_\mu^{(2)} &= - e \sum_{\lambda,\lambda'} \int \frac{\dd^3\vv{p}\, \dd^3\vv{p}'}{(2\pi)^6} \frac{1}{2\sqrt{\omega_{\vv{p}}\omega_{\vv{p}'}}} \pexp{-i[p-p']x} d_{\vv{p},\lambda} d_{\vv{p}',\lambda'}\dagger \overline{v}(\vv{p},\lambda) \gamma_\mu v(\vv{p}',\lambda')
    \label{eq:Opg4_A4_FourPartsOfTheCurrectExplicitJ2} \\
    %
    j_\mu^{(3)} &= - e \sum_{\lambda,\lambda'} \int \frac{\dd^3\vv{p}\, \dd^3\vv{p}'}{(2\pi)^6} \frac{1}{2\sqrt{\omega_{\vv{p}}\omega_{\vv{p}'}}} \pexp{-i[p+p']x} d_{\vv{p},\lambda} b_{\vv{p}',\lambda'} \overline{v}(\vv{p},\lambda) \gamma_\mu u(\vv{p}',\lambda')
    \label{eq:Opg4_A4_FourPartsOfTheCurrectExplicitJ3} \\
    %
    j_\mu^{(4)} &= - e \sum_{\lambda,\lambda'} \int \frac{\dd^3\vv{p}\, \dd^3\vv{p}'}{(2\pi)^6} \frac{1}{2\sqrt{\omega_{\vv{p}}\omega_{\vv{p}'}}} \pexp{i[p+p']x} b_{\vv{p},\lambda}\dagger d_{\vv{p}',\lambda'}\dagger \overline{u}(\vv{p},\lambda) \gamma_\mu v(\vv{p}',\lambda') \: ,
    \label{eq:Opg4_A4_FourPartsOfTheCurrectExplicitJ4}
\end{align}
\end{subequations}
og $-e\psibar\gamma_\mu\psi$ er summen af disse fire strømme, $-e\psibar\gamma_\mu\psi = \sum_n j_\mu^{(n)}, \; \forall n \in [1,2,3,4]$.
I definitionen herover kommer strømmenes givne $n$ af, at de skal passe på \cref{eq:Opg4_Q5_TransitionCurrents}. Dermed er \cref{eq:Opg4_Q4_FourPartsOfTheCurrent} vist.


%%%%%%%%%%%%%%%%%%%%%%%%%

\paragraph[5) Udled matrixelementer fra $j_\mu^{(n)}$]{\textbf{5)}}

I de følgende udregninger vil de fermioniske antikommutatorrelationer (se \cite[lign. 36--37]{problemSet3})
\begin{subequations}
\begin{align}
    (2\pi)^3 \delta^3(\vv{p} - \vv{p}') \delta_{\lambda,\lambda'}
        &= \anticommutator{b_{\vv{p},\lambda}}{b_{\vv{p}',\lambda'}\dagger}
        = b_{\vv{p},\lambda} b_{\vv{p}',\lambda'}\dagger + b_{\vv{p}',\lambda'}\dagger b_{\vv{p},\lambda} \: , \quad \text{og} \\
    (2\pi)^3 \delta^3(\vv{p} - \vv{p}') \delta_{\lambda,\lambda'}
        &= \anticommutator{d_{\vv{p},\lambda}}{d_{\vv{p}',\lambda'}\dagger}
        = d_{\vv{p},\lambda} d_{\vv{p}',\lambda'}\dagger + d_{\vv{p}',\lambda'}\dagger d_{\vv{p},\lambda} \: ,
\end{align}
\end{subequations}
hvor de resterende kombinationer af operatorerne giver $0$, blive benyttet, samt at
\begin{subequations}
\begin{align}
    b_{\vv{p},\lambda} \ket{0} &= 0
        \quad \xleftrightarrow{\textrm{ DC }} \quad
    \bra{0} b_{\vv{p},\lambda}\dagger = 0 \: , \quad \text{og} \\
    d_{\vv{p},\lambda} \ket{0} &= 0
        \quad \xleftrightarrow{\textrm{ DC }} \quad
    \bra{0} d_{\vv{p},\lambda}\dagger = 0 \: .
\end{align}
\end{subequations}
For matrixelementer betegner vi
\begin{subequations}
\begin{align}
    \ket{\mathrm{e}^-,\, \vv{p},\, \lambda} &= \sqrt{2 \omega_{\vv{p}}} \, b_{\vv{p},\lambda}\dagger \ket{0}
        \quad \xleftrightarrow{\textrm{ DC }} \quad
    \bra{\mathrm{e}^-,\, \vv{p},\, \lambda} = \bra{0} b_{\vv{p},\lambda} \sqrt{2 \omega_{\vv{p}}}
        \: , \quad \text{og} \\
    \ket{\mathrm{e}^+,\, \vv{p},\, \lambda} &= \sqrt{2 \omega_{\vv{p}}} \, d_{\vv{p},\lambda}\dagger \ket{0}
        \quad \xleftrightarrow{\textrm{ DC }} \quad
    \bra{\mathrm{e}^+,\, \vv{p},\, \lambda} = \bra{0} d_{\vv{p},\lambda} \sqrt{2 \omega_{\vv{p}}}
        \: ,
\end{align}
\end{subequations}
da $b_{\vv{p},\lambda}\dagger$ kreerer en partikel, mens $d_{\vv{p},\lambda}\dagger$ kreerer en antipartikel, og $\sqrt{2\omega_{\vv{p}}}$ er en normaliseringsfaktor.

I det følgende udregnes de fire matrixelemter i \cref{eq:Opg4_Q5_TransitionCurrents} ved først at  udregne matrixelementerne med kreations- og annihilationsoperatorerne, som fremkommer som dele af de fire matrixelementer i \cref{eq:Opg4_Q5_TransitionCurrents}, hvorefter disse ''store'' matrixelementer udregnes. Til dette benyttes de ovenstående relationer.
\\

For det første matrixelement, \cref{eq:Opg4_Q5_TransitionCurrent1}, har vi, at
\begin{align}
\begin{split}
    &\mel**{0}{b_{\vv{p}',\lambda'} b_{\vv{q},s}\dagger b_{\vv{q}',s'} b_{\vv{p},\lambda}\dagger}{0} \\
        &\quad = \mel**{0}{\left[ (2\pi)^3 \delta^3(\vv{q} - \vv{p}') \delta_{s,\lambda'} - b_{\vv{q},s}\dagger b_{\vv{p}',\lambda'} \right] \left[ (2\pi)^3 \delta^3(\vv{p} - \vv{q}') \delta_{\lambda,s'} - b_{\vv{p},\lambda}\dagger b_{\vv{q}',s'} \right]}{0} \\
        &\quad = (2\pi)^6 \delta^3(\vv{q} - \vv{p}') \delta^3(\vv{p} - \vv{q}') \delta_{s,\lambda'} \delta_{\lambda,s'} \innerproduct{0}{0}
            - (2\pi)^3 \delta^3(\vv{q} - \vv{p}') \delta_{s,\lambda'} \cancelto{0}{\mel**{0}{b_{\vv{p},\lambda}\dagger b_{\vv{q}',s'}}{0}} \\
            &\qquad\: - (2\pi)^3 \delta^3(\vv{p} - \vv{q}') \delta_{\lambda,s'} \cancelto{0}{\mel**{0}{b_{\vv{q},s}\dagger b_{\vv{p}',\lambda'}}{0}}
            + \cancelto{0}{\mel**{0}{b_{\vv{q},s}\dagger b_{\vv{p}',\lambda'} b_{\vv{p},\lambda}\dagger b_{\vv{q}',s'}}{0}} \\
        &\quad = (2\pi)^6 \delta^3(\vv{q} - \vv{p}') \delta^3(\vv{p} - \vv{q}') \delta_{s,\lambda'} \delta_{\lambda,s'} \: ,
\end{split}
\end{align}
hvorfor
\begin{align}
\begin{split}
    &\mel**{\mathrm{e}^-,\, \vv{p}',\, \lambda'}{j_\mu^{(1)}}{\mathrm{e}^-,\, \vv{p},\, \lambda} \\
        &\quad = \bigg\langle \mathrm{e}^-,\, \vv{p}',\, \lambda' \bigg\rvert - e \sum_{s,s'} \int \frac{\dd^3\vv{q}\, \dd^3\vv{q}'}{(2\pi)^6} \frac{1}{2\sqrt{\omega_{\vv{q}}\omega_{\vv{q}'}}} \pexp{i[q-q']x} \\
            &\qquad\qquad\qquad\qquad\qquad\qquad \times b_{\vv{q},s}\dagger b_{\vv{q}',s'} \overline{u}(\vv{q},s) \gamma_\mu u(\vv{q}',s') \bigg\lvert \mathrm{e}^-,\, \vv{p},\, \lambda \bigg\rangle \\
        &\quad = \bigg\langle 0 \bigg\rvert b_{\vv{p}',\lambda'} \sqrt{2 \omega_{\vv{p}'}}
            - e \sum_{s,s'} \int \frac{\dd^3\vv{q}\, \dd^3\vv{q}'}{(2\pi)^6} \frac{1}{2\sqrt{\omega_{\vv{q}}\omega_{\vv{q}'}}} \pexp{i[q-q']x} \\
            &\qquad\qquad\qquad\qquad\qquad\qquad \times b_{\vv{q},s}\dagger b_{\vv{q}',s'} \overline{u}(\vv{q},s) \gamma_\mu u(\vv{q}',s')
            \sqrt{2 \omega_{\vv{p}}} \, b_{\vv{p},\lambda}\dagger \bigg\lvert 0 \bigg\rangle \\
        &\quad = - 2 e \sqrt{\omega_{\vv{p}'}\omega_{\vv{p}}} \sum_{s,s'} \int \frac{\dd^3\vv{q}\, \dd^3\vv{q}'}{(2\pi)^6} \frac{1}{2\sqrt{\omega_{\vv{q}}\omega_{\vv{q}'}}} \pexp{i[q-q']x} \\
            &\qquad\qquad\qquad\qquad\qquad\qquad \times \mel**{0}{b_{\vv{p}',\lambda'} b_{\vv{q},s}\dagger b_{\vv{q}',s'} b_{\vv{p},\lambda}\dagger}{0} \overline{u}(\vv{q},s) \gamma_\mu u(\vv{q}',s') \\
        &\quad = - 2 e \sqrt{\omega_{\vv{p}'}\omega_{\vv{p}}} \sum_{s,s'} \int \frac{\dd^3\vv{q}\, \dd^3\vv{q}'}{(2\pi)^6} \frac{1}{2\sqrt{\omega_{\vv{q}}\omega_{\vv{q}'}}} \pexp{i[q-q']x} \\
            &\qquad\qquad\qquad \times (2\pi)^6 \delta^3(\vv{q} - \vv{p}') \delta^3(\vv{p} - \vv{q}') \delta_{s,\lambda'} \delta_{\lambda,s'} \overline{u}(\vv{q},s) \gamma_\mu u(\vv{q}',s') \\
        &\quad = - e \sqrt{\omega_{\vv{p}'}\omega_{\vv{p}}} \sum_{s,s'} \int \frac{\dd^3\vv{q}\, \dd^3\vv{q}'}{\sqrt{\omega_{\vv{q}}\omega_{\vv{q}'}}} \pexp{i[q-q']x} \\
            &\qquad\qquad\qquad \times \delta^3(\vv{q} - \vv{p}') \delta^3(\vv{p} - \vv{q}') \delta_{s,\lambda'} \delta_{\lambda,s'} \overline{u}(\vv{q},s) \gamma_\mu u(\vv{q}',s') \\
        &\quad = - e \sum_{s,s'} \pexp{i[p'-p]x} \delta_{s,\lambda'} \delta_{\lambda,s'} \overline{u}(\vv{p}',s) \gamma_\mu u(\vv{p},s') \\
        &\quad = - e \overline{u}(\vv{p}',\lambda') \gamma_\mu u(\vv{p},\lambda) \pexp{i[p'-p]x} \: ,
\end{split}
\end{align}
da $q^\mu = (\sqrt{\vv{q}^2 + m^2},\, \vv{q})$, hvorfor integrationen af deltafunktionerne med $\vv{q}$ også ændrer på energien, så f.eks. $\delta^3(\vv{q}-\vv{p}') \Rightarrow q \rightarrow p'$.
Derved er \cref{eq:Opg4_Q5_TransitionCurrent1} vist.
\\

Det andet matrixelement, \cref{eq:Opg4_Q5_TransitionCurrent2}, giver
\begin{align}
\begin{split}
    &\mel**{0}{d_{\vv{p}',\lambda'} d_{\vv{q},s} d_{\vv{q}',s'}\dagger d_{\vv{p},\lambda}\dagger}{0} \\
        &\quad = \mel**{0}{d_{\vv{p}',\lambda'} \left[(2\pi)^3 \delta^3(\vv{q}-\vv{q}') \delta_{s,s'} - d_{\vv{q}',s'}\dagger d_{\vv{q},s} \right] d_{\vv{p},\lambda}\dagger}{0} \\
        &\quad = (2\pi)^3 \delta^3(\vv{q}-\vv{q}') \delta_{s,s'} \mel**{0}{d_{\vv{p}',\lambda'} d_{\vv{p},\lambda}\dagger}{0}
            - \mel**{0}{d_{\vv{p}',\lambda'} d_{\vv{q}',s'}\dagger d_{\vv{q},s} d_{\vv{p},\lambda}\dagger}{0} \\
        &\quad = (2\pi)^3 \delta^3(\vv{q}-\vv{q}') \delta_{s,s'} \mel**{0}{\left[ (2\pi)^3 \delta^3(\vv{p}-\vv{p}') \delta_{\lambda,\lambda'} - d_{\vv{p},\lambda}\dagger d_{\vv{p}',\lambda'} \right]}{0} \\
            &\qquad\: - \mel**{0}{\left[ (2\pi)^3 \delta^3(\vv{p}'-\vv{q}') \delta_{\lambda',s'} - d_{\vv{q}',s'}\dagger d_{\vv{p}',\lambda'} \right] \left[ (2\pi)^3 \delta^3(\vv{p}-\vv{q}) \delta_{\lambda,s} - d_{\vv{p},\lambda}\dagger d_{\vv{q},s} \right]}{0} \\
        &\quad = (2\pi)^6 \delta^3(\vv{q}-\vv{q}') \delta^3(\vv{p}-\vv{p}') \delta_{s,s'} \delta_{\lambda,\lambda'} \innerproduct{0}{0}
            - (2\pi)^6 \delta^3(\vv{p}'-\vv{q}') \delta^3(\vv{p}-\vv{q}) \delta_{\lambda',s'} \delta_{\lambda,s} \innerproduct{0}{0} \\
        &\quad = (2\pi)^6 \left[ \delta^3(\vv{q}-\vv{q}') \delta^3(\vv{p}-\vv{p}') \delta_{s,s'} \delta_{\lambda,\lambda'}
            - \delta^3(\vv{p}'-\vv{q}') \delta^3(\vv{p}-\vv{q}) \delta_{\lambda',s'} \delta_{\lambda,s} \right] \: ,
\end{split}
\end{align}
hvorfor
\begin{align} \label{eq:Opg4_A5_SecondMatrixElementCalculationPart1}
\begin{split}
    &\mel**{\mathrm{e}^+,\, \vv{p}',\, \lambda'}{j_\mu^{(2)}}{\mathrm{e}^+,\, \vv{p},\, \lambda} \\
        &\quad = \bigg\langle \mathrm{e}^+,\, \vv{p}',\, \lambda' \bigg\rvert - e \sum_{s,s'} \int \frac{\dd^3\vv{q}\, \dd^3\vv{q}'}{(2\pi)^6} \frac{1}{2\sqrt{\omega_{\vv{q}}\omega_{\vv{q}'}}} \pexp{-i[q-q']x} \\
            &\qquad\qquad\qquad\qquad\qquad\qquad \times d_{\vv{q},s} d_{\vv{q}',s'}\dagger \overline{v}(\vv{q},s) \gamma_\mu v(\vv{q}',s') \bigg\rvert \mathrm{e}^+,\, \vv{p},\, \lambda \bigg\rangle \\
        &\quad = - e \sum_{s,s'} \int \frac{\dd^3\vv{q}\, \dd^3\vv{q}'}{(2\pi)^6} \sqf{\omega_{\vv{p}'}\omega_{\vv{p}}}{\omega_{\vv{q}}\omega_{\vv{q}'}} \pexp{-i[q-q']x} \\
            &\qquad\qquad\qquad\qquad\qquad\qquad \times \mel**{0}{d_{\vv{p}',\lambda'} d_{\vv{q},s} d_{\vv{q}',s'}\dagger d_{\vv{p},\lambda}\dagger}{0} \overline{v}(\vv{q},s) \gamma_\mu v(\vv{q}',s') \\
        &\quad = - e \sum_{s,s'} \int \dd^3\vv{q}\, \dd^3\vv{q}' \sqf{\omega_{\vv{p}'}\omega_{\vv{p}}}{\omega_{\vv{q}}\omega_{\vv{q}'}} \pexp{-i[q-q']x} \overline{v}(\vv{q},s) \gamma_\mu v(\vv{q}',s') \\
            &\qquad\qquad \times \left[ \delta^3(\vv{q}-\vv{q}') \delta^3(\vv{p}-\vv{p}') \delta_{s,s'} \delta_{\lambda,\lambda'}
            - \delta^3(\vv{p}'-\vv{q}') \delta^3(\vv{p}-\vv{q}) \delta_{\lambda',s'} \delta_{\lambda,s} \right] \\
        &\quad = e \overline{v}(\vv{p},\lambda) \gamma_\mu v(\vv{p}',\lambda') \pexp{i[p'-p]x} \\
            &\qquad\qquad\qquad\qquad\quad - e \delta^3(\vv{p}-\vv{p}') \delta_{\lambda,\lambda'} \sum_s \int \dd^3\vv{q}\, \frac{\omega_{\vv{p}}}{\omega_{\vv{q}}}\, \overline{v}(\vv{q},s) \gamma_\mu v(\vv{q},s) \: .
\end{split}
\end{align}
For andet led i \cref{eq:Opg4_A5_SecondMatrixElementCalculationPart1} ses det i Dirac deltafunktionen, at impulsen for elektronen før og efter vekselvirkningen skal være ens, og ligeså for myonen, men siden dette ikke kan lade sig gøre ved udveksling af en foton, er der altså ikke tale om en spredning men blot en loopkorrektion, hvorfor dette led ikke medtages men blot negligeres. Derved får vi\\
\begin{align}
\begin{split}
    &\mel**{\mathrm{e}^+,\, \vv{p}',\, \lambda'}{j_\mu^{(2)}}{\mathrm{e}^+,\, \vv{p},\, \lambda} \\
        &\quad = e \overline{v}(\vv{p},\lambda) \gamma_\mu v(\vv{p}',\lambda') \pexp{i[p'-p]x} \\
            &\qquad\qquad\qquad\qquad\quad - e \delta^3(\vv{p}-\vv{p}') \delta_{\lambda,\lambda'} \sum_s \int \dd^3\vv{q}\, \frac{\omega_{\vv{p}}}{\omega_{\vv{q}}}\, \overline{v}(\vv{q},s) \gamma_\mu v(\vv{q},s) \\
        &\quad = e \overline{v}(\vv{p},\lambda) \gamma_\mu v(\vv{p}',\lambda') \pexp{i[p'-p]x} \\
            &\qquad\qquad\qquad\qquad\quad - e \delta^3(\vv{p}-\vv{p}') \delta_{\lambda,\lambda'} \int \dd^3\vv{q}\, \left( \frac{\omega_{\vv{p}}}{\omega_{\vv{q}}}\, \gamma_\mu  \sum_s \overline{v}(\vv{q},s) v(\vv{q},s) \right) \\
        &\quad = e \overline{v}(\vv{p},\lambda) \gamma_\mu v(\vv{p}',\lambda') \pexp{i[p'-p]x} \: .
\end{split}
\end{align}
Derved er \cref{eq:Opg4_Q5_TransitionCurrent2} vist.
\\

I det tredje matrixelement, \cref{eq:Opg4_Q5_TransitionCurrent3}, optræder
\begin{align}
\begin{split}
    &\mel**{0}{d_{\vv{q},s} b_{\vv{q}',s'} b_{\vv{p},\lambda}\dagger d_{\vv{p}',\lambda'}\dagger}{0} \\
        &\quad = \mel**{0}{\left[- b_{\vv{q}',s'} d_{\vv{q},s} \right] b_{\vv{p},\lambda}\dagger d_{\vv{p}',\lambda'}\dagger}{0} \\
        &\quad = - \mel**{0}{b_{\vv{q}',s'} \left[ - b_{\vv{p},\lambda}\dagger d_{\vv{q},s} \right] d_{\vv{p}',\lambda'}\dagger}{0} \\
        &\quad = \mel**{0}{b_{\vv{q}',s'} b_{\vv{p},\lambda}\dagger d_{\vv{q},s} d_{\vv{p}',\lambda'}\dagger}{0} \\
        &\quad = \mel**{0}{\left[ (2\pi)^3 \delta^3(\vv{p}-\vv{q}') \delta_{\lambda,s'} - b_{\vv{p},\lambda}\dagger b_{\vv{q}',s'} \right] \left[ (2\pi)^3 \delta^3(\vv{q}-\vv{p}') \delta_{s,\lambda'} - d_{\vv{p}',\lambda'}\dagger d_{\vv{q},s} \right]}{0} \\
        &\quad = (2\pi)^6 \delta^3(\vv{p}-\vv{q}') \delta^3(\vv{q}-\vv{p}') \delta_{\lambda,s'} \delta_{s,\lambda'} \: ,
\end{split}
\end{align}
så vi får
\begin{align}
\begin{split}
    &\mel**{0}{j_\mu^{(3)}}{\mathrm{e}^-,\, \vv{p},\, \lambda;\, \mathrm{e}^+,\, \vv{p}',\, \lambda'} \\
        &\quad = \bigg\langle 0 \bigg\rvert - e \sum_{s,s'} \int \frac{\dd^3\vv{q}\, \dd^3\vv{q}'}{(2\pi)^6} \frac{1}{2\sqrt{\omega_{\vv{q}}\omega_{\vv{q}'}}} \pexp{-i[q+q']x} \\
            &\qquad\qquad\qquad\qquad\qquad \times d_{\vv{q},s} b_{\vv{q}',s'} \overline{v}(\vv{p},\lambda) \gamma_\mu u(\vv{p}',\lambda') \bigg\lvert \mathrm{e}^-,\, \vv{p},\, \lambda;\, \mathrm{e}^+,\, \vv{p}',\, \lambda' \bigg\rangle \\
        &\quad = - e \sum_{s,s'} \int \frac{\dd^3\vv{q}\, \dd^3\vv{q}'}{(2\pi)^6} \sqf{\omega_{\vv{p}}\omega_{\vv{p}'}}{\omega_{\vv{q}}\omega_{\vv{q}'}} \pexp{-i[q+q']x} \\
            &\qquad\qquad\qquad\qquad\qquad \times \mel**{0}{d_{\vv{q},s} b_{\vv{q}',s'} b_{\vv{p},\lambda}\dagger d_{\vv{p}',\lambda'}\dagger}{0} \overline{v}(\vv{q},s) \gamma_\mu u(\vv{q}',s') \\
        &\quad = - e \sum_{s,s'} \int \frac{\dd^3\vv{q}\, \dd^3\vv{q}'}{(2\pi)^6} \sqf{\omega_{\vv{p}}\omega_{\vv{p}'}}{\omega_{\vv{q}}\omega_{\vv{q}'}} \pexp{-i[q+q']x} \overline{v}(\vv{q},s) \gamma_\mu u(\vv{q}',s') \\
            &\qquad\qquad\qquad\qquad\qquad \times (2\pi)^6 \delta^3(\vv{p}-\vv{q}') \delta^3(\vv{q}-\vv{p}') \delta_{\lambda,s'} \delta_{s,\lambda'} \\
        &\quad = - e \overline{v}(\vv{p}',\lambda') \gamma_\mu u(\vv{p},\lambda) \pexp{-i[p+p']x} \: .
\end{split}
\end{align}
Derved er \cref{eq:Opg4_Q5_TransitionCurrent3} vist.
\\

For det fjerde og sidste matrixelement, \cref{eq:Opg4_Q5_TransitionCurrent1}, har vi, at
\begin{align}
\begin{split}
    &\mel**{0}{d_{\vv{p},\lambda} b_{\vv{p}',\lambda'} b_{\vv{q},s}\dagger d_{\vv{q}',s'}\dagger}{0} \\
        &\quad = \mel**{0}{\left[ - b_{\vv{p}',\lambda'} d_{\vv{p},\lambda} \right] b_{\vv{q},s}\dagger d_{\vv{q}',s'}\dagger}{0} \\
        &\quad = - \mel**{0}{b_{\vv{p}',\lambda'} \left[ - b_{\vv{q},s}\dagger d_{\vv{p},\lambda} \right] d_{\vv{q}',s'}\dagger}{0} \\
        &\quad = \mel**{0}{b_{\vv{p}',\lambda'} b_{\vv{q},s}\dagger d_{\vv{p},\lambda} d_{\vv{q}',s'}\dagger}{0} \\
        &\quad = \mel**{0}{\left[ (2\pi)^3 \delta^3(\vv{q}-\vv{p}') \delta_{s,\lambda'} - b_{\vv{q},s}\dagger b_{\vv{p}',\lambda'} \right] \left[ (2\pi)^3 \delta^3(\vv{p}-\vv{q}') \delta_{\lambda,s'} - d_{\vv{q}',s'}\dagger d_{\vv{p},\lambda} \right]}{0} \\
        &\quad = (2\pi)^6 \delta^3(\vv{q}-\vv{p}') \delta^3(\vv{p}-\vv{q}') \delta_{s,\lambda'} \delta_{\lambda,s'} \: ,
\end{split}
\end{align}
hvorfor
\begin{align}
\begin{split}
    &\mel**{\mathrm{e}^-,\, \vv{p}',\, \lambda';\, \mathrm{e}^+,\, \vv{p},\, \lambda}{j_\mu^{(4)}}{0} \\
        &\quad = \bigg\langle \mathrm{e}^-,\, \vv{p}',\, \lambda';\, \mathrm{e}^+,\, \vv{p},\, \lambda \bigg\rvert - e \sum_{s,s'} \int \frac{\dd^3\vv{q}\, \dd^3\vv{q}'}{(2\pi)^6} \frac{1}{2\sqrt{\omega_{\vv{q}}\omega_{\vv{q}'}}} \pexp{i[q+q']x} \\
            &\qquad\qquad\qquad\qquad\qquad\qquad\qquad \times b_{\vv{q},\lambda}\dagger d_{\vv{q}',\lambda'}\dagger \overline{u}(\vv{q},s) \gamma_\mu v(\vv{q}',s') \bigg\lvert 0 \bigg\rangle \\
        &\quad = - e \sum_{s,s'} \int \frac{\dd^3\vv{q}\, \dd^3\vv{q}'}{(2\pi)^6} \sqf{\omega_{\vv{p}'}\omega_{\vv{p}}}{\omega_{\vv{q}}\omega_{\vv{q}'}} \pexp{i[q+q']x} \\
            &\qquad\qquad\qquad\qquad\qquad \times \mel**{0}{d_{\vv{p},\lambda} b_{\vv{p}',\lambda'} b_{\vv{q},s}\dagger d_{\vv{q}',s'}\dagger}{0} \overline{u}(\vv{q},s) \gamma_\mu v(\vv{q}',s') \\
        &\quad = - e \sum_{s,s'} \int \frac{\dd^3\vv{q}\, \dd^3\vv{q}'}{(2\pi)^6} \sqf{\omega_{\vv{p}'}\omega_{\vv{p}}}{\omega_{\vv{q}}\omega_{\vv{q}'}} \pexp{i[q+q']x} \overline{u}(\vv{q},s) \gamma_\mu v(\vv{q}',s') \\
            &\qquad\qquad\qquad\qquad\qquad \times (2\pi)^6 \delta^3(\vv{q}-\vv{p}') \delta^3(\vv{p}-\vv{q}') \delta_{s,\lambda'} \delta_{\lambda,s'} \\
        &\quad = - e \overline{u}(\vv{p}',\lambda') \gamma_\mu v(\vv{p},\lambda) \pexp{i[p+p']x} \: .
\end{split}
\end{align}
Derved er \cref{eq:Opg4_Q5_TransitionCurrent4} vist.
\\

Overgangsstrømmene i \cref{eq:Opg4_Q5_TransitionCurrents} er matrixelementerne mellem start- og sluttilstandene for impulsudvekslingen mellem elektroner og positroner i forskellige kombinationer, dog ekskluderende fotonen, som også er en del af vekselvirkningen, i de to tilstande, som i stedet beskrives af strømoperatoren, hvorfor dette kun giver dele af Feynmandiagrammer og ikke de fulde diagrammer. Vi kan derfor tegne disse som Feynmandiagrammer, hvilke kan ses nedenfor. Matrixelementerne er alle på formen $\mel{f}{j_\mu}{i}$ hvor $i$ og $f$ betegner hhv. start- og sluttilstanden, hvorved det første matrixelement er en indkommende elektron med impuls $\vv{p}$ og helicitet $\lambda$ som ændres til en udgående elektron med impuls $\vv{p}'$ og helicitet $\lambda'$. For andet matrixelement er positron med impuls $\vv{p}$ og helicitet $\lambda$ som spredes til en positron med impuls $\vv{p}'$ og helicitet $\lambda'$. Det tredje matrixelement er annihilationen af et elektron-positron-par til vakuum (eller rettere end foton), og det fjerde matrixelement er blot denne annihilation set med tiden gående baglæns, altså dannelsen af et elektron-positron-par fra en foton.

\begin{tikzpicture}[scale=.85, transform shape]
    \begin{feynman}
        \vertex (vertex);
        \vertex [below left = 3em and 1.5em of vertex] (inParticle) {\(e^-\)};
        \vertex [above left = 3em and 1.5em of vertex] (outParticle) {\(e^-\)};
        \vertex [right = 3em of vertex] (photon) {\(\gamma\)};
        \vertex [above left = 1em and 1.5em of inParticle] (timeBot);
        \vertex [above = 5.5em of timeBot] (timeTop);
        \vertex [below = 5.5em of vertex] {\(j_\mu^{(1)}\)};
        \diagram*{
            (timeBot) -- [draw = none, momentum = $t$] (timeTop),
            (inParticle) -- [fermion] (vertex) -- [fermion] (outParticle),
            (vertex) -- [photon] (photon),
        };
    \end{feynman}
\end{tikzpicture}
\hfill
\begin{tikzpicture}[scale=.85, transform shape]
    \begin{feynman}
        \vertex (vertex);
        \vertex [below right = 3em and 1.5em of vertex] (inParticle) {\(e^+\)};
        \vertex [above right = 3em and 1.5em of vertex] (outParticle) {\(e^+\)};
        \vertex [left = 3em of vertex] (photon) {\(\gamma\)};
        \vertex [below = 5.5em of vertex] {\(j_\mu^{(2)}\)};
        \diagram*{
            (inParticle) -- [anti fermion] (vertex) -- [anti fermion] (outParticle),
            (vertex) -- [photon] (photon),
        };
    \end{feynman}
\end{tikzpicture}
\hfill
\begin{tikzpicture}[scale=.85, transform shape]
    \begin{feynman}
        \vertex (vertex);
        \vertex [below left = of vertex] (inParticle1) {\(e^-\)};
        \vertex [below right = of vertex] (inParticle2) {\(e^+\)};
        \vertex [above = 3em of vertex] (photon) {\(\gamma\)};
        \vertex [below = 5.5em of vertex] {\(j_\mu^{(3)}\)};
        \diagram*{
            (inParticle1) -- [fermion] (vertex) -- [photon] (photon),
            (inParticle2) -- [anti fermion] (vertex),
        };
    \end{feynman}
\end{tikzpicture}
\hfill
\begin{tikzpicture}[scale=.85, transform shape]
    \begin{feynman}
        \vertex (vertex);
        \vertex [above left = of vertex] (outParticle1) {\(e^-\)};
        \vertex [above right = of vertex] (outParticle2) {\(e^+\)};
        \vertex [below = 3em of vertex] (photon) {\(\gamma\)};
        \vertex [below = 5.5em of vertex] {\(j_\mu^{(4)}\)};
        \diagram*{
            (photon) -- [photon] (vertex) -- [fermion] (outParticle1),
            (vertex) -- [anti fermion] (outParticle2),
        };
    \end{feynman}
\end{tikzpicture}


%%%%%%%%%%%%%%%%%%%%%%%%%

\paragraph[6) Bevarede overgangsstrømme $J_\mu^{fi}$]{\textbf{6)}}

I det følgende vil det blive vist, at overgangsstrømmene i \cref{eq:Opg4_Q5_TransitionCurrents} er bevarede.

Anvender vi $\partial^\mu$ på det første matrixelement, \cref{eq:Opg4_Q5_TransitionCurrent1}, fås
\begin{align} \label{eq:Opg4_A6_ConservedTransitionCurrent1}
\begin{split}
    \partial^\mu \mel**{\mathrm{e}^-,\, \vv{p}',\, \lambda'}{j_\mu^{(1)}}{\mathrm{e}^-,\, \vv{p},\, \lambda}
        &= \partial^\mu \big[ -e\bar{u}(\vv{p}',\lambda') \gamma_\mu u(\vv{p},\lambda) \pexp{i[p' - p]x} \big] \\
        &= -e\bar{u}(\vv{p}',\lambda') \gamma_\mu u(\vv{p},\lambda) \partial^\mu \big[ \pexp{i[p' - p]x} \big] \\
        &= -e\bar{u}(\vv{p}',\lambda') \gamma_\mu u(\vv{p},\lambda) i (p'^\mu - p^\mu) \pexp{i[p' - p]x} \\
        &= -ie\bar{u}(\vv{p}',\lambda') \gamma_\mu (p'^\mu - p^\mu) u(\vv{p},\lambda) \pexp{i[p' - p]x} \\
        &= -ie\bar{u}(\vv{p}',\lambda') (\slashed{p}' - \slashed{p}) u(\vv{p},\lambda) \pexp{i[p' - p]x} \\
        &= -ie\bar{u}(\vv{p}',\lambda') (m - m) u(\vv{p},\lambda) \pexp{i[p' - p]x} \\
        &= 0 \: ,
\end{split}
\end{align}
hvor relationen $(\slashed{p} - m)u(\vv{p}, \lambda) = 0$ fra \cite[opgave 2.4]{problemSet3} og $\overline{u}(\vv{p}, \lambda) (\slashed{p} - m) = 0$ er benyttet. Sidstnævnte findes nemt ved at tage den Hermitiskkonjugerede af førstnævnte relation, hvorved der fås
\begin{align} \label{eq:Opg4_A6_BaredVersionOfDiracLikeEquationFromProbSet3Ex2.4}
\begin{split}
    0 &= \big[ (\slashed{p} - m)u(\vv{p}, \lambda) \big]\dagger \\
        &= u\dagger(\vv{p}, \lambda) (\slashed{p} - m)\dagger \\
        &= u\dagger(\vv{p}, \lambda) \big[(\gamma^\mu)\dagger p_\mu - m\big] \\
        &= u\dagger(\vv{p}, \lambda) \id \big[(\gamma^\mu)\dagger p_\mu - m\big] \\
        &= u\dagger(\vv{p}, \lambda) \gamma^0 \gamma^0 \big[(\gamma^\mu)\dagger p_\mu - m\big] \\
        &= \overline{u}(\vv{p}, \lambda) \big[\gamma^\mu p_\mu \gamma^0 - m \gamma^0\big] \\
        &= \Big\{ \overline{u}(\vv{p}, \lambda) \big[\slashed{p} - m\big] \Big\} \gamma^0 \\
    \Rightarrow 0 &= \overline{u}(\vv{p}, \lambda) (\slashed{p} - m) \: ,
\end{split}
\end{align}
hvor det er benyttet, at $p^\mu$ er Hermitisk, at $\commutator{p_\mu}{\gamma^0} = 0$ siden $\gamma^0$ er diagonal og består af tal, at $\gamma^0 \gamma^0 = \id$ og at $\gamma^0 (\gamma^\mu)\dagger = \gamma^\mu \gamma^0$ vist i \cref{eq:Opg4_A3_gamma0*(gamma^mu)^dagger=gamma^mu*gamma0}.
\\

Samme fremgangsmåde benyttes for det andet matrixelement, \cref{eq:Opg4_Q5_TransitionCurrent2}, men her benyttes relationerne $(\slashed{p} + m)v(\vv{p}, \lambda) = 0$ (igen fra \cite[opgave 2.4]{problemSet3}) og $\overline{v}(\vv{p}, \lambda) (\slashed{p} + m) = 0$, hvor sidstnævnte kan findes ved samme udregninger som i \cref{eq:Opg4_A6_BaredVersionOfDiracLikeEquationFromProbSet3Ex2.4},
\begin{align} \label{eq:Opg4_A6_ConservedTransitionCurrent2}
\begin{split}
    \partial^\mu \mel**{\mathrm{e}^+,\, \vv{p}',\, \lambda'}{j_\mu^{(2)}}{\mathrm{e}^+,\, \vv{p},\, \lambda}
        &= \partial^\mu \big[ e\bar{v}(\vv{p},\lambda) \gamma_\mu v(\vv{p}',\lambda') \pexp{i[p' - p]x} \big] \\
        &= ie\bar{v}(\vv{p},\lambda) \gamma_\mu (p'^\mu - p^\mu) v(\vv{p}',\lambda') \pexp{i[p' - p]x} \\
        &= ie\bar{v}(\vv{p},\lambda) (\slashed{p}' - \slashed{p}) v(\vv{p}',\lambda') \pexp{i[p' - p]x} \\
        &= ie\bar{v}(\vv{p},\lambda) \big(-m - [-m]\big) v(\vv{p}',\lambda') \pexp{i[p' - p]x} \\
        &= 0 \: .
\end{split}
\end{align}
\\

For både det tredje matrixelement, \cref{eq:Opg4_Q5_TransitionCurrent3}, og det fjerde matrixelement, \cref{eq:Opg4_Q5_TransitionCurrent4}, benyttes samme fremgangsmåde som for de to foregående, så
\begin{align} \label{eq:Opg4_A6_ConservedTransitionCurrent3}
\begin{split}
    \partial^\mu \mel**{0}{j_\mu^{(3)}}{\mathrm{e}^-,\, \vv{p},\, \lambda;\, \mathrm{e}^+,\, \vv{p}',\, \lambda'}
        &= \partial^\mu \big[ -e\bar{v}(\vv{p}',\lambda') \gamma_\mu u(\vv{p},\lambda) \pexp{-i[p' + p]x} \big] \\
        &= ie\bar{v}(\vv{p}',\lambda') \gamma_\mu (p'^\mu + p^\mu) u(\vv{p},\lambda) \pexp{-i[p' + p]x} \\
        &= ie\bar{v}(\vv{p}',\lambda') (\slashed{p}' + \slashed{p}) u(\vv{p},\lambda) \pexp{-i[p' + p]x} \\
        &= ie\bar{v}(\vv{p}',\lambda') (-m + m) u(\vv{p},\lambda) \pexp{-i[p' + p]x} \\
        &= 0 \: ,
\end{split}
\end{align}
%
\begin{align} \label{eq:Opg4_A6_ConservedTransitionCurrent4}
\begin{split}
    \partial^\mu \mel**{\mathrm{e}^-,\, \vv{p}',\, \lambda';\, \mathrm{e}^+,\, \vv{p},\, \lambda}{j_\mu^{(4)}}{0}
        &= \partial^\mu \big[ -e\bar{u}(\vv{p}',\lambda') \gamma_\mu v(\vv{p},\lambda) \pexp{i[p' + p]x} \big] \\
        &= -ie\bar{u}(\vv{p}',\lambda') \gamma_\mu (p'^\mu + p^\mu) v(\vv{p},\lambda) \pexp{i[p' + p]x} \\
        &= -ie\bar{u}(\vv{p}',\lambda') (\slashed{p}' + \slashed{p}) v(\vv{p},\lambda) \pexp{i[p' + p]x} \\
        &= -ie\bar{u}(\vv{p}',\lambda') \big(m + [-m]\big) v(\vv{p},\lambda) \pexp{i[p' + p]x} \\
        &= 0 \: .
\end{split}
\end{align}
\\

Dermed ses det tydeligt, at alle fire overgangsstrømme er bevarede i den forstand, at $\partial^\mu J_\mu^{fi} = 0$, \cref{eq:Opg4_A6_ConservedTransitionCurrent1,eq:Opg4_A6_ConservedTransitionCurrent2,eq:Opg4_A6_ConservedTransitionCurrent3,eq:Opg4_A6_ConservedTransitionCurrent4}.


%%%%%%%%%%%%%%%%%%%%%%%%%

\paragraph[7) Beregning på fysiske processer i QED giver led på formen $J_\mu^{fi}(0)\epsilon^\mu(\sigma)$]{\textbf{7)}}

Med en Lagrangedensitet for virkningen på formen $\L_I = - J_\mu A^\mu$ vil vi have en Hamiltondensitet for vekselvirkningen, som er $\H_I = J_\mu A^\mu$, da $\L_I$ ikke indeholder nogle differentialopeartorled, som ellers ville ''besværliggøre'' Legendretransformationen. Betragter vi nu kvantiseringen af feltet for en foton, \cite[lign. 8]{Q&A10}
\begin{align}
    A^\mu(x) &= \sum_\sigma \int \frac{\dd^3 \vv{p}}{(2\pi)^3} \invsqrt{2 \omega_{\vv{p}}} \left[ \epsilon^\mu(\sigma) a_{\vv{p},\sigma} \pexp{-ipx} + \epsilon^\mu(\sigma)^* a_{\vv{p},\sigma}\dagger \pexp{ipx} \right] \: ,
\end{align}
samt benytter Dysonserien og tidsafhængigperturbationsteori, som vi har lært i kurset, da er det klart at vi vil få matrixelementer på formen (\cite[lign. 53--54]{problemSet2})
\begin{align}
    \mel**{f}{J_\mu(x) A^\mu(x)}{i} \propto \mel**{f}{J_\mu(0)}{i} \epsilon^\mu(\sigma)
        \propto J_\mu^{fi}(0) \epsilon^\mu(\sigma)
    \: ,
\end{align}
da feltudvidelsen for $A^\mu$ kun indeholder impulsrumsintegration og dermed kan tages ud af matrixelementet, samt at $J_\mu(x) \propto J_\mu(0)$, da vi kan translatere operatorer med komplekse eksponentialfunktioner, og at kun overgangsstrømmene fra \textbf{5)} (\cref{eq:Opg4_Q5_TransitionCurrents}) ikke-trivielle, altså forskellige fra $0$. $J_\mu^{fi}(x)$ er overgangsstrømmene regnet i \textbf{5)}.
\\

Når vi taler sandsynlighed, så tages normkvadratet af matrixelementet, altså
\begin{align}
    \abs{J_\mu^{fi}(0) \epsilon^\mu(\sigma)}^2 &= J_\mu^{fi}(0)^* \epsilon^\mu(\sigma)^* J_\nu^{fi}(0) \epsilon^\nu(\sigma)
        = \epsilon^\mu(\sigma)^* \epsilon^\nu(\sigma) J_\mu^{fi}(0)^* J_\nu^{fi}(0) \: .
\end{align}
\\

Går vi til højere orden i perturbationen, da vil vil blot få flere matrixelementer på denne form, da vi vil få nogen ''mellemliggende'' (eng: intermediate) tilstande.



%%%%%%%%%%%%%%%%%%%%%%%%%

\paragraph[8) Sum af polarisationstilstande for en reel foton]{\textbf{8)}}

Fra \textbf{opgave 1.2} (\textbf{delopgave 2} i \textbf{Spin-1 partikler med masse}) i \cref{sec:Opg1_Q2} er det givet, at $\epsilon_\mu q^\mu = 0$ samt at $\epsilon_\mu(\sigma) \epsilon^\mu(\sigma') = -\delta_{\sigma\sigma'}$. Fra første krav kan vi se, at siden $q^\mu = (q_0,\, 0,\, 0,\, q_0)$, da skal $\epsilon^0 = \epsilon^3 = 0$ \footnote{
    Yderlige giver det mening at $\epsilon^3 = 0$, idet at fotoner, da de er masseløse, ikke kan være polariserede i propageringsretningen, som her trivielt er valgt til at være $\zhat$-retningen.
}. Sidstnævnte krav skyldes, at polarisationstilstandene skal være ortogonale, da vi kræver, at en partikel kun kan være i én tilstand (enten den ene eller den anden). Vi har i \textbf{opgave 1: Spin-1-partikler med masse} set to mulige sæt af tilstande, som opfylder begge krav: De lineært polariserede tilstande $e_x^\mu$ og $e_y^\mu$ fra \cref{eq:Opg1_Q3_PolarisationStatesInRestFrame} samt de cirkulært polariserede tilstande $e^\mu(\sigma = \pm 1)$ fra \cref{eq:Opg1_Q4_HelicityStates},
\begin{align}
    e^\mu(\sigma = \pm 1) &= \left( 0,\, \mp \frac{\xhat \pm i \yhat}{\sqrtTo} \right)
        = \invsqrtTo \FourRowMat{0}{\mp 1}{-i}{0} \: ,
\end{align}
Det ses trivielt fra de lineært polariserede tilstande, at $\sum_\sigma \epsilon^\mu(\sigma)^* \epsilon^\nu(\sigma) = \delta_1^\mu \delta_1^\nu + \delta_2^\mu \delta_2^\nu$, hvorfor det også må gælde for de cirkulært polariserede tilstande, da disse blot er sammenhængende ved en unitære transformationer (se \textbf{Spin-1 partikler med masse}), men dette kan også eksplicit vises. $\sum_\sigma \epsilon^\mu(\sigma)^* \epsilon^\nu(\sigma)$ bliver for de cirkulært polariserede tilstande derved følgende, hvor ledene med $\epsilon^0$ og $\epsilon^3$ trivielt er $0$:
\begin{subequations}
\begin{align}
    \sum_\sigma \epsilon^1(\sigma)^* \epsilon^1(\sigma) &= \frac{-1}{\sqrtTo} \frac{-1}{\sqrtTo} + \frac{1}{\sqrtTo} \frac{1}{\sqrtTo}
        = \frac{1}{2} + \frac{1}{2}
        = 1 \: , \\
    \sum_\sigma \epsilon^1(\sigma)^* \epsilon^2(\sigma) &= \frac{-1}{\sqrtTo} \frac{-i}{\sqrtTo} + \frac{1}{\sqrtTo} \frac{-i}{\sqrtTo}
        = \frac{i}{2} + \frac{-i}{2}
        = 0 \: , \\
    \sum_\sigma \epsilon^2(\sigma)^* \epsilon^1(\sigma) &= \frac{i}{\sqrtTo} \frac{-1}{\sqrtTo} + \frac{i}{\sqrtTo} \frac{1}{\sqrtTo}
        = \frac{-i}{2} + \frac{i}{2}
        = 0 \: , \quad \text{og} \\
    \sum_\sigma \epsilon^2(\sigma)^* \epsilon^2(\sigma) &= \frac{i}{\sqrtTo} \frac{-i}{\sqrtTo} + \frac{i}{\sqrtTo} \frac{-i}{\sqrtTo}
        = \frac{1}{2} + \frac{1}{2}
        = 1 \: .
\end{align}
\end{subequations}
Dermed er \cref{eq:Opg4_Q8_IdentityWithSumOfPolarisation}, $\sum_\sigma \epsilon^\mu(\sigma)^* \epsilon^\nu(\sigma) = \delta_1^\mu \delta_1^\nu + \delta_2^\mu \delta_2^\nu$, vist.


%%%%%%%%%%%%%%%%%%%%%%%%%

\paragraph[9) Polarisationstilstande og den metriske tensor for en reel foton]{\textbf{9)}}

Først betragtes venstresiden af \cref{eq:Opg4_Q9_Result}, og benyttes identiteten vist i \cref{eq:Opg4_Q8_IdentityWithSumOfPolarisation} fås
\begin{align} \label{eq:Opg4_A9_LeftHandSide}
\begin{split}
    \left[ \sum_\sigma \epsilon^\mu(\sigma)^* \epsilon^\nu(\sigma) \right] J_\mu^{fi}(0)^* \J_\nu^{fi}(0)
        &= \big[ \delta_1^\mu \delta_1^\nu + \delta_2^\mu \delta_2^\nu \big] J_\mu^{fi}(0)^* \J_\nu^{fi}(0) \\
        &= \abs{J_1^{fi}(0)}^2 + \abs{\J_2^{fi}(0)}^2 \: .
\end{split}
\end{align}

Betragtes nu højresiden af \cref{eq:Opg4_Q9_Result} fås
\begin{align} \label{eq:Opg4_A9_RightHandSide}
    - g^{\mu\nu} J_\mu^{fi}(0)^* \J_\nu^{fi}(0) &= - \abs{J_0^{fi}(0)}^2 + \abs{J_1^{fi}(0)}^2 + \abs{J_2^{fi}(0)}^2 + \abs{J_3^{fi}(0)}^2 \: ,
\end{align}
da $g^{\mu\nu} = \mathrm{diag}(1,\, -\id)$. Det huskes at $J_\mu^{fi}(x)$ er en bevaret strøm (fra \textbf{6)}) samt at $J_\mu^{fi}(x) = \pexp{iqx} J_\mu^{fi}(0)$ fra \cref{eq:Opg4_Q5_TransitionCurrents}, hvor $q = p_f - p_i$ er fotonens impulsoverførsel ($p_i$ og $p_f$ er hhv. start- og sluttlistandens impuls), hvorfor
\begin{align} \label{eq:Opg4_A9_CalculationOfJ0=J3}
\begin{split}
    0 &= \partial^\mu J_\mu^{fi}(x) \\
        &= \partial^\mu \left[ \pexp{iqx} J_\mu^{fi}(0) \right] \\
        &= i q^\mu \pexp{iqx} J_\mu^{fi}(0) \\
        &= i q^\mu J_\mu^{fi}(x) \\
        &= i q_0 \left[ J_0^{fi}(x) - J_3^{fi}(x) \right] \\
    \Rightarrow
    J_0^{fi}(x) &= J_3^{fi}(x) \\
    \Rightarrow
    \abs{J_0^{fi}(0)}^2 &= \abs{J_3^{fi}(0)}^2 \: ,
\end{split}
\end{align}
da vi fra \textbf{8)} har at $q^\mu = (q_0,\, 0,\, 0,\, q_0)$. Indsættes dette i \cref{eq:Opg4_A9_RightHandSide} og sammenligned med \cref{eq:Opg4_A9_LeftHandSide} fås
\begin{align}
\begin{split}
    - g^{\mu\nu} J_\mu^{fi}(0)^* \J_\nu^{fi}(0) &= - \abs{J_0^{fi}(0)}^2 + \abs{J_1^{fi}(0)}^2 + \abs{J_2^{fi}(0)}^2 + \abs{J_3^{fi}(0)}^2 \\
        &= \abs{J_1^{fi}(0)}^2 + \abs{\J_2^{fi}(0)}^2 \\
        &= \left[ \sum_\sigma \epsilon^\mu(\sigma)^* \epsilon^\nu(\sigma) \right] J_\mu^{fi}(0)^* \J_\nu^{fi}(0) \: ,
\end{split}
\end{align}
hvilket er \cref{eq:Opg4_Q9_Result} som skulle vises.


%%%%%%%%%%%%%%%%%%%%%%%%%

\paragraph[10) Polarisationstilstande og den metriske tensor for reelle og \\ virtuelle fotoner]{\textbf{10)}}

Betragter vi udregningen i \cref{eq:Opg4_Q9_Result}, da ser vi, at vi har benyttet bevarelse af $J_\mu^{fi}$ er bevaret.
% Vi kan se udtrykket i \cref{eq:Opg4_Q10_ExpressionToBeTrue} som værende resultatet fra egenværdiligningen fra \textbf{9)}, \cref{eq:Opg4_Q9_Result}, så længe at $J_\mu^{fi}(0) \ne 0$.
\\

Vi skal nu kigge på hvilke yderligere led, som kunne opstå i \cref{eq:Opg4_Q10_ExpressionToBeTrue} for en reel og en virtuel foton. Et sådan ekstra led ville være en enhedsløs rank-2-tensor, således at den kan sammenlægges med metrikken, som også er en enhedsløs rank-2-tensor,
\begin{align}
    \sum_\sigma \epsilon^\mu(\sigma)^* \epsilon^\nu(\sigma) &= -g^{\mu\nu} + C^{\mu\nu} \: ,
\end{align}
og denne tensor skal overholde, at $C^{\mu\nu} J_\mu^{fi}(0)^* J_\nu^{fi}(0) = 0$, således at \cref{eq:Opg4_Q9_Result} stadig overholdes. For et system med kendte parametre, kan en sådan tensor kun konstrueres ved $C^{\mu\nu} = q^\mu q^\nu / q^2$, hvilket er veldefineret for en virtuel foton med $q^2 \ne 0$. Dette led dur dog ikke for en reel foton, da denne er masseløs, hvorfor $q_{\mathrm{reel}}^2 = 0$.


%%%%%%%%%%%%%%%%%%%%%%%%%

\paragraph[11) Feynmandiagram for spredning af elektroner på myoner]{\textbf{11)}}

Feynmandiagrammet for spredningen af elektroner på myoner, $\mathrm{e}^- + \mu^- \rightarrow \mathrm{e}^- + \mu^-$, til anden orden er vist herunder, og tiden er valgt til at gå opad.

\begin{tikzpicture}
    \begin{feynman}
        \vertex (vertexLeft);
        \vertex [right = 6em of vertexLeft] (vertexRight);
        \vertex [above left = 6em of vertexLeft] (upLeft) {\(e^-\)};
        \vertex [below left = 6em of vertexLeft] (downLeft) {\(e^-\)};
        \vertex [above right = 6em of vertexRight] (upRight) {\(\mu^-\)};
        \vertex [below right = 6em of vertexRight] (downRight) {\(\mu^-\)};
        \vertex [above left = 3em of downLeft] (timeBot);
        \vertex [above = 6em of timeBot] (timeTop);
        \diagram*{
            (timeBot) -- [draw = none, momentum = $t$] (timeTop),
            (downLeft) -- [fermion, momentum' = $p_1$] (vertexLeft) -- [fermion, momentum' = $p_3$] (upLeft),
            % momentum and momentum' changed positions (above below) of label on path
            (downRight) -- [fermion, momentum = $p_2$] (vertexRight) -- [fermion, momentum = $p_4$] (upRight),
            (vertexLeft) -- [photon, edge label = $\gamma$, momentum' = $q$] (vertexRight),
            % momentum and momentum' changed positions (above below) of label on path
        };
    \end{feynman}
\end{tikzpicture}


%%%%%%%%%%%%%%%%%%%%%%%%%

\paragraph[12) S-matricen til anden orden for elektron-myon-spredningen]{\textbf{12)}}

Lagrangedentiteten for vekselvirkningen er $\L_I = - (j_\mu^{\mathrm{e}^-} + j_\mu^{\mu^-}) A^\mu = - J_\mu A^\mu$, hvor $j_\mu^{\mathrm{e}^-}$ og $j_\mu^{\mu^-}$ er strømmen for hhv. elektronen og myonen og $J_\mu$ er den totale firstrøm, og da $\L_I$ ikke indeholder nogle differentialer, så er Hamiltondensiteten af vekselvirkningen $\H = - \L_I = J_\mu A^\mu$. Fra \cite[lign. 56 og 59]{problemSet2} har vi, at S-matrixelementet til anden orden kan skrives som
\begin{align}
    S_{fi}^{(2)} &= \frac{(-i)^2}{2} \int \dd^4 x_1\, \dd^4 x_2\, \mel**{f}{T[\H_I(x_1) \H_I(x_2)]}{i} \: .
\end{align}
Siden udregningen er for spredningen af en elektron med impuls $\vv{p_1}$ og helicitet $\lambda_1$ over i en elektron med $\vv{p}_3$ og $\lambda_3$, når den spredes på en myon med impuls $\vv{p_2}$ og helicitet $\lambda_2$ over i en elektron med $\vv{p}_4$ og $\lambda_4$, da bliver start- og sluttilstanden
\begin{align}
    \ket{i} &= \ket{\mathrm{e}^-,\, \vv{p_1},\, \lambda_1;\, \mu^-,\, \vv{p_2},\, \lambda_2}
        \qquad \text{og} \qquad
    \ket{f} = \ket{\mathrm{e}^-,\, \vv{p_3},\, \lambda_3;\, \mu^-,\, \vv{p_4},\, \lambda_4} \: .
\end{align}
Vi kan nu udregne S-matrixelementet
\begin{align} \label{eq:Opg4_A12_CaulculationOfSMatrixPart1}
\begin{split}
    S_{fi}^{(2)} &= \frac{(-i)^2}{2} \int \dd^4 x_1\, \dd^4 x_2\, \mel**{f}{T[\H_I(x_1) \H_I(x_2)]}{i} \\
        &= \frac{(-i)^2}{2} \int \dd^4 x_1\, \dd^4 x_2\, \mel**{f}{T[J_\mu(x_1) A^\mu(x_1) J_\nu(x_2) A^\nu(x_2)]}{i} \\
        &= \frac{(-i)^2}{2} \int \dd^4 x_1\, \dd^4 x_2\, \mel**{f}{J_\mu(x_1) J_\nu(x_2)}{i} \mel**{0}{T[A^\mu(x_1) A^\nu(x_2)]}{0} \\
        &= \frac{(-i)^2}{2} \int \dd^4 x_1\, \dd^4 x_2\, \mel**{f}{\left( j_\mu^{\mathrm{e}^-}(x_1) + j_\mu^{\mu^-}(x_1) \right) \left( j_\nu^{\mathrm{e}^-}(x_2) + j_\nu^{\mu^-}(x_2) \right)}{i} \\
            &\qquad\qquad\qquad\qquad\qquad\qquad \times \mel**{0}{T[A^\mu(x_1) A^\nu(x_2)]}{0} \\
        &= \frac{(-i)^2}{2} \int \dd^4 x_1\, \dd^4 x_2\, \bigg( \mel**{f}{j_\mu^{\mathrm{e}^-}(x_1) j_\nu^{\mathrm{e}^-}(x_2)}{i} + \mel**{f}{j_\mu^{\mathrm{e}^-}(x_1) j_\nu^{\mu^-}(x_2)}{i} \\
            &\qquad\qquad\qquad\qquad\: + \mel**{f}{j_\mu^{\mu^-}(x_1) j_\nu^{\mathrm{e}^-}(x_2)}{i} + \mel**{f}{j_\mu^{\mu^-}(x_1) j_\nu^{\mu^-}(x_2)}{i} \bigg)
            \\
            &\qquad\qquad\qquad\qquad\quad \times \mel**{0}{T[A^\mu(x_1) A^\nu(x_2)]}{0} \\
        &= \frac{(-i)^2}{2} \int \dd^4 x_1\, \dd^4 x_2 \\
            &\quad \times \bigg( \mel**{\mathrm{e}^-,\, \vv{p_3},\, \lambda_3}{j_\mu^{\mathrm{e}^-}(x_1) j_\nu^{\mathrm{e}^-}(x_2)}{\mathrm{e}^-,\, \vv{p_1},\, \lambda_1} \innerproduct{\mu^-,\, \vv{p_4},\, \lambda_4}{\mu^-,\, \vv{p_2},\, \lambda_2} \\
            &\qquad\:\: + \mel**{\mathrm{e}^-,\, \vv{p_3},\, \lambda_3}{j_\mu^{\mathrm{e}^-}(x_1)}{\mathrm{e}^-,\, \vv{p_1},\, \lambda_1} \mel**{\mu^-,\, \vv{p_4},\, \lambda_4}{j_\nu^{\mu^-}(x_2)}{\mu^-,\, \vv{p_2},\, \lambda_2} \\
            &\qquad\:\: + \mel**{\mu^-,\, \vv{p_4},\, \lambda_4}{j_\mu^{\mu^-}(x_1)}{\mu^-,\, \vv{p_2},\, \lambda_2} \mel**{\mathrm{e}^-,\, \vv{p_3},\, \lambda_3}{j_\nu^{\mathrm{e}^-}(x_2)}{\mathrm{e}^-,\, \vv{p_1},\, \lambda_1} \\
            &\qquad\:\: + \innerproduct{\mathrm{e}^-,\, \vv{p_3},\, \lambda_3}{\mathrm{e}^-,\, \vv{p_1},\, \lambda_1} \mel**{\mu^-,\, \vv{p_4},\, \lambda_4}{j_\mu^{\mu^-}(x_1) j_\nu^{\mu^-}(x_2)}{\mu^-,\, \vv{p_2},\, \lambda_2} \bigg)
            \\
            &\quad \times \mel**{0}{T[A^\mu(x_1) A^\nu(x_2)]}{0} \: ,
\end{split}
\end{align}
hvor vi har kunnet opdele matrixelementerne i den sidste lighed, da operatorerne kun virker på deres eget rum, altså $j_\mu^{\mathrm{e}^-}$ virker kun på $\ket{\mathrm{e}^-}$.

Af de fire strømled i \cref{eq:Opg4_A12_CaulculationOfSMatrixPart1} svarer det første og sidste blot til hhv. at elektronen og myonen vekselvirker med sig selv gennem en foton, men siden dette ikke er spredningen, som vi regner på, da kan disse led negligeres. Kigger vi nu på det tredje led og lader $x_1 \leftrightarrow x_2$ og $\mu \leftrightarrow \nu$ får vi, at
\begin{align}
\begin{split}
    &\mel**{\mu^-,\, \vv{p_4},\, \lambda_4}{j_\nu^{\mu^-}(x_2)}{\mu^-,\, \vv{p_2},\, \lambda_2} \mel**{\mathrm{e}^-,\, \vv{p_3},\, \lambda_3}{j_\mu^{\mathrm{e}^-}(x_1)}{\mathrm{e}^-,\, \vv{p_1},\, \lambda_1} \\
        &\qquad\qquad\qquad\qquad\qquad\qquad\qquad \times \mel**{0}{T[A^\nu(x_2) A^\mu(x_1)]}{0} \\
    &\quad = \mel**{\mathrm{e}^-,\, \vv{p_3},\, \lambda_3}{j_\mu^{\mathrm{e}^-}(x_1)}{\mathrm{e}^-,\, \vv{p_1},\, \lambda_1} \mel**{\mu^-,\, \vv{p_4},\, \lambda_4}{j_\nu^{\mu^-}(x_2)}{\mu^-,\, \vv{p_2},\, \lambda_2} \\
        &\qquad\qquad\qquad\qquad\qquad\qquad\qquad \times \mel**{0}{T[A^\mu(x_1) A^\nu(x_2)]}{0} \: ,
\end{split}
\end{align}
da $T[A^\mu(x_1) A^\nu(x_2)] = T[A^\nu(x_2) A^\mu(x_1)]$. Dermed er andet og tredje led i \cref{eq:Opg4_A12_CaulculationOfSMatrixPart1} ens, hvorfor S-matrixelementet til anden orden bliver
\begin{align}
\begin{split}
    S_{fi}^{(2)} &= (-i)^2 \int \dd^4 x_1\, \dd^4 x_2 \mel**{\mathrm{e}^-,\, \vv{p_3},\, \lambda_3}{j_\mu^{\mathrm{e}^-}(x_1)}{\mathrm{e}^-,\, \vv{p_1},\, \lambda_1} \\
            &\qquad\qquad\quad \times \mel**{\mu^-,\, \vv{p_4},\, \lambda_4}{j_\nu^{\mu^-}(x_2)}{\mu^-,\, \vv{p_2},\, \lambda_2} \mel**{0}{T[A^\mu(x_1) A^\nu(x_2)]}{0} \\
            &= (-i)^2 \int \dd^4 x_1\, \dd^4 x_2\, J_\mu^{\mathrm{e}^-}(x_1) J_\nu^{\mu^-}(x_2) \mel**{0}{T\left[A^\mu(x_1)A^\nu(x_2)\right]}{0}
\end{split}
\end{align}
hvor vi har defineret
\begin{subequations}
\begin{align}
    J_\mu^{\mathrm{e}^-}(x_1) &= \mel**{\mathrm{e}^-,\, \vv{p_3},\, \lambda_3}{j_\mu^{\mathrm{e}^-}(x_1)}{\mathrm{e}^-,\, \vv{p_1},\, \lambda_1} \qquad \text{og} \\
    J_\nu^{\mu^-}(x_2) &= \mel**{\mu^-,\, \vv{p_4},\, \lambda_4}{j_\nu^{\mu^-}(x_2)}{\mu^-,\, \vv{p_2},\, \lambda_2} \: ,
\end{align}
\end{subequations}
hvilke er præcis på formen for \cref{eq:Opg4_Q5_TransitionCurrent1} og dermed kun giver noget for $j_\mu^{(1)}$-strømmen fra \cref{eq:Opg4_A4_FourPartsOfTheCurrectExplicitJ1}, hvorfor disse eksplicit bliver
\begin{subequations} \label{eq:Opg4_A12_ExplicitElectronAndMuonCurrent}
\begin{align}
    J_\mu^{\mathrm{e}^-}(x_1) &= -e\bar{u}(\vv{p}_3,\lambda_3) \gamma_\mu u(\vv{p}_1,\lambda_1) \pexp{i[p_3 - p_1]x_1} \qquad \text{og} \\
    J_\nu^{\mu^-}(x_2) &= -e\bar{u}(\vv{p}_4,\lambda_4) \gamma_\mu u(\vv{p}_2,\lambda_2) \pexp{i[p_4 - p_2]x_2} \: .
\end{align}
\end{subequations}

Dermed er \cref{eq:Opg4_Q12_SMatrixSecondOrder} vist.

% Vis at S-matricen til anden orden for elektron-myon-spredningsprocessen kan skrives som
% \begin{align}
%     S_{fi}^{(2)} &= (-i)^2 \int \dd^4 x_1 \dd^4 x_2 J_\mu^{\mathrm{e}^-}(x_1) J_\nu^{\mu^-}(x_2) \mel**{0}{T\left[A^\mu(x_1)A^\nu(x_2)\right]}{0} \: .
% \end{align}
% Opskriv eksplicit elektron- og myonstrømmen (eng: electron and muon current) ved brug af overgangsstrømmene fra \textbf{5)}. Sørg for at mærke alle dele nødvendige for at specificere start- og sluttilstandene korrekt.


%%%%%%%%%%%%%%%%%%%%%%%%%

\paragraph[13) Fotonpropagator ved analogi med Klein-Gordon-proagatoren]{\textbf{13)}}

For $G^{\mu\nu}(q)$ værende impulsrumspropagatoren for fotonen må $\mel{0}{T[A^\mu(x) A^\nu(0)]}{0}$ være stedrumsfotonpropagatoren grundet den inverse Fouriertransformation. At $\mel{0}{T[A^\mu(x) A^\nu(0)]}{0}$ giver mening som værende fotonpropagatoren i stedrum vises herunder.
\\

Kvantiseringen af feltet for en foton er \cite[lign. 8]{Q&A10}
\begin{align}
    A^\mu(x) &= \sum_\sigma \int \frac{\dd^3 \vv{p}}{(2\pi)^3} \invsqrt{2 \omega_{\vv{p}}} \left[ \epsilon^\mu(\sigma) a_{\vv{p},\sigma} \pexp{-ipx} + \epsilon^\mu(\sigma)^* a_{\vv{p},\sigma}\dagger \pexp{ipx} \right] \: ,
\end{align}
så hvis stedrumspropagatoren udregnes, fås\\
\begin{align} \label{eq:Opg4_A13_CalculatingThePropagatorPart1}
\begin{split}
    &\mel**{0}{T \left[ A^\mu(x) A^\nu(0) \right]}{0} \\
        &\quad = \Bigg\langle 0 \Bigg\rvert \left\{ \sum_\sigma \int \frac{\dd^3 \vv{p}}{(2\pi)^3} \invsqrt{2 \omega_{\vv{p}}} \left[ \epsilon^\mu(\sigma) a_{\vv{p},\sigma} \pexp{-ipx} + \epsilon^\mu(\sigma)^* a_{\vv{p},\sigma}\dagger \pexp{ipx} \right] \right\} \\
            &\qquad\qquad \times \left\{ \sum_{\sigma'} \int \frac{\dd^3 \vv{p}'}{(2\pi)^3} \invsqrt{2 \omega_{\vv{p}'}} \left[ \epsilon^\nu(\sigma') a_{\vv{p}',\sigma'} + \epsilon^\nu(\sigma')^* a_{\vv{p}',\sigma'}\dagger \right] \right\} \Bigg\lvert 0 \Bigg\rangle \\
        &\quad = \theta(t) \sum_{\sigma,\sigma'} \int \frac{\dd^3 \vv{p}\, \dd^3 \vv{p}'}{(2\pi)^6} \inv{2 \sqrt{\omega_{\vv{p}} \omega_{\vv{p}'}}} \Big\langle 0 \Big\rvert \left[ \epsilon^\mu(\sigma) a_{\vv{p},\sigma} \pexp{-ipx} + \epsilon^\mu(\sigma)^* a_{\vv{p},\sigma}\dagger \pexp{ipx} \right] \\
            &\qquad\qquad\qquad\qquad\qquad\qquad\qquad\qquad \times \left[ \epsilon^\nu(\sigma') a_{\vv{p}',\sigma'} + \epsilon^\nu(\sigma')^* a_{\vv{p}',\sigma'}\dagger \right] \Big\lvert 0 \Big\rangle \\
            &\qquad\: + \theta(-t) \sum_{\sigma,\sigma'} \int \frac{\dd^3 \vv{p}\, \dd^3 \vv{p}'}{(2\pi)^6} \inv{2 \sqrt{\omega_{\vv{p}} \omega_{\vv{p}'}}} \Big\langle 0 \Big\rvert \left[ \epsilon^\nu(\sigma') a_{\vv{p}',\sigma'} + \epsilon^\nu(\sigma')^* a_{\vv{p}',\sigma'}\dagger \right] \\
            &\qquad\qquad\qquad\qquad\qquad\qquad\quad \times \left[ \epsilon^\mu(\sigma) a_{\vv{p},\sigma} \pexp{-ipx} + \epsilon^\mu(\sigma)^* a_{\vv{p},\sigma}\dagger \pexp{ipx} \right] \Big\lvert 0 \Big\rangle \\
        &\quad = \theta(t) \sum_{\sigma,\sigma'} \int \frac{\dd^3 \vv{p}\, \dd^3 \vv{p}'}{(2\pi)^6} \inv{2 \sqrt{\omega_{\vv{p}} \omega_{\vv{p}'}}} \mel**{0}{\epsilon^\mu(\sigma) a_{\vv{p},\sigma} \pexp{-ipx} \epsilon^\nu(\sigma')^* a_{\vv{p}',\sigma'}\dagger}{0} \\
            &\qquad\: + \theta(-t) \sum_{\sigma,\sigma'} \int \frac{\dd^3 \vv{p}\, \dd^3 \vv{p}'}{(2\pi)^6} \inv{2 \sqrt{\omega_{\vv{p}} \omega_{\vv{p}'}}} \mel**{0}{\epsilon^\nu(\sigma') a_{\vv{p}',\sigma'} \epsilon^\mu(\sigma)^* a_{\vv{p},\sigma}\dagger \pexp{ipx}}{0} \\
        &\quad = \theta(t) \sum_{\sigma,\sigma'} \int \frac{\dd^3 \vv{p}\, \dd^3 \vv{p}'}{(2\pi)^6} \inv{2 \sqrt{\omega_{\vv{p}} \omega_{\vv{p}'}}} \epsilon^\mu(\sigma) \epsilon^\nu(\sigma')^* \pexp{-ipx} (2\pi)^3 \delta^3(\vv{p} - \vv{p}') \delta_{\sigma,\sigma'} \\
            &\qquad\: + \theta(-t) \sum_{\sigma,\sigma'} \int \frac{\dd^3 \vv{p}\, \dd^3 \vv{p}'}{(2\pi)^6} \inv{2 \sqrt{\omega_{\vv{p}} \omega_{\vv{p}'}}} \epsilon^\nu(\sigma') \epsilon^\mu(\sigma)^* \pexp{ipx} (2\pi)^3 \delta^3(\vv{p} - \vv{p}') \delta_{\sigma,\sigma'} \\
        &\quad = \theta(t) \sum_{\sigma} \epsilon^\mu(\sigma) \epsilon^\nu(\sigma)^* \int \frac{\dd^3 \vv{p}}{(2\pi)^3} \inv{2 \omega_{\vv{p}}} \pexp{-ipx} \\
            &\qquad\: + \theta(-t) \sum_{\sigma} \epsilon^\nu(\sigma) \epsilon^\mu(\sigma)^* \int \frac{\dd^3 \vv{p}}{(2\pi)^3} \inv{2 \omega_{\vv{p}}} \pexp{ipx} \\
        &\quad = \theta(t) (-g^{\mu\nu}) \int \frac{\dd^3 \vv{p}}{(2\pi)^3} \inv{2 \omega_{\vv{p}}} \pexp{-ipx} + \theta(-t) (-g^{\nu\mu}) \int \frac{\dd^3 \vv{p}}{(2\pi)^3} \inv{2 \omega_{\vv{p}}} \pexp{ipx} \\
        &\quad = -g^{\mu\nu} \left[ \theta(t) \int \frac{\dd^3 \vv{p}}{(2\pi)^3} \inv{2 \omega_{\vv{p}}} \pexp{-ipx} + \theta(-t) \int \frac{\dd^3 \vv{p}}{(2\pi)^3} \inv{2 \omega_{\vv{p}}} \pexp{ipx} \right] \: ,
\end{split}
\end{align}
hvor identiteten fra \textbf{10)}, \cref{eq:Opg4_Q10_ExpressionToBeTrue}, er blevet benyttet, hvor det også noteres, at $\sum_\sigma \epsilon^\mu(\sigma) \epsilon^\nu(\sigma)^* = [\sum_\sigma \epsilon^\mu(\sigma)^* \epsilon^\nu(\sigma)]^* = [-g^{\mu\nu}]^* = -g^{\mu\nu}$, da $g^{\mu\nu} = \mathrm{diag}(1,\, -\id)$. Yderligere er det blevet benyttet, at kun ledene, hvor en partikel først skabes og derefter annihileres overlever, idet at $a_{\vv{p},\sigma} \ket{0} = 0 \xleftrightarrow{\text{DC}} \bra{0} a_{\vv{p},\sigma}\dagger = 0$, samt gjort brug at de bosoniske kommutatorrelationer, $[a_{\vv{p},\sigma},\, a_{\vv{p}',\sigma'}\dagger] = (2\pi)^3 \delta^3(\vv{p} - \vv{p}') \delta_{\sigma,\sigma'}$.

Udtrykket i parentesen i \cref{eq:Opg4_A13_CalculatingThePropagatorPart1} er præcis Klein-Gordon-propagatoren $\mel{0}{T[\phi(x) \phi(0)]}{0}$ for et skalarfelt $\phi$ (\cite[lign. 40]{problemSet3}), så
\begin{align} \label{eq:Opg4_A13_CalculatingThePropagatorPart2}
    \mel**{0}{T \left[ A^\mu(x) A^\nu(0) \right]}{0} &= - g^{\mu\nu} \mel**{0}{T \left[ \phi(x) \phi(0) \right]}{0} \: ,
\end{align}
hvorfor det giver mening at $\mel{0}{T[A^\mu(x) A^\nu(0)]}{0}$ er fotonpropagatoren i stedrum.
\\

Benytter vi os af udtrykket for Klein-Gordon-stedrumspropagatoren fra \cite[lign. 40]{problemSet3}
\begin{align}
    \mel**{0}{T \left[ \phi(x) \phi(0) \right]}{0} &= \int \frac{\dd^4 q}{(2\pi)^4} \frac{i}{q^2 - m^2} \pexp{-iqx}
        = \int \frac{\dd^4 q}{(2\pi)^4} \frac{i}{q^2} \pexp{-iqx} \: ,
\end{align}
hvor $m = 0$, da der er tale om en reel foton, så ser vi at \cref{eq:Opg4_Q13_PhotonPropagator} bliver
\begin{align}
\begin{split}
    G^{\mu\nu}(q) &= \int \dd^4 x \pexp{iqx} \mel**{0}{T\left[A^\mu(x)A^\nu(0)\right]}{0} \\
        &= - g^{\mu\nu} \int \dd^4 x \pexp{iqx} \mel**{0}{T \left[ \phi(x) \phi(0) \right]}{0} \\
        &= \frac{-ig^{\mu\nu}}{q^2} \: ,
\end{split}
\end{align}
da vi tager den inverse Fouriertransformation (\cite[lign. 72]{problemSet2}) af en fouriertransformation (\cite[lign. 71]{problemSet2}) af en funktion, hvilket blot giver funktionen selv (\cite[lign. 2.6]{fourierAnalysis}, $\mathcal{F}^{-1}[\mathcal{F}[f]] = f$). Dermed er \cref{eq:Opg4_Q13_PhotonPropagator} vist.

% Vi vil finde impulsrumsfotonpropagatoren, hvorfor det ''ekstra'' integral

% Ved at benytte Klein-Gordon-propagatoren som analogi, argumentér for at vi kan benytte den følgende form for fotonpropagatoren
% \begin{align}
%     G^{\mu\nu}(q) &= \int \dd^4 x \pexp{iqx} \mel**{0}{T\left[A^\mu(x)A^\nu(0)\right]}{0}
%         = \frac{-ig^{\mu\nu}}{q^2}
% \end{align}
% i udtrykket for $S_{fi}^{(2)}$.


%%%%%%%%%%%%%%%%%%%%%%%%%

\paragraph[14) Matrixelementet $\M_{fi}$ i $S_{fi}^{(2)}$]{\textbf{14)}}

Vi starter med at translatere $J_\mu^{\mathrm{e}^-}(x_1)$, $J_\nu^{\mu^-}(x_2)$ og matrixelementet med tidsombytningsoperatoren fra \cref{eq:Opg4_Q12_SMatrixSecondOrder}, hvilket gøres ved at lade $P$ være den totale firimpulsoperator, $x = x_1 - x_2$ og $Px = P^\mu x_\mu$ være sammentrækningen (eng: the contraction) af to firvektorer. Da får vi følgende
\begin{align}
    \begin{split}
        J_\mu^{\mathrm{e}^-}(x_1) &= \mel**{\mathrm{e}^-,\, \vv{p}_3,\, \lambda_3}{j_\mu^{\mathrm{e}^-}(x_1)}{\mathrm{e}^-,\, \vv{p}_1,\, \lambda_1} \\
            &= \mel**{\mathrm{e}^-,\, \vv{p}_3,\, \lambda_3}{\pexp{iPx_1} j_\mu^{\mathrm{e}^-}(0) \pexp{-iPx_1}}{\mathrm{e}^-,\, \vv{p}_1,\, \lambda_1} \\
            &= \mel**{\mathrm{e}^-,\, \vv{p}_3,\, \lambda_3}{\pexp{ip_3x_1} j_\mu^{\mathrm{e}^-}(0) \pexp{-ip_1x_1}}{\mathrm{e}^-,\, \vv{p}_1,\, \lambda_1} \\
            &= \pexp{ip_3x_1} \pexp{-ip_1x_1} \mel**{\mathrm{e}^-,\, \vv{p}_3,\, \lambda_3}{j_\mu^{\mathrm{e}^-}(0)}{\mathrm{e}^-,\, \vv{p}_1,\, \lambda_1} \\
            &= \pexp{i[p_3 - p_1]x_1} J_\mu^{\mathrm{e}^-}(0) \\
            &= \pexp{i[p_3 - p_1]x} \pexp{i[p_3 - p_1]x_2} J_\mu^{\mathrm{e}^-}(0) \: ,
    \end{split} \\
    \begin{split}
        J_\nu^{\mu^-}(x_2) &= \mel**{\mu^-,\, \vv{p}_4,\, \lambda_4}{j_\nu^{\mu^-}(x_2)}{\mu^-,\, \vv{p}_2,\, \lambda_2} \\
            &= \mel**{\mu^-,\, \vv{p}_4,\, \lambda_4}{\pexp{iPx_2} j_\nu^{\mu^-}(0) \pexp{-iPx_2}}{\mu^-,\, \vv{p}_2,\, \lambda_2} \\
            &= \mel**{\mu^-,\, \vv{p}_4,\, \lambda_4}{\pexp{ip_4x_2} j_\nu^{\mu^-}(0) \pexp{-ip_2x_2}}{\mu^-,\, \vv{p}_2,\, \lambda_2} \\
            &= \pexp{ip_4x_2} \pexp{-ip_2x_2} \mel**{\mu^-,\, \vv{p}_4,\, \lambda_4}{j_\nu^{\mu^-}(0)}{\mu^-,\, \vv{p}_2,\, \lambda_2} \\
            &= \pexp{i[p_4 - p_2]x_2} J_\nu^{\mu^-}(0) \: , \quad \text{og}
    \end{split}
\end{align}
\begin{align}
    \begin{split}
        &\mel**{0}{T[A^\mu(x_1) A^\nu(x_2)]}{0} \\
            &\quad = \mel**{0}{T[A^\mu(x_1) \pexp{iPx_2} A^\nu(0) \pexp{-iPx_2}]}{0} \\
            &\quad = \mel**{0}{T[\id A^\mu(x_1) \pexp{iPx_2} A^\nu(0) \pexp{-iPx_2}]}{0} \\
            &\quad = \mel**{0}{T[\pexp{iPx_2} \pexp{-iPx_2} A^\mu(x_1) \pexp{iPx_2} A^\nu(0) \pexp{-iPx_2}]}{0} \\
            &\quad = \mel**{0}{T[\pexp{iPx_2} A^\mu(x_1 - x_2) A^\nu(0) \pexp{-iPx_2}]}{0} \\
            &\quad = \mel**{0}{\pexp{-i\vv{p}\cdot\vv{x}_2} T[\pexp{iEt_2} A^\mu(x) A^\nu(0) \pexp{-iEt_2}] \pexp{i\vv{p}\cdot\vv{x}_2}}{0} \\
            &\quad = \mel**{0}{\pexp{0} T[\pexp{iEt_2} \pexp{-iEt_2} A^\mu(x) A^\nu(0)] \pexp{0}}{0} \\
            &\quad = \mel**{0}{T[A^\mu(x) A^\nu(0)]}{0} \: ,
    \end{split}
\end{align}
hvor eksponentialfunktionerne udenfor tidsombytningsoperatoren ''forsvinder'', da $\pexp{0} = 1$, og eksponentialfunktionerne indenfor tidsombytningsoperatoren kan rykkes hen til hinanden, da tidsombytningsoperatoren sørger for, at operatorerne står tidsordnet og eksponentialerne har samme tid $t_2$, så de vil stå sammen og dermed danner de en identitet.

Indsætter vi disse beregninger i udtrykket fra \textbf{12)}, \cref{eq:Opg4_Q12_SMatrixSecondOrder}, får vi
\begin{align} \label{eq:Opg4_A14_CalculationOfTheSMatrix}
\begin{split}
    S_{fi}^{(2)} &= (-i)^2 \int \dd^4 x_1\, \dd^4 x_2\, J_\mu^{\mathrm{e}^-}(x_1) J_\nu^{\mu^-}(x_2) \mel**{0}{T\left[A^\mu(x_1)A^\nu(x_2)\right]}{0} \\
        &= (-i)^2 \int \dd^4 x_2\, \dd^4 x\, \pexp{i[p_3 - p_1]x} \pexp{i[p_3 - p_1]x_2} J_\mu^{\mathrm{e}^-}(0) \\
            &\qquad\qquad\qquad\quad \times \pexp{i[p_4 - p_2]x_2} J_\nu^{\mu^-}(0) \mel**{0}{T[A^\mu(x) A^\nu(0)]}{0} \\
        &= (-i)^2 J_\mu^{\mathrm{e}^-}(0) J_\nu^{\mu^-}(0) \int \dd^4 x_2\, \pexp{i[p_3 + p_4 - p_1 - p_2]x_2} \\
            &\qquad\qquad\qquad\quad \times \int \dd^4 x\, \pexp{i[p_3 - p_1]x} \mel**{0}{T[A^\mu(x) A^\nu(0)]}{0} \\
        &= (-i)^2 J_\mu^{\mathrm{e}^-}(0) J_\nu^{\mu^-}(0) (2\pi)^4 \delta^4(p_3 + p_4 - p_1 - p_2) \frac{-ig^{\mu\nu}}{(p_3 - p_1)^2} \\
        &= \left( -i J_\mu^{\mathrm{e}^-}(0) \right) \frac{-ig^{\mu\nu}}{q^2} \left( -i J_\nu^{\mu^-}(0) \right) (2\pi)^4 \delta^4(p_f - p_i) \: ,
\end{split}
\end{align}
hvor vi har benyttet propagatoren fundet i \textbf{13)} (\cref{eq:Opg4_Q13_PhotonPropagator}), samt at vi betegner startimpulsen $p_i = p_1 + p_2$, slutimpulsen betegnes $p_f = p_3 + p_4$ og fotonens impuls betegnes $q = p_1 - p_3$. Sammenligner vi \cref{eq:Opg4_A14_CalculationOfTheSMatrix} med den generelle definition af S-matrixelementet til anden orden, $S_{fi}^{(2)} = - i \M_{fi} (2\pi)^4 \delta^4(p_f - p_i)$, så ses det, at
\begin{align}
    - i \M_{fi} &= \left( -i J_\mu^{\mathrm{e}^-}(0) \right) \frac{-ig^{\mu\nu}}{q^2} \left( -i J_\nu^{\mu^-}(0) \right) \: ,
\end{align}
hvilket skulle vises.


%%%%%%%%%%%%%%%%%%%%%%%%%

\paragraph[15) Amplituden af $\M_{fi}$]{\textbf{15)}}

Først bemærkes det, at der er bevarelse af firimpulsen, altså er $p_i = p_f$, hvor $i$ og $f$ betegner hhv. start- og sluttilstanden. Af dette følger, at
\begin{align}
    p_1 + p_2 &= p_3 + p_4 \label{eq:Opg4_A15_p1+p2=p3+p4} \\
    \Rightarrow p_1 - p_4 &= p_3 - p_2 \: . \label{eq:Opg4_A15_p1-p4=p3-p2}
\end{align}
Ydermere er kvadratet på firimpulsen
\begin{align} \label{eq:Opg4_A15_SquareOfFourMomentum}
    p^\mu p_\mu &= E^2 - \vv{p}^2 = (\vv{p}^2 + m^2) - \vv{p}^2 = m^2 \: .
\end{align}

Betragter vi først \cref{eq:Opg4_A15_p1+p2=p3+p4} og vælger at kvadrere begge sider, da fås
\begin{align} \label{eq:Opg4_A15_p1p2=p3p4}
\begin{split}
    p_1 + p_2 &= p_3 + p_4 \\
    \Rightarrow p_1^2 + p_2^2 + 2 p_1 p_2 &= p_3^2 + p_4^2 + 2 p_3 p_4 \\
    \Rightarrow m_e^2 + m_\mu^2 + 2 p_1 p_2 &= m_e^2 + m_\mu^2 + 2 p_3 p_4 \\
    \Rightarrow p_1 p_2 &= p_3 p_4 \: ,
\end{split}
\end{align}
da $p_1$ og $p_3$ er hhv. den indgående og udgående impuls for elektronen, mens $p_2$ og $p_4$ hhv. er den indgående og udgående impuls for myonen, altså $p_1^2 = p_3^2 = m_e^2$ og $p_2^2 = p_4^2 = m_\mu^2$.

Betragter vi nu \cref{eq:Opg4_A15_p1-p4=p3-p2} og kvadrerer igen fås
\begin{align} \label{eq:Opg4_A15_p1p4=p2p3}
\begin{split}
    p_1 - p_4 &= - p_2 + p_3 \\
    \Rightarrow p_1^2 + p_4^2 - 2 p_1 p_4 &= p_2^2 + p_3^2 - 2 p_2 p_3 \\
    \Rightarrow m_e^2 + m_\mu^2 - 2 p_1 p_4 &= m_\mu^2 + m_e^2 - 2 p_2 p_3 \\
    \Rightarrow p_1 p_4 &= p_2 p_3 \: .
\end{split}
\end{align}

Indsætter vi \cref{eq:Opg4_A15_p1p2=p3p4,eq:Opg4_A15_p1p4=p2p3} i den kvadrerede amplitude, hvor der er summeret over sluttilistande og taget gennemsnit af stattilstandene, \cref{eq:Opg4_Q15_AmplitudeOfMByAllMomenta}, fås
\begin{align} \label{eq:Opg4_A15_AmplitudeOfMByMomentaSquared}
\begin{split}
    \inv{4} \sum_{\mathrm{spins}} \abs{\M_{fi}}^2 &= \frac{8e^4}{q^4} \big( \left[ p_1 p_2 \right] \left[ p_3 p_4 \right] + \left[ p_1 p_4 \right] \left[ p_2 p_3 \right] \big) \\
        &= \frac{8e^4}{q^4} \big( \left[ p_1 p_2 \right] \left[ p_1 p_2 \right] + \left[ p_1 p_4 \right] \left[ p_1 p_4 \right] \big) \\
        &= \frac{8e^4}{q^4} \left( \left[ p_1 p_2 \right]^2 + \left[ p_1 p_4 \right]^2 \right) \: .
\end{split}
\end{align}
\\

Når vi er i massemidtpunktssystemet, da er $p_i = p_f = 0$, altså $\vv{p}_2 = -\vv{p}_1$ og $\vv{p}_4 = -\vv{p}_3$, og da vi betragter det hyperrelativistiske tilfælde $\vv{p} \gg m_n$ (hvor $n$ betegner elektronen eller myonen), så er $E \simeq \abs{\vv{p}}$, og faktisk er $E = E_1 = E_2 = E_3 = E_4$. Betragter vi nu de to kvadrerede led i \cref{eq:Opg4_A15_AmplitudeOfMByMomentaSquared} ses, at
\begin{align}
    \begin{split} \label{eq:Opg4_A15_FirstSquaredTerm}
        (p_1 p_2)^2 &= (E_1 E_2 - \vv{p}_1 \cdot \vv{p}_2)^2 \\
            &= \Big( E E - \vv{p}_1 \cdot [-\vv{p}_1] \Big)^2 \\
            &= \big( E^2 + \vv{p}_1^2 \big)^2 \\
            &\simeq \big( E^2 + E^2 \big)^2 \\
            &= 4 E^4 \: ,
    \end{split} \\
    \begin{split} \label{eq:Opg4_A15_SecondSquaredTerm}
        (p_1 p_4)^2 &= (E_1 E_4 - \vv{p}_1 \cdot \vv{p}_4)^2 \\
            &= \Big( E E - \vv{p}_1 \cdot [-\vv{p}_3] \Big)^2 \\
            &= \big( E^2 + \vv{p}_1 \cdot \vv{p}_3 \big)^2 \\
            &= \Big[ E^2 + \abs{\vv{p}_1} \abs{\vv{p}_3} \cos(\theta) \Big]^2 \\
            &\simeq \Big[ E^2 \big\{ 1 + \cos(\theta) \big\} \Big]^2 \\
            &= \left[ 2 E^2 \cos^2\pfrac{\theta}{2} \right]^2 \\
            &= 4 E^4 \cos^4\pfrac{\theta}{2} \: ,
    \end{split}
\end{align}
hvor den trigonometriske identitet $2\cos^2(A) = 1 + \cos(2A)$ er blevet benyttet.

Sidst vil vi kigge på ledet $q^4$, hvor $q = p_1 - p_3$ er impulsoverførslen, hvorfor
\begin{align}
\begin{split} \label{eq:Opg4_A15_qToFourthPower}
    q^4 &= (p_1 - p_3)^4 \\
        &= (p_1^2 + p_3^2 - 2 p_1 p_3)^2 \\
        &= \Big(m_e^2 + m_e^2 - 2 \big[ E_1 E_3 - \vv{p}_1 \cdot \vv{p}_3 \big] \Big)^2 \\
        &= \Big(m_e^2 + m_e^2 - 2 \big[ E E - \abs{\vv{p}_1} \abs{\vv{p}_3} \cos(\theta) \big] \Big)^2 \\
        &\simeq \Big(m_e^2 + m_e^2 - 2 E^2\big[ 1 - \cos(\theta) \big] \Big)^2 \\
        &\approx \Big(- 2 E^2\big[ 1 - \cos(\theta) \big] \Big)^2 \\
        &= \left(-2 E^2 \left[2 \sin^2\pfrac{\theta}{2} \right] \right)^2 \\
        &= 16 E^4 \sin^4\pfrac{\theta}{2} \: ,
\end{split}
\end{align}
hvor den trigonometriske identitet $2\sin^2(A) = 1 - \cos(2A)$ er blevet benyttet.

Indsætter vi nu \cref{eq:Opg4_A15_FirstSquaredTerm,eq:Opg4_A15_SecondSquaredTerm,eq:Opg4_A15_qToFourthPower} i \cref{eq:Opg4_A15_AmplitudeOfMByMomentaSquared} fås
\begin{align}
\begin{split}
    \inv{4} \sum_{\mathrm{spins}} \abs{\M_{fi}}^2 &= \frac{8e^4}{q^4} \left( \left[ p_1 p_2 \right]^2 + \left[ p_1 p_4 \right]^2 \right) \\
        &= \frac{8e^4}{16 E^4 \sin^4\pfrac{\theta}{2}} \left[ 4 E^4 + 4 E^4 \cos^4\pfrac{\theta}{2} \right] \\
        &= \frac{8e^4}{4 \sin^4\pfrac{\theta}{2}} \left[ 1 + \cos^4\pfrac{\theta}{2} \right] \\
        &= \frac{2e^4}{\sin^4\pfrac{\theta}{2}} \left[ 1 + \cos^4\pfrac{\theta}{2} \right] \: ,
\end{split}
\end{align}
hvilket er \cref{eq:Opg4_Q15_AmplitudeOfMByAngleInCMFrame}, som skulle vises.


%%%%%%%%%%%%%%%%%%%%%%%%%

\paragraph[16) Tværsnit for elektron-myon-spredningen i CM-systemet og \\ Rutherfordspredningstværsnit]{\textbf{16)}}

Fra \cite[lign. 6.129]{AitchisonHey} vides det, at det generelle udtryk for tværsnittet i massemidtpunktssystemet af en spredning mellem to partikler er
\begin{align}
    \dif{\sigma}{\Omega}\biggr\rvert_{CM} &= \inv{(8\pi E_i)^2} \abs{\M_{fi}}^2 \: ,
\end{align}
hvor $E_i$ er den indkomne energi. Denne formel kan benyttes, da vi er i højenergigrænsen, altså $p \gg m$, hvorfor masserne er negligerbare.

For elektron-myon-spredningen bliver tværsnittet
\begin{align}
\begin{split}
    \dif{\sigma}{\Omega}\biggr\rvert_{CM} &= \inv{(8\pi E_i)^2} \abs{\M_{fi}}^2 \\
        &= \inv{\big( 8 \pi [ E_1 + E_2 ] \big)^2} \abs{\M_{fi}}^2 \\
        &= \inv{\big( 8 \pi [ E + E ] \big)^2} \abs{\M_{fi}}^2 \\
        &= \inv{\big( 16 \pi E \big)^2} \abs{\M_{fi}}^2 \\
        &= \inv{\big( 8 \pi E \big)^2} \inv{4} \abs{\M_{fi}}^2 \\
        &= \inv{\big( 8 \pi E \big)^2} \frac{2e^4}{\sin^4\pfrac{\theta}{2}} \left[ 1 + \cos^4\pfrac{\theta}{2} \right] \\
        &= \frac{2e^4}{64 \pi^2 E^2} \frac{1 + \cos^4\pfrac{\theta}{2}}{\sin^4\pfrac{\theta}{2}} \\
        &= \frac{\alpha^2}{2 E_{CM}^2} \frac{1 + \cos^4\pfrac{\theta}{2}}{\sin^4\pfrac{\theta}{2}} \: ,
\end{split}
\end{align}
hvor $E = E_{CM}$ er energien i massemidtpunktssytemet, $\alpha = e^2/(4\pi)$ er finstrukturkonstanten (i naturlige enheder), og resultatet fra \textbf{15)} (\cref{eq:Opg4_Q15_AmplitudeOfMByAngleInCMFrame}) er blevet benyttet.
\\

Vi kigger nu på funktionsopførslen af \cref{eq:Opg4_Q15_AmplitudeOfMByAngleInCMFrame}, når $\theta \rightarrow 0$, hvilket gør at
\begin{align}
    \dif{\sigma}{\Omega}\biggr\rvert_{CM} (\theta \rightarrow 0) &= \frac{\alpha^2}{2 E_{CM}^2} \frac{1 + \cos^4\pfrac{\theta}{2}}{\sin^4\pfrac{\theta}{2}}
        \simeq \frac{\alpha^2}{2 E_{CM}^2} \frac{1 + 1^4}{\pfrac{\theta}{2}^2}
        = \frac{\alpha^2}{E_{CM}^2 \pfrac{\theta}{2}^2}
        \propto \theta^{-4} \: .
\end{align}
Nu betragtes Rutherfordtværsnittet (\cite[lign. 5.16]{povh_particles_2015})
\begin{align}
    \dif{\sigma}{\Omega} \biggr\rvert_{CM,\mathrm{R}} &= \frac{\alpha^2}{16 E_{CM}^2 \sin^4\pfrac{\theta}{2}} \: ,
\end{align}
og tager vi her også grænsen $\theta \rightarrow 0$, får vi
\begin{align}
    \dif{\sigma}{\Omega} \biggr\rvert_{CM,\mathrm{R}} (\theta \rightarrow 0) &= \frac{\alpha^2}{16 E_{CM}^2 \sin^4\pfrac{\theta}{2}}
        \simeq \frac{\alpha^2}{16 E_{CM}^2 \pfrac{\theta}{2}^4}
        \propto \theta^{-4} \: .
\end{align}
Altså er funktionsopførslen for begge tværsnit, at de udvikles som funktion af $(\dd \sigma / \dd \Omega)_{CM} \propto \theta^{-4}$ i grænsen hvor $\theta \rightarrow 0$.


%%%%%%%%%%%%%%%%%%%%%%%%%

\paragraph[17) Feynmanregler for QED]{\textbf{17)}}

Først kigger vi på indgående og udgående fermioner og antifermioner. Til dette betragter vi matrixelementerne fra \cref{eq:Opg4_Q5_TransitionCurrents} \textbf{5)}, og sammenligner vi specifikt matrixelementerne med $j_µ^{(1)}$ og $j_µ^{(3)}$ (\cref{eq:Opg4_Q5_TransitionCurrent1,eq:Opg4_Q5_TransitionCurrent3}) ses det, at bidraget fra en indgående elektron med impuls $\vv{p}$ og helicitet $\lambda$ er $u(\vv{p},\lambda)$, hvormed bidraget fra udgående elektron med impuls $\vv{p}$ og helicitet $\lambda$ er $\overline{u}(\vv{p},\lambda)$. Ligeså kan det ved en sammenligning af matrixelementerne med $j_µ^{(2)}$ og $j_µ^{(3)}$ (\cref{eq:Opg4_Q5_TransitionCurrent2,eq:Opg4_Q5_TransitionCurrent3}) ses, at en indgående positron med impuls $\vv{p}$ og helicitet $\lambda$ er $\overline{v}(\vv{p},\lambda)$ og bidraget fra udgående positron med impuls $\vv{p}$ og helicitet $\lambda$ er $v(\vv{p},\lambda)$.
\\

Kigger vi nu på fotonens bidrag, så ses det ud fra kvantiseringen af feltet for en foton, \cite[lign. 8]{Q&A10}
\begin{align}
    A^\mu(x) &= \sum_\sigma \int \frac{\dd^3 \vv{p}}{(2\pi)^3} \invsqrt{2 \omega_{\vv{p}}} \left[ \epsilon^\mu(\sigma) a_{\vv{p},\sigma} \pexp{-ipx} + \epsilon^\mu(\sigma)^* a_{\vv{p},\sigma}\dagger \pexp{ipx} \right] \: ,
\end{align}
ses, at en indkommende foton giver anledning til et bidrag af $\epsilon^\mu(\sigma)$, mens en udgående foton giver anledning til et bidrag af $\epsilon^\mu(\sigma)^*$ \cite[s. 123]{peskinSchroeder}.
\\

Som det næste har vi propagatorerne, hvor vi i Problemset 3, opg. 4.7 \cite[lign. 34]{problemSet3} har udregnet femrionpropagatoren til
\begin{align}
    G_{F_{\alpha\beta}}(p_\mu) &= \frac{i (\slashed{p} + m)_{\alpha\beta}}{p^2 - m^2} \: ,
\end{align}
hvor $p$ er firimpulsen af fermionen, og i \textbf{13)} (\cref{eq:Opg4_Q13_PhotonPropagator}) fandt vi fotonpropagatoren til
\begin{align}
    G_\gamma^{\mu\nu}(q_\mu) &= \frac{-ig^{\mu\nu}}{q^2} \: ,
\end{align}
hvor $q$ er firimpulsen, som overføres gennem denne propagator og $g^{\mu\nu}$ er den metriske tensor i Minkowskimetrikken.
\\

Sidst kigger vi på knudepunkterne. Ud fra \cref{eq:Opg4_Q14_AmplitudeMfi} i \textbf{14)} samt \cref{eq:Opg4_A12_ExplicitElectronAndMuonCurrent} kan det ses, at de resterende bidrag, når vi har taget højde for indgående og udgående partikler samt propagatorer, må være fra knudepunkterne, hvorfor disse må være $i e \gamma_\mu$ for hvert knudepunkt. Dog har vi kun kigget på fermioner i vores beregninger, men havde vi kigget på antifermioner ville det ses, at de ville bidrage med en knudepunktsfaktor på $-ie\gamma_\mu$ \cite[s. 106]{seiden}.
\\

For at opsummere, så er Feynmanreglerne for kvanteelektrodynamik
\begin{itemize}
    \item En indgående fermioner bidrager med $u(\vv{p},\lambda)$.
    \item En udgående fermion bidrager med $\overline{u}(\vv{p},\lambda)$.
    \item En indgående antifermion bidrager med $\overline{v}(\vv{p},\lambda)$.
    \item En udgående antifermion bidrager med $v(\vv{p},\lambda)$.
    \item En indgående foton bidrager med $\epsilon^\mu(\sigma)$.
    \item En udgående foton bidrager med $\epsilon^\mu(\sigma)^*$.
    \item En fermion propagator bidrager med $i(\slashed{p}+m)_{\alpha\beta}/(p^2 - m^2)$. \footnote{
        Dette kan også skrives som
        \begin{align}
            G_\mathrm{fermion} &= \frac{i}{\slashed{p}-m} \: ,
        \end{align}
        da $\slashed{p}^2 = p^2$, hvis man tager sig i agt for notationen med her at dele med en matrix.
    }
    \item En foton propagator bidrager med $-ig^{\mu\nu}/q^2$.
    \item Et knudepunkt for en fermion bidrager med $ie\gamma_\mu$, mens det for en antifermion bidrager med $-ie\gamma_\mu$.
\end{itemize}
Herudover skal der selvfølgelig være impulsbevarelse i hvert knudepunkt.


%%%%%%%%%%%%%%%%%%%%%%%%%%%%%%%%%%%%%%%%%%%%%%%%%%%%%%%%%%%%%%%%%%%%%%%%%%%%%%%%%%%%%

\end{document}