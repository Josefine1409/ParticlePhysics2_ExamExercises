\documentclass[../main.tex]{subfiles}

\begin{document}

%%%%%%%%%%%%%%%%%%%%%%%%%%%%%%%%%%%%%%%%%%%%%%%%%%%%%%%%%%%%%%%%%%%%%%%%%%%%%%%%%%%%%

\section{''Matrix''-feltet}

Betragt en mængde af felter $\phi_{ij}$, hvor $i,j = 1,\ldots,N$, og antag at $\phi_{ij} = \phi_{ji}$. Antag yderligere at felterne er reelle, altså at $\phi_{ij} \in \R,\,\: \forall i,j=1,\ldots,N$. Lagrangedensiteten for dette system er
\begin{align} \label{eq:Opg2_Lagrangian}
    \L &= \inv{2} \partial_\mu \phi_{ij} \partial^\mu \phi_{ji} - \inv{2} m^2 \phi_{ij} \phi_{ji} \: ,
\end{align}
hvor vi har implicitte summer over felterne således at
\begin{align} \label{eq:Opg2_ImplicitSums}
    \phi_{ij} \phi_{jk} &= \sum_{j=1}^N \phi_{ij}\phi_{jk} \quad \text{og} \quad
    \phi_{ij} \phi_{ji} = \sum_{i=1}^N \sum_{j=1}^N \phi_{ij}\phi_{ji} \: .
\end{align}

Bemærk hvordan $\phi_{ij} \phi_{ji}$ svarer til at betragte feltet som værende en matrix, multiplicere matricen med sig selv og sidst tage trace af den resulterende matrix.


%%%%%%%%%%%%%%%%%%%%%%%%%%%%%%%%%%%%%%%%%%%%%%%%%%%%%%%%%%%%%%%%%%%%%%%%%%%%%%%%

\paragraph*{\textbf{1)}}

Hvor mange frihedsgrader har feltet $\phi_{ij}$?


%%%%%%%%%%%%%%%%%%%%%%%%%

\paragraph*{\textbf{2)}}

Vis at bevægelsesligningerne er
\begin{align} \label{eq:Opg2_Q2_EquationOfMotion}
    0 &= \left(\partial_\mu \partial^\mu + m^2\right)\phi_{ij} \: , \quad \forall i,j=1,\ldots,N \: .
\end{align}


%%%%%%%%%%%%%%%%%%%%%%%%%

\paragraph*{\textbf{3)}}

Vis at løsningerne er planbølger på formen $\phi_{ij} = M_{ij}\exp(\pm ipx)$, hvor $px = p^\mu x_\mu = p_\mu x^\mu$ med $p^\mu = (E,\, \vv{p})$ og $M_{ij}$ er uafhængig af $x^\mu$. Find udtrykket for E.


%%%%%%%%%%%%%%%%%%%%%%%%%

\paragraph*{\textbf{4)}}

Definer det konjugerede felt, $\pi_{ij}$, som den naturlige generalisering af enkeltkomponent skalarfelter
\begin{align} \label{eq:Opg2_Q4_ConjugatedMomentaField}
    \pi_{ij} &= \pdif{\L}{(\partial_0 \phi_{ij})} \: .
\end{align}
Vis at $\pi_{ij} = \partial^0 \phi_{ij}$.


%%%%%%%%%%%%%%%%%%%%%%%%%

\paragraph*{\textbf{5)}}

Vis at Hamiltonfunktionen er
\begin{align} \label{eq:Opg2_Q5_Hamiltonian}
    H &= \int \dd^3\vv{x} \left[ \inv{2} \pi_{ij} \pi_{ji} + \inv{2} \Grad{\phi_{ij}} \cdot \Grad{\phi_{ji}} + \inv{2} m^2 \phi_{ij} \phi_{ji} \right] \: .
\end{align}


%%%%%%%%%%%%%%%%%%%%%%%%%

\paragraph*{\textbf{6)}}

Argumentér for at en fornuftig feltudvidelse (eng: field expansion) for det kvantemekaniske ''matrix''-felt er
\begin{align} \label{eq:Opg2_Q6_ResultingField}
    \phi_{ij} &= \int \frac{\dd^3\vv{p}}{(2\pi)^3} \invsqrt{2\omega_{\vv{p}}} \left(a_{\vv{p}ij} \bexp{-ipx} + a_{\vv{p}ij}\dagger \bexp{ipx} \right) \: ,
\end{align}
hvor $\omega_{\vv{p}} = \sqrt{\vv{p}^2 + m^2}$.


%%%%%%%%%%%%%%%%%%%%%%%%%

\paragraph*{\textbf{7)}}

Vi introducerer nu et indre produkt, kaldet Klein-Gordon-indreproduktet, mellem to bølgefunktioner på en lidt besynderlig måde. Lad $f$ og $g$ være funktioner af tid og rum og definer
\begin{align}
\begin{split} \label{eq:Opg2_Q7_KGInnerProduct}
    (f,g) &= i \int \dd^3\vv{x} f^*(\vv{x},\, t) \overset{\leftrightarrow}{\partial_0} g(\vv{x},\, t) \\
        &= i \int \dd^3\vv{x} \left[ f^*(\vv{x},\, t) \pdif{g(\vv{x},\, t)}{t} - \pdif{f^*(\vv{x},\, t)}{t} g(\vv{x},\, t) \right] \: ,
\end{split}
\end{align}
hvor $\overset{\leftrightarrow}{\partial_0}$ short-hand notation for operatoren, som tidsdifferentierer til først til højre og så til venstre med en minus imellem. Bemærk at dette indre produkt ikke er positivt bestemt (eng: positive definite) og derfor ikke er et matematisk indre produkt, hvis man skal være stringent. Men det er en dejlig og nyttig konstruktion, hvilket vi nu skal vise.

Betragt en planbølge på formen $u_{\vv{p}}(x) = \pexp{-ipx}/\sqrt{2\omega_{\vv{p}}}$, hvor $x$ og $p$ er firvektorer som sædvanligt. Vis de følgende relationer
\begin{subequations} \label{eq:Opg2_Q7_KGInnerProductsWithU}
\begin{align}
    (u_{\vv{p}}, u_{\vv{p}'}) &= (2\pi)^3 \delta^3(\vv{p} - \vv{p}') \: , \\
    (u_{\vv{p}}^*, u_{\vv{p}'}^*) &= - (2\pi)^3 \delta^3(\vv{p} - \vv{p}') \: , \\
    (u_{\vv{p}}^*, u_{\vv{p}'}) &= 0 \: , \quad \text{og} \\
    (u_{\vv{p}}, u_{\vv{p}'}^*) &= 0 \: .
\end{align}
\end{subequations}


%%%%%%%%%%%%%%%%%%%%%%%%%

\paragraph*{\textbf{8)}}

Benyt nu det indre produkt med feltet fra \cref{eq:Opg2_Q6_ResultingField} og vis at
\begin{subequations} \label{eq:Opg2_Q8_StepOperatorsUsingKGInnerProduct}
\begin{align}
    a_{\vv{p}ij} &= (u_{\vv{p}},\phi_{ij}) \: , \quad \text{og} \\
    a_{\vv{p}ij}\dagger &= -(u_{\vv{p}}^*,\phi_{ij}) \: .
\end{align}
\end{subequations}


%%%%%%%%%%%%%%%%%%%%%%%%%

\paragraph*{\textbf{9)}}

Kommutatorrelationerne for matrix-felterne er givet ved
\begin{align} \label{eq:Opg2_Q9_CommutatorOfFieldAndItsTimeDerivativePi}
    \commutator{\phi_{ij}(\vv{x},\, t)}{\pi_{kl}(\vv{x}',\, t)} &= i\delta^3(\vv{x} - \vv{x}')\delta_{ik}\delta{jl} \: ,
\end{align}
mens alle andre kombinationer af $\phi_{ij}$ og $\pi_{kl}$ forsvinder. Vis at dette betyder, at kreations- og annihilationsoperatorerne, $a_{\vv{p}ij}\dagger$ og $a_{\vv{p}ij}$, har bosniske kommutationsrelationer.


%%%%%%%%%%%%%%%%%%%%%%%%%

\paragraph*{\textbf{10)}}

Vis at Hamiltonfunktionen for det kvantiserede felt kan skrives som
\begin{align} \label{eq:Opg2_Q10_HamiltonianWithCreationAndAnnihilationOperators}
    H &= \sum_{i,j=1}^N \int \frac{\dd^3\vv{p}}{(2\pi)^3} \left[ \omega_{\vv{p}} a_{\vv{p}ij}\dagger a_{\vv{p}ij} \right] \: .
\end{align}


%%%%%%%%%%%%%%%%%%%%%%%%%

\paragraph*{\textbf{11)}}

Vis at feltoperatoren overholder Heisenbergs bevægelsesligning
\begin{align} \label{eq:Opg2_Q11_HeisenbergEquationOfMotion}
    i\pdif{\phi_{ij}}{t} &= \commutator{\phi_{ij}}{H} \: .
\end{align}

Argumentér for at dette betyder, at feltoperatorerne fra \cref{eq:Opg2_Q6_ResultingField} er de rigtige feltoperatorer at benytte i vekselvirkningsbilledet (eng: interaction picture).


%%%%%%%%%%%%%%%%%%%%%%%%%

\paragraph*{\textbf{12)}}

Vi vil nu kigge på symmetrier for vores matrix-felter. Betragt gruppen af ortogonale $N \times N$ matricer, hvilken kaldes $O(N)$ og defineres som mængden
\begin{align}
    O(N) &= \left\{ R \:\vert\: R^\intercal R = \id \right\} \: ,
\end{align}
hvor $R$ er en reel $N \times N$ matrix og $R^\intercal$ betegner den transponerede matrix af $R$.

Vælg $R \in O(N)$ og lad komponenterne betegnes $R_{ij}$. Vis at Lagrangedensiteten i \cref{eq:Opg2_Lagrangian} er invariant under transformationen
\begin{align} \label{eq:Opg2_Q12_TransformationOfField}
    \phi_{ij}' &= R_{ik}R_{jl}\phi_{kl} \: .
\end{align}
Dette kaldes $O(N)^2$-symmetri.


%%%%%%%%%%%%%%%%%%%%%%%%%

\paragraph*{\textbf{13)}}

Find et vekselvirkningsled (eng: interaction term) som kan lægges til Lagrangedensiteten således, at denne stadig har $O(N)^2$-symmetri. Find hernæst et vekselvirkningsled, som eksplicit \emph{bryder} $O(N)^2$-symmetrien for Lagrangedensiteten.


%%%%%%%%%%%%%%%%%%%%%%%%%

\paragraph*{\textbf{14)}}

Betragt et element $R$ i $O(N)$ meget tæt på identitetsmatricen,  således at man kan skrive elementet $R = \id + A$, hvor $A$ er en $N \times N$ matrix med elementer $\abs{A_{ij}} \ll 1$. Vis at $R \in O(N)$ medfører at $A^\intercal = -A$, altså at $A$ er antisymmetrisk. Mængden af sådanne matricer $A$ kaldes \emph{generatoren} af $O(N)$. Hvor mange lineært uafhængige matrice findes med denne egenskab for dimensionen $N$? Dette kaldes dimensionen af $O(N)$.


%%%%%%%%%%%%%%%%%%%%%%%%%

\paragraph*{\textbf{15)}}

Find den bevarede Noetherstrøm (eng: Noether current) som associeres med $O(N)^2$-symmetrien.


%%%%%%%%%%%%%%%%%%%%%%%%%

\paragraph*{\textbf{16)}}

Hvad ville ændre sig, hvis feltet $\phi_{ij}$ var komplekst i stedet for reelt? Beskriv blot de overordnede forskellige; du skal \emph{ikke} beregne hele opgaven igen med et komplekst felt.


%%%%%%%%%%%%%%%%%%%%%%%%%%%%%%%%%%%%%%%%%%%%%%%%%%%%%%%%%%%%%%%%%%%%%%%%%%%%%%%%

\subsection{Besvarelse}

%%%%%%%%%%%%%%%%%%%%%%%%%

\paragraph[1) Frihedsgrader for feltet $\phi_{ij}$]{\textbf{1)}}

Vi genkender at $\phi_{ij}$ er en rang-2-tensor, hvilke beskrives ved matricer, som set i opgavebeskrivelsen. Her er der tale om, at $i,j \in \{1,\ldots,N\}$, altså vil det for en generel reel matrix af denne størelse gøre sig gældende, at der er $N^2$ frihedsgrader. Dog vides det, at feltet er symmetrisk, $\phi_{ij} = \phi_{ji}$, hvorved alle elementerne under diagonalen bestemmes af elementerne over den. Elementerne på diagonalen skal også medtælles som frihedsgrader, da disse ikke bestemmes af andre elementer.

I den nedre trekant af matricen er der $N^2/2 - N/2$ elementer, hvor første del giver halvdelen af alle elementer i matricen, mens anden del fratrækker elementerne på diagonalen. Dermed må antallet af frihedsgrader for feltet $\phi_{ij}$ være de resterende elementer, altså
\begin{align}
    N^2 - \frac{N^2 - N}{2} &= \frac{2N^2 - N^2 + N}{2}
        = \frac{N^2 + N}{2}
        = \frac{N(N+1)}{2} \: .
\end{align}


%%%%%%%%%%%%%%%%%%%%%%%%%

\paragraph[2) Bevægelsesligningerne for $\psi_{ij}$]{\textbf{2)}}

Vi betragter først Lagrangedensiteten \cref{eq:Opg2_Lagrangian} og konstaterer, at idet $\phi_{ij} = \phi_{ji}$ samt at vi kan benytte metrikken til at ændre kovariante firvektorer til kontravariante og omvendt, så kan Lagrangedensiteten skrives som
\begin{align}
    \L &= \inv{2} \partial_\mu \phi_{ij} \partial^\mu \phi_{ji} - \inv{2} m^2 \phi_{ij} \phi_{ji}
        = \inv{2} g^{\mu\nu} \partial_\mu \phi_{ij} \partial_\nu \phi_{ij} - \inv{2} m^2 \phi_{ij} \phi_{ij} \: .
\end{align}

For at finde bevægelsesligningerne benytter vi Euler-Lagrangeligningen
\begin{align} \label{eq:Opg2_A2_EulerLagrangeEquation}
    0 &= \partial_\mu \pdif{\L}{(\partial_\mu \phi_{ij})} - \pdif{\L}{\phi_{ij}}
\end{align}
for $i,j = 1,\ldots,N$, hvormed vi altså får $i \cdot j$ bevægelsesligninger.

Først beregner vi den afledede mht. feltet selv
\begin{align} \label{eq:Opg2_A2_EulerLagrangeEquation_PartPotential}
\begin{split}
    \pdif{\L}{\phi_{ij}} &= \pdif{}{\phi_{ij}} \left( \inv{2} g^{\rho\nu} \partial_\rho \phi_{kl} \partial_\nu \phi_{kl} - \inv{2} m^2 \phi_{kl} \phi_{kl} \right) \\
        &= \pdif{}{\phi_{ij}} \left( - \inv{2} m^2 \phi_{kl}^2 \right) \\
        &= - \inv{2} m^2 2 \delta_k^i \delta_l^j \phi_{kl} \\
        &= - \inv{2} m^2 2 \phi_{ij} \\
        &= - m^2 \phi_{ij} \: ,
\end{split}
\end{align}
og hernæst den afledede af Lagrangedensiteten mht. det afledede feltet
\begin{align} \label{eq:Opg2_A2_EulerLagrangeEquation_PartKinetic}
\begin{split}
    \pdif{\L}{(\partial_\mu \phi_{ij})} &= \pdif{}{(\partial_\mu \phi_{ij})} \left( \inv{2} g^{\rho\nu} \partial_\rho \phi_{kl} \partial_\nu \phi_{kl} - \inv{2} m^2 \phi_{kl} \phi_{kl} \right) \\
        &= \pdif{}{(\partial_\mu \phi_{ij})} \left( \inv{2} g^{\rho\nu} \partial_\rho \phi_{kl} \partial_\nu \phi_{kl} \right) \\
        &= \inv{2} g^{\rho\nu} \left( \pdif{}{(\partial_\mu \phi_{ij})} \big( \partial_\rho \phi_{kl} \big) \partial_\nu \phi_{kl} + \partial_\rho \phi_{kl} \pdif{}{(\partial_\mu \phi_{ij})} \big( \partial_\nu \phi_{kl} \big) \right) \\
        &= \inv{2} g^{\rho\nu} \left( \delta_\nu^\mu \delta_k^i \delta_l^j \partial_\nu \phi_{kl} + \delta_\rho^\mu \delta_k^i \delta_l^j \partial_\rho \phi_{kl} \right) \\
        &= \inv{2} \left( \delta_\nu^\mu \delta_k^i \delta_l^j \partial^\nu \phi_{kl} + \delta_\rho^\mu \delta_k^i \delta_l^j \partial^\rho \phi_{kl} \right) \\
        &= \inv{2} \left( \partial^\mu \phi_{ij} + \partial^\mu \phi_{ij} \right) \\
        % &= \inv{2} \left( \partial^\rho \phi_{ij} + \partial^\nu \phi_{ij} \right) \\
        % &= \inv{2} \left( \partial^{\cancelto{\nu}{\rho}} \phi_{ij} + \partial^\nu \phi_{ij} \right) \\
        &= \partial^\nu \phi_{ij} \: .
\end{split}
\end{align}

Indsætter vi nu de beregnede afledede, \cref{eq:Opg2_A2_EulerLagrangeEquation_PartPotential,eq:Opg2_A2_EulerLagrangeEquation_PartKinetic}, i Euler-Lagrangeligningen, \cref{eq:Opg2_A2_EulerLagrangeEquation}, fås bevægelsesligningen for feltet $\phi_{ij}$ til
\begin{align}
\begin{split}
    0 &= \partial_\mu \big( \partial^\mu \phi_{ij} \big) - \big( -m^2 \phi_{ij} \big) \\
        &= \big( \partial_\mu \partial^\mu + m^2 \big) \phi_{ij} \: .
\end{split}
\end{align}

Altså er bevægelsesligningerne
\begin{align}
    0 &= \left(\partial_\mu \partial^\mu + m^2\right)\phi_{ij} \: , \quad \forall i,j=1,\ldots,N \: .
\end{align}
som givet af \cref{eq:Opg2_Q2_EquationOfMotion}.


%%%%%%%%%%%%%%%%%%%%%%%%%

\paragraph[3) Løsningerne for felterne er planbølger]{\textbf{3)}}

For at vise, at løsningerne til \cref{eq:Opg2_Q2_EquationOfMotion} er planbølger på formen $\phi_{ij} = M_{ij}\exp(\pm ipx)$, hvor $px = p^\mu x_\mu = p_\mu x^\mu$ med $p^\mu = (p^0,\, p^a) = (E,\, \vv{p})$ og $M_{ij}$ er uafhængig af $x^\mu$, indsættes disse planbølger i bevægelsesligningen \cref{eq:Opg2_Q2_EquationOfMotion}
\begin{align}
\begin{split}
    \left(\partial_\mu \partial^\mu + m^2\right)\phi_{ij} &= \left(\partial_\mu \partial^\mu + m^2\right) M_{ij}\pexp{\pm ipx} \\
        &= M_{ij} \left(\partial_\mu \partial^\mu + m^2\right) \pexp{\pm ipx} \\
        &= M_{ij} \Big(\partial_\mu \big[ \pm i \pexp{\pm ipx} p^\mu \big] + \pexp{\pm ipx} m^2\Big) \\
        &= M_{ij} \Big(\pm i \big[ \partial_\mu \pexp{\pm ipx} \big] p^\mu + \pexp{\pm ipx} m^2\big) \\
        &= M_{ij} \Big([\pm i]^2 \pexp{\pm ipx} p_\mu p^\mu + \pexp{\pm ipx} m^2\Big) \\
        &= M_{ij} \pexp{\pm ipx} \Big(- p_\mu p^\mu + m^2\Big) \\
        &= \phi_{ij} \Big(- p_0 p^0 - p_a p^a + m^2\Big) \\
        &= \phi_{ij} \Big(- EE - \big[-\vv{p}\big] \cdot \vv{p} + m^2\Big) \\
        &= \phi_{ij} \Big(- E^2 + \vv{p}^2 + m^2\Big) \\
        &= \phi_{ij} \Big(- E^2 + E^2 \Big) \\
        &= 0 \: ,
\end{split}
\end{align}
hvor den relativistiske energi er $E = \sqrt{\vv{p}^2 + m^2}$. Altså er planbølger på formen $\phi_{ij} = M_{ij}\exp(\pm ipx)$ løsninger til bevægelseligningen fra \cref{eq:Opg2_Q2_EquationOfMotion}.


%%%%%%%%%%%%%%%%%%%%%%%%%

\paragraph[4) Konjugeret felt er $\pi_{ij}  = \partial^0 \psi_{ij}$]{\textbf{4)}}

Vi er blevet givet, at det konjugerede felt er
\begin{align} \label{eq:Opg2_A4_ConjugatedMomentaField_RepeatedInAnswer}
    \pi_{ij} &= \pdif{\L}{(\partial_0 \phi_{ij})} \: ,
\end{align}
samt at $\phi_{ij} = \phi_{ji}$. Vi indsætter derfor blot Lagrangedensiteten fra \cref{eq:Opg2_Lagrangian} i ligningen for det konjugerede felt (\cref{eq:Opg2_Q4_ConjugatedMomentaField,eq:Opg2_A4_ConjugatedMomentaField_RepeatedInAnswer}) (step a), benytter at indekserne på feltet $\phi_{ij}$ kan byttes rundt uden omkostning (step b), samt benytter produktreglen til at differentiere produktfunktionen (step c).\\
\begin{align}
\begin{split}
    \pi_{ij} &= \pdif{\L}{(\partial_0 \phi_{ij})} \\
        &\xleq{(a)} \pdif{}{(\partial_0 \phi_{ij})} \left( \inv{2} \partial_\mu \phi_{kl} \partial^\mu \phi_{lk} - \inv{2} m^2 \phi_{kl} \phi_{lk} \right) \\
        &= \pdif{}{(\partial_0 \phi_{ij})} \left( \inv{2} \partial_\mu \phi_{kl} \partial^\mu \phi_{lk} \right) \\
        &= \pdif{}{(\partial_0 \phi_{ij})} \left( \inv{2} \partial_0 \phi_{kl} \partial^0 \phi_{lk} \right) \\
        &\xleq{(b)} \pdif{}{(\partial_0 \phi_{ij})} \left( \inv{2} \partial_0 \phi_{kl} \partial^0 \phi_{kl} \right) \\
        &= \pdif{}{(\partial_0 \phi_{ij})} \left( \inv{2} g^{00} \partial_0 \phi_{kl} \partial_0 \phi_{kl} \right) \\
        &= \inv{2} g^{00} \pdif{}{(\partial_0 \phi_{ij})} \left( \partial_0 \phi_{kl} \partial_0 \phi_{kl} \right) \\
        &\xleq{(c)} \inv{2} g^{00} \left( \pdif{}{(\partial_0 \phi_{ij})} \left[\partial_0 \phi_{kl}\right] \partial_0 \phi_{kl} + \partial_0 \phi_{kl} \pdif{}{(\partial_0 \phi_{ij})} \left[\partial_0 \phi_{kl}\right] \right) \\
        &= \inv{2} g^{00} \left( \delta_k^i \delta_l^j \partial_0 \phi_{kl} + \delta_k^i \delta_l^j \partial_0 \phi_{kl} \right) \\
        &= \inv{2} g^{00} \left( \partial_0 \phi_{ij} + \partial_0 \phi_{ij} \right) \\
        &= g^{00} \partial_0 \phi_{ij} \\
        &= \partial^0 \phi_{ij} \: .
\end{split}
\end{align}
Det er dermed vist, at det konjugerede felt er den kontravariante tidsafledede af feltet, $\pi_{ij} = \partial^0 \phi_{ij}$.


%%%%%%%%%%%%%%%%%%%%%%%%%

\paragraph[5) Hamiltonfunktion med felter]{\textbf{5)}}

Hamiltonfunktionen er givet som
\begin{align} \label{eq:Opg2_A5_HamiltonianDefinition}
    H &= \int \dd^3\vv{x} \H \: ,
\end{align}
hvor $\H$ er Hamiltondensiteten defineret som
\begin{align} \label{eq:Opg2_A5_HamiltonianDensityDefinition}
    \H &= \sum_{ij} \pi_{ij} \left( \partial_0 \phi_{ij} \right) - \L \: .
\end{align}

I kurset benyttes metrikken $g_{\mu\nu} = \mathrm{diag}(1,\, -\id)$, altså at de rummelige koordinater i de kovariante firvektorer får et negativt fortegn, mens tidskoordinatet forbliver det samme. For differentialoperatorer er det dog den kontravariante firvektors rummelige koordinater, som får det negative fortegn, da
\begin{align}
    \partial^\mu &= \pdif{}{x_\mu} = \left( \partial_0,\, -\Grad{} \right) \: .
\end{align}
Af denne grund kan vi også skrive $\pi_{ij}$ som
\begin{align}
    \pi_{ij} &= \partial^0 \phi_{ij}
        = g^{00} \partial_0 \phi_{ij}
        = 1 \partial_0 \phi_{ij}
        = \partial_0 \phi_{ij} \: .
\end{align}
Samtidig har vi, at $\phi_{ij} = \phi_{ji}$, hvorfor
\begin{align}
    \pi_{ij} &= \partial^0 \phi_{ij}
        = \partial^0 \phi_{ji}
        = \pi_{ji} \: .
\end{align}
Sammensættes dette kan vi skrive $\pi_{ji} = \partial_0 \phi_{ji}$.

Nu beregnes Hamiltondensiteten ved \cref{eq:Opg2_A5_HamiltonianDensityDefinition}, hvor der ligesom for $\phi_{ij} \phi_{ji}$ (\cref{eq:Opg2_ImplicitSums}) er en implicit sum for $\pi_{ij} \pi_{ji}$, og det romerske bogstav $a$ er brugt for implicit summering over de rummelige koordinater fra firvektoren.
\begin{align}
\begin{split}
    \H &= \sum_{ij} \pi_{ij} \left( \partial_0 \phi_{ij} \right) - \L \\
        &= \pi_{ij} \pi_{ji} - \inv{2} \partial_\mu \phi_{ij} \partial^\mu \phi_{ji} + \inv{2} m^2 \phi_{ij} \phi_{ji} \\
        &= \pi_{ij} \pi_{ji} - \inv{2} \partial_0 \phi_{ij} \partial^0 \phi_{ji} - \inv{2} \partial_a \phi_{ij} \partial^a \phi_{ji} + \inv{2} m^2 \phi_{ij} \phi_{ji} \\
        &= \pi_{ij} \pi_{ji} - \inv{2} \pi_{ij} \pi_{ji} - \inv{2} \Grad{\phi_{ij}} \cdot \left(-\Grad{\phi_{ji}}\right) + \inv{2} m^2 \phi_{ij} \phi_{ji} \\
        &= \inv{2} \pi_{ij} \pi_{ji} + \inv{2} \Grad{\phi_{ij}} \cdot \Grad{\phi_{ji}} + \inv{2} m^2 \phi_{ij} \phi_{ji} \: .
\end{split}
\end{align}

Indsættes denne Hamiltondensitet i \cref{eq:Opg2_A5_HamiltonianDefinition} fås Hamiltonfunktionen
\begin{align}
    H &= \int \dd^3\vv{x} \left[ \inv{2} \pi_{ij} \pi_{ji} + \inv{2} \Grad{\phi_{ij}} \cdot \Grad{\phi_{ji}} + \inv{2} m^2 \phi_{ij} \phi_{ji} \right] \: ,
\end{align}
hvilken var den, som skulle vises (\cref{eq:Opg2_Q5_Hamiltonian}).


%%%%%%%%%%%%%%%%%%%%%%%%%

\paragraph[6) Feltudvidelse for ''matrix''-feltet]{\textbf{6)}}

Betragter vi feltudvidelsen i \cref{eq:Opg2_Q6_ResultingField},
\begin{align}
    \phi_{ij} &= \int \frac{\dd^3\vv{p}}{(2\pi)^3} \invsqrt{2\omega_{\vv{p}}} \left(a_{\vv{p}ij} \bexp{-ipx} + a_{\vv{p}ij}\dagger \bexp{ipx} \right) \: ,
\end{align}
ses det, at der er stor lighed med feltudvidelsen for det reelle Klein-Gordonfelt i \cite[lign. 38]{problemSet2}, hvor vi sætter $a_{\vv{p}} = c_{\vv{p}}$ da feltet er reelt. Denne lighed er forventelig, da matrixfeltet er et set af reelle skalarfelter.

Hvis vi blot skal argumentere ud fra komponenterne i feltudvidelsen, så kan vi starte med at se på, hvad de forskellige komponenter gør. Vi ved at vi for en feltudvidelse ønsker et integral over normal modes med deres tilsvarende løsninger til bevægelsesligningen, hvilke for et skalarfelt er svarende til en Fouriertransformation, da løsningerne til Klein-Gordonligningen er planbølger, alstå eksponentialfunktioner. $a_{\vv{p}ij}$ er annihilationsoperatoren, som multipliceres med den positive energiløsning til Klein-Gordonligningen, mens $a_{\vv{p}ij}\dagger$ er kreationsoperatoren, som multipliceres med den negative energiløsning til Klein-Gordonligningen. $(2\pi)^3$ er en normaliseringsfaktor, og $\int \dd^3 \vv{p} / \sqrt{2 \omega_{\vv{p}}}$ er impulsrumsfaktoren, hvilken er Lorentzinvariant, hvorfor hele feltet bliver Lorentzinvariant (de andre dele er Lorentzinvariante også, hvorfor dette er sant).

I korte træk så skaber eller annihilerer vi partikler med de tilstande, som er løsninger til Klein-Gordonligningen, multiplicerer med normaliseringskonstanter og summerer over alle impulser, hvor vi selvfølgelig kun får et bidrag fra dem, hvor der er en partiklen med den givne impuls.


%%%%%%%%%%%%%%%%%%%%%%%%%

\paragraph[7) Vis relationer med Klein-Gordon-indreproduktet]{\textbf{7)}}

Vi betragter funktionen
\begin{align}
    u_{\vv{p}}(x) &= \frac{\pexp{-ipx}}{\sqrt{2\omega_{\vv{p}}}} \: ,
\end{align}
hvor $x$ og $p$ er firvektorer, $x^\mu = (t,\, \vv{x})$ og $p^\mu = (\omega_{\vv{p}},\, \vv{p})$, og $px = p_\mu x^\mu = p^\mu x_\mu$.

Vi skal nu regne de indre produkter i \cref{eq:Opg2_Q7_KGInnerProductsWithU}, til hvilket vi benytter os af definitionen af Klein-Gordonindreproduktet i \cref{eq:Opg2_Q7_KGInnerProduct}.

For ikke at gentage beregningerne for mange gange betegnes
\begin{align}
    u_{\vv{p}} = u_{\vv{p}}^- \quad \text{og} \quad u_{\vv{p}}^* = u_{\vv{p}}^+
\end{align}
i det følgende. Derved får vi også, at
\begin{align}
    (u_{\vv{p}}^+)^* = (u_{\vv{p}}^*)^* = u_{\vv{p}} = u_{\vv{p}}^- \: .
\end{align}

Først beregnes $(u_{\vv{p}},u_{\vv{p}'})$ og $(u_{\vv{p}}^*,u_{\vv{p}'}^*)$:
\begin{align}
\begin{split}
    (u_{\vv{p}}^\pm,u_{\vv{p}}^\pm) &= i \int \dd^3\vv{x} \left[ u_{\vv{p}}^\mp \pdif{u_{\vv{p}'}^\pm}{t} - \pdif{u_{\vv{p}}^\mp}{t} u_{\vv{p}'}^\pm \right] \\
        &= i \int \dd^3\vv{x} \left[ \frac{\pexp{\mp ipx}}{\sqrt{2\omega_{\vv{p}}}} \pdif{}{t} \left( \frac{\pexp{\pm ip'x}}{\sqrt{2\omega_{\vv{p}'}}} \right) - \pdif{}{t} \left( \frac{\pexp{\mp ipx}}{\sqrt{2\omega_{\vv{p}}}} \right) \frac{\pexp{\pm ip'x}}{\sqrt{2\omega_{\vv{p}'}}} \right] \\
        &= i \int \dd^3\vv{x} \inv{2\sqrt{\omega_{\vv{p}} \omega_{\vv{p}'}}} \Big[ \pexp{\mp ipx} \big\{\pm i \omega_{\vv{p}'} \pexp{\pm ip'x} \big\} \\
            &\qquad\qquad\qquad\qquad\qquad - \big\{ \mp i \omega_{\vv{p}} \pexp{\mp ipx} \big\} \pexp{\pm ip'x} \Big] \\
        &= i \int \dd^3\vv{x} \inv{2\sqrt{\omega_{\vv{p}} \omega_{\vv{p}'}}} \Big[ \pexp{\mp ipx} \big\{\pm i \omega_{\vv{p}'} \pexp{\pm ip'x} \big\} \\
            &\qquad\qquad\qquad\qquad\qquad + \big\{ \pm i \omega_{\vv{p}} \pexp{\mp ipx} \big\} \pexp{\pm ip'x} \Big] \\
        &= \pm i^2 \int \dd^3\vv{x} \inv{2} \left[ \sqf{\omega_{\vv{p}'}}{\omega_{\vv{p}}} + \sqf{\omega_{\vv{p}}}{\omega_{\vv{p}'}} \right] \pexp{\mp ipx} \pexp{\pm ip'x} \\
        &= \mp \int \dd^3\vv{x} \inv{2} \left[ \sqf{\omega_{\vv{p}'}}{\omega_{\vv{p}}} + \sqf{\omega_{\vv{p}}}{\omega_{\vv{p}'}} \right] \pexp{\mp i[p - p']x} \\
        &= \mp \inv{2} \left[ \sqf{\omega_{\vv{p}'}}{\omega_{\vv{p}}} + \sqf{\omega_{\vv{p}}}{\omega_{\vv{p}'}} \right] (2\pi)^3 \delta^3(\vv{p} - \vv{p}') \pexp{\mp i[\omega_{\vv{p}} - \omega_{\vv{p}'}]t} \\
        &= \mp \inv{2} \left[ \sqf{\omega_{\vv{p}}}{\omega_{\vv{p}}} + \sqf{\omega_{\vv{p}}}{\omega_{\vv{p}}} \right] (2\pi)^3 \delta^3(\vv{p} - \vv{p}') \pexp{\mp i[\omega_{\vv{p}} - \omega_{\vv{p}}]t} \\
        &= \mp \inv{2} \left[ 1 + 1 \right] (2\pi)^3 \delta^3(\vv{p} - \vv{p}') \pexp{0} \\
        &= \mp (2\pi)^3 \delta^3(\vv{p} - \vv{p}') \: ,
\end{split}
\end{align}
altså
\begin{subequations}
\begin{align}
    (u_{\vv{p}},u_{\vv{p}}) &= (2\pi)^3 \delta^3(\vv{p} - \vv{p}') \quad \text{og} \\
    (u_{\vv{p}}^*,u_{\vv{p}}^*) &= - (2\pi)^3 \delta^3(\vv{p} - \vv{p}') \: ,
\end{align}
\end{subequations}
idet at deltafunktioner fremkommer ved $\int \dd k \pexp{iky} = 2\pi\delta(y)$, at $\delta(y-y') = \delta(y'-y)$ og at $f(y') \delta(y - y') = f(y) \delta(y - y')$.

Dernæst beregnes $(u_{\vv{p}}^*,u_{\vv{p}'})$ og $(u_{\vv{p}},u_{\vv{p}'}^*)$:
\begin{align}
\begin{split}
    (u_{\vv{p}}^\pm,u_{\vv{p}}^\mp) &= i \int \dd^3\vv{x} \left[ u_{\vv{p}}^\mp \pdif{u_{\vv{p}'}^\mp}{t} - \pdif{u_{\vv{p}}^\mp}{t} u_{\vv{p}'}^\mp \right] \\
        &= i \int \dd^3\vv{x} \inv{2\sqrt{\omega_{\vv{p}} \omega_{\vv{p}'}}} \Big[ \pexp{\mp ipx} \big\{\mp i \omega_{\vv{p}'} \pexp{\mp ip'x} \big\} \\
            &\qquad\qquad\qquad\qquad\qquad - \big\{ \mp i \omega_{\vv{p}} \pexp{\mp ipx} \big\} \pexp{\mp ip'x} \Big] \\
        &= \mp i^2 \int \dd^3\vv{x} \inv{2\sqrt{\omega_{\vv{p}} \omega_{\vv{p}'}}} \Big[ \pexp{\mp ipx} \big\{\omega_{\vv{p}'} \pexp{\mp ip'x} \big\} \\
            &\qquad\qquad\qquad\qquad\qquad - \big\{ \omega_{\vv{p}} \pexp{\mp ipx} \big\} \pexp{\mp ip'x} \Big] \\
        &= \mp \int \dd^3\vv{x} \inv{2} \left[ \sqf{\omega_{\vv{p}'}}{\omega_{\vv{p}}} - \sqf{\omega_{\vv{p}}}{\omega_{\vv{p}'}} \right] \pexp{\mp i[p + p']x} \\
        &= \mp \inv{2} \left[ \sqf{\omega_{\vv{p}'}}{\omega_{\vv{p}}} - \sqf{\omega_{\vv{p}}}{\omega_{\vv{p}'}} \right] (2\pi)^3 \delta^3(\vv{p} + \vv{p}') \pexp{\mp i[\omega_{\vv{p}} + \omega_{\vv{p}'}]t} \\
        &= \mp \inv{2} \left[ \sqf{\omega_{-\vv{p}}}{\omega_{\vv{p}}} - \sqf{\omega_{\vv{p}}}{\omega_{-\vv{p}}} \right] (2\pi)^3 \delta^3(\vv{p} + \vv{p}') \pexp{\mp i[\omega_{\vv{p}} + \omega_{-\vv{p}}]t} \\
        &= \mp \inv{2} \left[ \sqf{\omega_{\vv{p}}}{\omega_{\vv{p}}} - \sqf{\omega_{\vv{p}}}{\omega_{\vv{p}}} \right] (2\pi)^3 \delta^3(\vv{p} + \vv{p}') \pexp{\mp i[\omega_{\vv{p}} + \omega_{\vv{p}}]t} \\
        &= 0 \: ,
\end{split}
\end{align}
idet
\begin{align}
    \omega_{\vv{p}} &= \sqrt{\vv{p}^2 + m^2}
        \quad \Rightarrow \quad
    \omega_{-\vv{p}} = \sqrt{\left(-\vv{p}\right)^2 + m^2} = \sqrt{\vv{p}^2 + m^2} = \omega_{\vv{p}} \: ,
\end{align}
altså får vi at
\begin{align}
    (u_{\vv{p}}^*,u_{\vv{p}}) &= (u_{\vv{p}},u_{\vv{p}}^*) = 0 \: .
\end{align}

Dermed har vi vist de fire indre produkter i \cref{eq:Opg2_Q7_KGInnerProductsWithU}.


%%%%%%%%%%%%%%%%%%%%%%%%%

\paragraph[8) Relationer for kreations- og annihilationsoperatorerne med \\ Klein-Gordon-indreproduktet]{\textbf{8)}}

Definerer vi, som i besvarelsen til \textbf{7)}, $u_{\vv{p}} = u_{\vv{p}}^-$ og $u_{\vv{p}}^* = u_{\vv{p}}^+$ for short-hand notation og husker, at $u_{\vv{p}}^\pm = \pexp{\pm ipx}/\sqrt{2\omega_{\vv{p}}}$, hvorfor
\begin{align}
\begin{split}
    \phi_{ij} &= \int \frac{\dd^3\vv{p}}{(2\pi)^3} \invsqrt{2\omega_{\vv{p}}} \left(a_{\vv{p}ij} \bexp{-ipx} + a_{\vv{p}ij}\dagger \bexp{ipx} \right) \\
        &= \int \frac{\dd^3\vv{p}}{(2\pi)^3} \left(a_{\vv{p}ij} u_{\vv{p}}^- + a_{\vv{p}ij}\dagger u_{\vv{p}}^+ \right) \: ,
\end{split}
\end{align}
så får vi, at
\begin{align} \label{eq:Opg2_A8_CalculationOfStepOperators}
\begin{split}
    \big(u_{\vv{p}}^\pm,\phi_{ij}(p')\big)
        &= i \int \dd^3\vv{x} \left[ u_{\vv{p}}^\mp \pdif{\phi_{ij}(p')}{t} - \pdif{u_{\vv{p}}^\mp}{t} \phi_{ij}(p') \right] \\
        &= i \int \dd^3\vv{x} \bigg[ u_{\vv{p}}^\mp \pdif{}{t}\left( \int \frac{\dd^3\vv{p}'}{(2\pi)^3} \left\{a_{\vv{p}'ij} u_{\vv{p}'}^- + a_{\vv{p}'ij}\dagger u_{\vv{p}'}^+ \right\} \right) \\
            &\qquad\qquad\quad - \pdif{u_{\vv{p}}^\mp}{t} \left( \int \frac{\dd^3\vv{p}'}{(2\pi)^3} \left\{a_{\vv{p}'ij} u_{\vv{p}'}^- + a_{\vv{p}'ij}\dagger u_{\vv{p}'}^+ \right\} \right) \bigg] \\
        &= i \int \frac{\dd^3\vv{x} \, \dd^3\vv{p}'}{(2\pi)^3} \Bigg[ u_{\vv{p}}^\mp \pdif{}{t}\left( a_{\vv{p}'ij} u_{\vv{p}'}^- + a_{\vv{p}'ij}\dagger u_{\vv{p}'}^+ \right) \\
            &\qquad\qquad\qquad\quad - \pdif{u_{\vv{p}}^\mp}{t} \left( a_{\vv{p}'ij} u_{\vv{p}'}^- + a_{\vv{p}'ij}\dagger u_{\vv{p}'}^+ \right) \Bigg] \\
        &= i \int \frac{\dd^3\vv{x} \, \dd^3\vv{p}'}{(2\pi)^3} \Bigg[ u_{\vv{p}}^\mp \left( a_{\vv{p}'ij} \pdif{u_{\vv{p}'}^-}{t} + a_{\vv{p}'ij}\dagger \pdif{u_{\vv{p}'}^+}{t} \right) \\
            &\qquad\qquad\qquad\quad - \pdif{u_{\vv{p}}^\mp}{t} \left( a_{\vv{p}'ij} u_{\vv{p}'}^- - a_{\vv{p}'ij}\dagger u_{\vv{p}'}^+ \right) \Bigg] \\
        &= i \int \frac{\dd^3\vv{x} \, \dd^3\vv{p}'}{(2\pi)^3} \Bigg[ a_{\vv{p}'ij} \left( u_{\vv{p}}^\mp \pdif{u_{\vv{p}'}^-}{t} - \pdif{u_{\vv{p}}^\mp}{t} u_{\vv{p}'}^- \right) \\
            &\qquad\qquad\qquad\quad + a_{\vv{p}'ij}\dagger \left( u_{\vv{p}}^\mp \pdif{u_{\vv{p}'}^+}{t} - \pdif{u_{\vv{p}}^\mp}{t} u_{\vv{p}'}^+ \right) \Bigg] \\
        &= \int \frac{\dd^3\vv{p}'}{(2\pi)^3} \bigg[ a_{\vv{p}'ij} ( u_{\vv{p}}^\pm , u_{\vv{p}'}^- ) + a_{\vv{p}'ij}\dagger ( u_{\vv{p}}^\pm , u_{\vv{p}'}^+ ) \bigg] \: ,
\end{split}
\end{align}
hvor Leibniz integrationsregel\footnote{
    Leibniz's integrationsregl er \cite{wiki:leibnizIntegrationRule} \label{footnote:LeibnitzIntegrationRule}
    \begin{align*}
        \dif{}{x}\left( \int_{a(x)}^{b(x)} \dd y \, f(x,y) \right) &= f\big(x, b(x)\big) \dif{b(x)}{x} - f\big(x, a(x)\big) \dif{a(x)}{x} + \int_{a(x)}^{b(x)} \dd y \, \pdif{f(x,y)}{x} \: .
    \end{align*}
    Når $a(x)$ og $b(x)$ er konstanter i stedet for funktioner ($a$, $b$), så bliver Leibniz integrationsregl
    \begin{align*}
        \dif{}{x}\left( \int_a^b \dd y \, f(x,y) \right) &= \int_a^b \dd y \, \pdif{f(x,y)}{x} \: .
    \end{align*}
}
er blevet benyttet for at bytte integralet og den partielle differentialoperator, og der ingen ekstra faktor kommer, da grænserne er konstanter.

Fra \cref{eq:Opg2_A8_CalculationOfStepOperators} får vi altså, at
\begin{subequations}
\begin{align}
    \begin{split}
        (u_{\vv{p}},\phi_{ij}) &= \int \frac{\dd^3\vv{p}'}{(2\pi)^3} \Big[ a_{\vv{p}'ij} ( u_{\vv{p}} , u_{\vv{p}'} ) + a_{\vv{p}'ij}\dagger ( u_{\vv{p}} , u_{\vv{p}'}^* ) \Big] \\
            &= \int \frac{\dd^3\vv{p}'}{(2\pi)^3} \Big[ a_{\vv{p}'ij} (2\pi)^3\delta^3(\vv{p} - \vv{p}') + 0 \Big] \\
            &= a_{\vv{p}ij}\dagger \: ,
    \end{split}
    \\
    \begin{split}
        (u_{\vv{p}}^*,\phi_{ij}) &= \int \frac{\dd^3\vv{p}'}{(2\pi)^3} \Big[ a_{\vv{p}'ij} ( u_{\vv{p}}^* , u_{\vv{p}'} ) + a_{\vv{p}'ij}\dagger ( u_{\vv{p}}^* , u_{\vv{p}'}^* ) \Big] \\
            &= \int \frac{\dd^3\vv{p}'}{(2\pi)^3} \Big[ 0 + a_{\vv{p}'ij}\dagger  \big\{ -(2\pi)^3\delta^3(\vv{p} - \vv{p}') \big\} \Big] \\
            &= -a_{\vv{p}ij}\dagger \: ,
    \end{split}
\end{align}
\end{subequations}
hvilke er \cref{eq:Opg2_Q8_StepOperatorsUsingKGInnerProduct}, som vi skulle vise.


%%%%%%%%%%%%%%%%%%%%%%%%%

\paragraph[9) Bosonisk kommutationsrelation for kreations- og annihilations-operatorerne]{\textbf{9)}}

Fra \textbf{8)} har vi, at $a_{\vv{p}ij}$ og $a_{\vv{p}ij}\dagger$ er givet som (\cref{eq:Opg2_Q8_StepOperatorsUsingKGInnerProduct})
\begin{subequations}
\begin{align}
    \begin{split}
        a_{\vv{p}ij} &= (u_{\vv{p}},\phi_{ij}) \\
            &= i \int \dd^3\vv{x} \left( u_{\vv{p}}^* \pdif{\phi_{ij}}{t} - \pdif{u_{\vv{p}}^*}{t} \phi_{ij} \right) \\
            &= i \int \dd^3\vv{x} \left( u_{\vv{p}}^* \pi_{ij} - i \omega_{\vv{p}} u_{\vv{p}}^* \phi_{ij} \right)
                \: , \qquad \text{og}
    \end{split}
    \\
    \begin{split}
        a_{\vv{p}ij}\dagger &= - (u_{\vv{p}}^*,\phi_{ij}) \\
            &= - i \int \dd^3\vv{x} \left( u_{\vv{p}} \pdif{\phi_{ij}}{t} - \pdif{u_{\vv{p}}}{t} \phi_{ij} \right) \\
            &= - i \int \dd^3\vv{x} \left( u_{\vv{p}} \pi_{ij} + i \omega_{\vv{p}} u_{\vv{p}} \phi_{ij} \right) \: .
    \end{split}
\end{align}
\end{subequations}
Fra dette kan vi beregne de bosoniske kommutatorrelationer, hvor vi for nemhedens skyld beregner de to tilfælde samlet ved at betegne $a^+ = a\dagger$ og $a^- = a$ og igen betegne $u_{\vv{p}}^* = u_{\vv{p}}^+$ og $u_{\vv{p}} = u_{\vv{p}}^-$:
\begin{align} \label{eq:Opg2_A9_BosonicCommutatorRelationsCalculation_Part1}
\begin{split}
    \commutator{a_{\vv{p}ij}}{\, a_{\vv{p}'kl}^\pm} &= \commutator{(u_{\vv{p}}, \phi_{ij})}{\, (u_{\vv{p}'}^\pm, \phi_{kl})} \\
        &= i^2 \int \dd^3\vv{x} \, \dd^3\vv{x}' \commutator{u_{\vv{p}}^+ \pi_{ij} - i \omega_{\vv{p}} u_{\vv{p}}^+ \phi_{ij}}{\, \mp \left( u_{\vv{p}'}^\mp \pi_{kl} \pm i \omega_{\vv{p}'} u_{\vv{p}'}^\mp \phi_{kl} \right)} \\
        &= \pm \int \dd^3\vv{x} \, \dd^3\vv{x}' \, u_{\vv{p}}^+ u_{\vv{p}'}^\mp \commutator{ \pi_{ij} - i \omega_{\vv{p}} \phi_{ij}}{\, \pi_{kl} \pm i \omega_{\vv{p}'} \phi_{kl}} \\
        &= \pm \int \dd^3\vv{x} \, \dd^3\vv{x}' \, u_{\vv{p}}^+ u_{\vv{p}'}^\mp \Big(
        \cancelto{0}{\commutator{\pi_{ij}}{\, \pi_{kl}}} \:\:\:
        \pm i \omega_{\vv{p}'} \commutator{\pi_{ij}}{\, \phi_{kl}} \\
            &\qquad\qquad\qquad\qquad\qquad\: - i \omega_{\vv{p}} \commutator{\phi_{ij}}{\, \pi_{kl}}
            \mp i^2 \omega_{\vv{p}} \omega_{\vv{p}'} \cancelto{0}{\commutator{\phi_{ij}}{\, \phi_{kl}}}
            \quad \Big) \\
        &= \pm \int \dd^3\vv{x} \, \dd^3\vv{x}' \, u_{\vv{p}}^+ u_{\vv{p}'}^\mp \big( \pm \omega_{\vv{p}'} + \omega_{\vv{p}} \big) \delta^3(\vv{x} - \vv{x}') \delta_{ik} \delta_{jl} \: ,
\end{split}
\end{align}
hvor vi har benyttet kommutatorrelationen mellem $\phi_{ij}$ og $\pi_{kl}$, \cref{eq:Opg2_Q9_CommutatorOfFieldAndItsTimeDerivativePi}.

Integrerer vi nu eksponentialfunktionerne og deltafunktionen fra \cref{eq:Opg2_A9_BosonicCommutatorRelationsCalculation_Part1} samt husker på, at vi taler om same-time kommutatorrelationer, altså $t = t'$, så får vi\\
\begin{align} \label{eq:Opg2_A9_BosonicCommutatorRelationsCalculation_IntegrateExponentials}
\begin{split}
    \int \dd^3\vv{x} \, \dd^3\vv{x}' \, u_{\vv{p}}^+ u_{\vv{p}'}^\mp \delta^3(\vv{x} - \vv{x}') &= \int \dd^3\vv{x} \, \dd^3\vv{x}' \, \frac{\pexp{i[px \mp p'x']}}{2 \sqrt{\omega_{\vv{p}} \omega_{\vv{p}'}}} \delta^3(\vv{x} - \vv{x}') \\
        &= \int \dd^3\vv{x} \, \frac{\pexp{i[\omega_{\vv{p}}t \mp \omega_{\vv{p}'}t']} \pexp{-i[\vv{p} \mp \vv{p}'] \cdot \vv{x}}}{2 \sqrt{\omega_{\vv{p}} \omega_{\vv{p}'}}} \\
        &= \int \dd^3\vv{x} \, \frac{\pexp{i[\omega_{\vv{p}} \mp \omega_{\vv{p}'}]t} \pexp{-i[\vv{p} \mp \vv{p}'] \cdot \vv{x}}}{2 \sqrt{\omega_{\vv{p}} \omega_{\vv{p}'}}} \\
        &= (2\pi)^3 \frac{\pexp{i[\omega_{\vv{p}} \mp \omega_{\vv{p}'}]t}}{2 \sqrt{\omega_{\vv{p}} \omega_{\vv{p}'}}} \delta^3(\vv{p} \mp \vv{p}') \: .
\end{split}
\end{align}

Sætter vi resultatet fra \cref{eq:Opg2_A9_BosonicCommutatorRelationsCalculation_IntegrateExponentials} tilbage i \cref{eq:Opg2_A9_BosonicCommutatorRelationsCalculation_Part1} og husker, at $\omega_{-\vv{p}} = \sqrt{(-\vv{p})^2 + m^2} = \sqrt{\vv{p}^2 + m^2} = \omega_{\vv{p}}$, så får vi at
\begin{align} \label{eq:Opg2_A9_BosonicCommutatorRelationsCalculation_Part2}
\begin{split}
    \commutator{a_{\vv{p}ij}}{\, a_{\vv{p}'kl}^\pm} &= \pm (2\pi)^3 \frac{\pexp{i[\omega_{\vv{p}} \mp \omega_{\vv{p}'}]t}}{2 \sqrt{\omega_{\vv{p}} \omega_{\vv{p}'}}} \delta^3(\vv{p} \mp \vv{p}') \big( \pm \omega_{\vv{p}'} + \omega_{\vv{p}} \big) \delta_{ik} \delta_{jl} \\
        &= \pm (2\pi)^3 \frac{\pexp{i[\omega_{\vv{p}} \mp \omega_{\pm \vv{p}}]t}}{2 \sqrt{\omega_{\vv{p}} \omega_{\pm \vv{p}}}} \delta^3(\vv{p} \mp \vv{p}') \big( \pm \omega_{\pm \vv{p}} + \omega_{\vv{p}} \big) \delta_{ik} \delta_{jl} \\
        &= \pm (2\pi)^3 \frac{\pexp{i[\omega_{\vv{p}} \mp \omega_{\vv{p}}]t}}{2 \omega_{\vv{p}}} \delta^3(\vv{p} \mp \vv{p}') \big( \pm \omega_{\vv{p}} + \omega_{\vv{p}} \big) \delta_{ik} \delta_{jl}
\end{split}
\end{align}

Altså er de bosoniske kommutatorrelationer
\begin{subequations} \label{eq:Opg2_A9_BosonicCommutatationRelations}
\begin{align}
    \commutator{a_{\vv{p}ij}}{\, a_{\vv{p}'kl}^\pm} &= (2\pi)^3 \frac{2 \omega_{\vv{p}} \pexp{0}}{2 \omega_{\vv{p}}} \delta^3(\vv{p} \mp \vv{p}') \delta_{ik} \delta_{jl}
        = (2\pi)^3 \delta^3(\vv{p} \mp \vv{p}') \delta_{ik} \delta_{jl} \: ,
        \label{eq:Opg2_A9_BosonicCommutatationRelations_a_aDagger} \\
    \commutator{a_{\vv{p}ij}}{\, a_{\vv{p}'kl}^\pm} &= - (2\pi)^3 \frac{\pexp{2i\omega_{\vv{p}}t}}{2 \omega_{\vv{p}}} \delta^3(\vv{p} \mp \vv{p}') \big( - \omega_{\vv{p}} + \omega_{\vv{p}} \big) \delta_{ik} \delta_{jl}
        = 0 \: .
        \label{eq:Opg2_A9_BosonicCommutatationRelations_a_a}
\end{align}
Fra \cref{eq:Opg2_A9_BosonicCommutatationRelations_a_aDagger} findes trivielt kommutatoren $\commutator{a_{\vv{p}ij}\dagger}{\, a_{\vv{p}'kl}}$ ved\footnote{
    Trivielt er $\commutator{A}{B} = AB - BA = - (BA - AB) = - \commutator{B}{A}$.
}
\begin{align} \label{eq:Opg2_A9_BosonicCommutatationRelations_aDagger_a}
    \commutator{a_{\vv{p}ij}\dagger}{\, a_{\vv{p}'kl}} &= - \commutator{a_{\vv{p}'kl}}{\, a_{\vv{p}ij}\dagger} = - (2\pi)^3 \delta^3(\vv{p} - \vv{p}') \delta_{ik} \delta_{jl} \: ,
\end{align}
og kommutatoren $\commutator{a_{\vv{p}ij}\dagger}{\, a_{\vv{p}'kl}\dagger}$ findes nemt fra \cref{eq:Opg2_A9_BosonicCommutatationRelations_a_a} ved at hermitisk konjugere\footnote{
    $\commutator{A\dagger}{B\dagger} = A\dagger B\dagger - B\dagger A\dagger = (BA)\dagger - (AB)\dagger = (BA - AB)\dagger = \commutator{B}{A}\dagger$.
} (eng: taking the conjugate transpose) denne
\begin{align} \label{eq:Opg2_A9_BosonicCommutatationRelations_aDagger_aDagger}
    \commutator{a_{\vv{p}ij}\dagger}{\, a_{\vv{p}'kl}\dagger} &= \commutator{a_{\vv{p}'kl}}{\, a_{\vv{p}ij}}\dagger
        = 0\dagger
        = 0 \: .
\end{align}
\end{subequations}


%%%%%%%%%%%%%%%%%%%%%%%%%

\paragraph[10) Hamiltonfunktion med kreations- og annihilationsoperatorer]{\textbf{10)}}

Det skal vises, at Hamiltonfunktionen fra \cref{eq:Opg2_Q5_Hamiltonian} er lig Hamiltonfunktionen opskrevet ud fra kreations- og annihilationsoperatorer i \cref{eq:Opg2_Q10_HamiltonianWithCreationAndAnnihilationOperators}.

Vi starter dermed med at beregne de indgående felter, hvor det vides, at feltet er givet som (\cref{eq:Opg2_Q6_ResultingField})
\begin{align}
    \phi_{ij} &= \int \frac{\dd^3\vv{p}}{(2\pi)^3} \left( a_{\vv{p}ij} u_{\vv{p}} + a_{\vv{p}ij}\dagger u_{\vv{p}}^* \right) \: ,
\end{align}
hvor $\omega_{\vv{p}} = \sqrt{\vv{p}^2 + m^2}$ og $u_{\vv{p}}(x) = \pexp{-ipx}/\sqrt{2\omega_{\vv{p}}}$, hvormed vi får
\begin{align}
    \pi_{ij} &= \int \frac{\dd^3\vv{p}}{(2\pi)^3} \left( a_{\vv{p}ij} \pdif{u_{\vv{p}}}{t} + a_{\vv{p}ij}\dagger \pdif{u_{\vv{p}}^*}{t} \right)
        = -i \int \frac{\dd^3\vv{p}}{(2\pi)^3} \omega_{\vv{p}} \left( a_{\vv{p}ij} u_{\vv{p}} - a_{\vv{p}ij}\dagger u_{\vv{p}}^* \right) \: , \\
    \Grad{\phi_{ij}} &= \int \frac{\dd^3\vv{p}}{(2\pi)^3} \left( a_{\vv{p}ij} \pdif{u_{\vv{p}}}{\vv{x}} + a_{\vv{p}ij}\dagger \pdif{u_{\vv{p}}^*}{\vv{x}} \right)
        = i \int \frac{\dd^3\vv{p}}{(2\pi)^3} \vv{p} \left( a_{\vv{p}ij} u_{\vv{p}} - a_{\vv{p}ij}\dagger u_{\vv{p}}^* \right) \: ,
\end{align}
hvor Leibnitz regel (\cref{footnote:LeibnitzIntegrationRule} på \cpageref{footnote:LeibnitzIntegrationRule}) igen er benyttet.

Udregnes nu de tre led indgående i Hamiltonfunktionen, \cref{eq:Opg2_Q5_Hamiltonian}, og huskes der at $\phi_{ij} = \phi_{ji}$, fås
\begin{align}
    \begin{split} \label{eq:Opg2_A10_PhiPhi}
        \phi_{ij} \phi_{ij} &= \int \frac{\dd^3\vv{p} \, \dd^3\vv{p}'}{(2\pi)^6} \left( a_{\vv{p}ij} u_{\vv{p}} + a_{\vv{p}ij}\dagger u_{\vv{p}}^* \right) \left( a_{\vv{p}'ij} u_{\vv{p}'} + a_{\vv{p}'ij}\dagger u_{\vv{p}'}^* \right) \\
            &= \int \frac{\dd^3\vv{p} \, \dd^3\vv{p}'}{(2\pi)^6} \Big( a_{\vv{p}ij} u_{\vv{p}} a_{\vv{p}'ij} u_{\vv{p}'} + a_{\vv{p}ij} u_{\vv{p}} a_{\vv{p}'ij}\dagger u_{\vv{p}'}^* \\
                &\qquad\qquad\qquad\quad + a_{\vv{p}ij}\dagger u_{\vv{p}}^* a_{\vv{p}'ij} u_{\vv{p}'} + a_{\vv{p}ij}\dagger u_{\vv{p}}^* a_{\vv{p}'ij}\dagger u_{\vv{p}'}^* \Big) \: ,
    \end{split}\\
    \begin{split} \label{eq:Opg2_A10_PiPi}
        \pi_{ij} \pi_{ij} &= (-i)^2 \int \frac{\dd^3\vv{p} \, \dd^3\vv{p}'}{(2\pi)^6} \left( a_{\vv{p}ij} u_{\vv{p}} - a_{\vv{p}ij}\dagger u_{\vv{p}}^* \right) \left( a_{\vv{p}'ij} u_{\vv{p}'} - a_{\vv{p}'ij}\dagger u_{\vv{p}'}^* \right) \: , \\
            &= - \int \frac{\dd^3\vv{p} \, \dd^3\vv{p}'}{(2\pi)^6} \omega_{\vv{p}} \omega_{\vv{p}'} \Big( a_{\vv{p}ij} u_{\vv{p}} a_{\vv{p}'ij} u_{\vv{p}'} - a_{\vv{p}ij} u_{\vv{p}} a_{\vv{p}'ij}\dagger u_{\vv{p}'}^* \\
                &\qquad\qquad\qquad\qquad\qquad\:\: - a_{\vv{p}ij}\dagger u_{\vv{p}}^* a_{\vv{p}'ij} u_{\vv{p}'} + a_{\vv{p}ij}\dagger u_{\vv{p}}^* a_{\vv{p}'ij}\dagger u_{\vv{p}'}^* \Big) \: ,
    \end{split}\\
    \begin{split} \label{eq:Opg2_A10_GradPhiGradPhi}
        \Grad{\phi_{ij}} \cdot \Grad{\phi_{ij}} &= i^2 \int \frac{\dd^3\vv{p} \, \dd^3\vv{p}'}{(2\pi)^6} \vv{p} \cdot \vv{p}' \left( a_{\vv{p}ij} u_{\vv{p}} - a_{\vv{p}ij}\dagger u_{\vv{p}}^* \right) \left( a_{\vv{p}'ij} u_{\vv{p}'} - a_{\vv{p}'ij}\dagger u_{\vv{p}'}^* \right) \\
            &= - \int \frac{\dd^3\vv{p} \, \dd^3\vv{p}'}{(2\pi)^6} \vv{p} \cdot \vv{p}' \Big( a_{\vv{p}ij} u_{\vv{p}} a_{\vv{p}'ij} u_{\vv{p}'} - a_{\vv{p}ij} u_{\vv{p}} a_{\vv{p}'ij}\dagger u_{\vv{p}'}^* \\
                &\qquad\qquad\qquad\qquad\qquad\: - a_{\vv{p}ij}\dagger u_{\vv{p}}^* a_{\vv{p}'ij} u_{\vv{p}'} + a_{\vv{p}ij}\dagger u_{\vv{p}}^* a_{\vv{p}'ij}\dagger u_{\vv{p}'}^* \Big)
    \end{split}
\end{align}

De udregnede led i \cref{eq:Opg2_A10_PhiPhi,eq:Opg2_A10_PiPi,eq:Opg2_A10_GradPhiGradPhi} indsættes i Hamiltonfunktionen, \cref{eq:Opg2_Q5_Hamiltonian}, hvor den implicitte sum nu er udskrevet, og vi får at
\begin{align} \label{eq:Opg2_A10_HamiltonianCalculationPart1}
\begin{split}
    H &= \sum_{i,j=1}^N \int \dd^3\vv{x} \left[ \inv{2} \pi_{ij} \pi_{ij} + \inv{2} \Grad{\phi_{ij}} \cdot \Grad{\phi_{ij}} + \inv{2} m^2 \phi_{ij} \phi_{ij} \right] \\
        &= \sum_{i,j=1}^N \int \frac{\dd^3\vv{x} \, \dd^3\vv{p} \, \dd^3\vv{p}'}{2 (2\pi)^6} \bigg[ u_{\vv{p}} u_{\vv{p}'} a_{\vv{p}ij} a_{\vv{p}'ij} \left( - \omega_{\vv{p}} \omega_{\vv{p}'} - \vv{p} \cdot \vv{p}' + m^2 \right) \\
            &\qquad\qquad\qquad\qquad\qquad\:\: + u_{\vv{p}} u_{\vv{p}'}^* a_{\vv{p}ij} a_{\vv{p}'ij}\dagger \left( \omega_{\vv{p}} \omega_{\vv{p}'} + \vv{p} \cdot \vv{p}' + m^2 \right) \\
            &\qquad\qquad\qquad\qquad\qquad\:\: + u_{\vv{p}}^* u_{\vv{p}'} a_{\vv{p}ij}\dagger a_{\vv{p}'ij} \left( \omega_{\vv{p}} \omega_{\vv{p}'} + \vv{p} \cdot \vv{p}' + m^2 \right) \\
            &\qquad\qquad\qquad\qquad\qquad\:\: + u_{\vv{p}}^* u_{\vv{p}'}^* a_{\vv{p}ij}\dagger a_{\vv{p}'ij}\dagger \left( - \omega_{\vv{p}} \omega_{\vv{p}'} - \vv{p} \cdot \vv{p}' + m^2 \right) \bigg]
\end{split}
\end{align}

Vi kan nu kommutere integralerne således, at vi kan udregne integralet af planbølgerne, $u_{\vv{p}}$ mht. $\vv{x}$, hvorved vi får
\begin{subequations}
\begin{align}
    \begin{split} \label{eq:Opg2_A10_IntegrationOfUU}
        \int \dd^3 \vv{x} u_{\vv{p}} u_{\vv{p}'} &= \int \dd^3 \vv{x} \: \frac{\pexp{-i[p + p']x}}{2 \sqrt{\omega_{\vv{p}} \omega_{\vv{p}'}}} \\
            &= \frac{\pexp{-i\left[\omega_{\vv{p}} + \omega_{\vv{p}'}\right]t}}{2 \sqrt{\omega_{\vv{p}} \omega_{\vv{p}'}}} (2\pi)^3 \delta^3(\vv{p} + \vv{p}') \: ,
    \end{split} \\
    \begin{split} \label{eq:Opg2_A10_IntegrationOfU*U}
        \int \dd^3 \vv{x} u_{\vv{p}}^* u_{\vv{p}'} &= \int \dd^3 \vv{x} \: \frac{\pexp{i[p - p']x}}{2 \sqrt{\omega_{\vv{p}} \omega_{\vv{p}'}}} \\
            &= \frac{\pexp{i\left[\omega_{\vv{p}} - \omega_{\vv{p}'}\right]t}}{2 \sqrt{\omega_{\vv{p}} \omega_{\vv{p}'}}} (2\pi)^3 \delta^3(\vv{p} - \vv{p}') \: ,
    \end{split} \\
    \begin{split} \label{eq:Opg2_A10_IntegrationOfU*U*}
        \int \dd^3 \vv{x} u_{\vv{p}}^* u_{\vv{p}'}^* &= \left( \int \dd^3 \vv{x} u_{\vv{p}} u_{\vv{p}'} \right)^* \\
            &= \left( \frac{\pexp{-i\left[\omega_{\vv{p}} + \omega_{\vv{p}'}\right]t}}{2 \sqrt{\omega_{\vv{p}} \omega_{\vv{p}'}}} (2\pi)^3 \delta^3(\vv{p} + \vv{p}') \right)^* \\
            &= \frac{\pexp{i\left[\omega_{\vv{p}} + \omega_{\vv{p}'}\right]t}}{2 \sqrt{\omega_{\vv{p}} \omega_{\vv{p}'}}} (2\pi)^3 \delta^3(\vv{p} + \vv{p}') \: , \qquad\quad \text{og}
    \end{split} \\
    \begin{split} \label{eq:Opg2_A10_IntegrationOfUU*}
        \int \dd^3 \vv{x} u_{\vv{p}} u_{\vv{p}'}^* &= \left( \int \dd^3 \vv{x} u_{\vv{p}}^* u_{\vv{p}'} \right)^* \\
            &= \left( \frac{\pexp{i\left[\omega_{\vv{p}} - \omega_{\vv{p}'}\right]t}}{2 \sqrt{\omega_{\vv{p}} \omega_{\vv{p}'}}} (2\pi)^3 \delta^3(\vv{p} - \vv{p}') \right)^* \\
            &= \frac{\pexp{-i\left[\omega_{\vv{p}} - \omega_{\vv{p}'}\right]t}}{2 \sqrt{\omega_{\vv{p}} \omega_{\vv{p}'}}} (2\pi)^3 \delta^3(\vv{p} - \vv{p}') \: .
    \end{split}
\end{align}
\end{subequations}

Indsættes \cref{eq:Opg2_A10_IntegrationOfUU,eq:Opg2_A10_IntegrationOfU*U,eq:Opg2_A10_IntegrationOfU*U*,eq:Opg2_A10_IntegrationOfUU*} i Hamiltonfunktionen, \cref{eq:Opg2_A10_HamiltonianCalculationPart1}, og husker vi at $\omega_{\vv{p}}^2 = \vv{p}^2 + m^2$ og dermed at $\omega_{-\vv{p}} = \sqrt{(-\vv{p})^2 + m^2} = \sqrt{\vv{p}^2 + m^2} = \omega_{\vv{p}}$, så får vi\\
\begin{align} \label{eq:Opg2_A10_HamiltonianCalculationPart2}
\begin{split}
    H &= \sum_{i,j=1}^N \int \frac{\dd^3\vv{p} \, \dd^3\vv{p}'}{2 (2\pi)^3} \\
            &\qquad \bigg[ \frac{\pexp{-i\left[\omega_{\vv{p}} + \omega_{\vv{p}'}\right]t}}{2 \sqrt{\omega_{\vv{p}} \omega_{\vv{p}'}}} \delta^3(\vv{p} + \vv{p}') a_{\vv{p}ij} a_{\vv{p}'ij} \left( - \omega_{\vv{p}} \omega_{\vv{p}'} - \vv{p} \cdot \vv{p}' + m^2 \right) \\
            &\qquad\: + \frac{\pexp{-i\left[\omega_{\vv{p}} - \omega_{\vv{p}'}\right]t}}{2 \sqrt{\omega_{\vv{p}} \omega_{\vv{p}'}}} \delta^3(\vv{p} - \vv{p}') a_{\vv{p}ij} a_{\vv{p}'ij}\dagger \left( \omega_{\vv{p}} \omega_{\vv{p}'} + \vv{p} \cdot \vv{p}' + m^2 \right) \\
            &\qquad\: + \frac{\pexp{i\left[\omega_{\vv{p}} - \omega_{\vv{p}'}\right]t}}{2 \sqrt{\omega_{\vv{p}} \omega_{\vv{p}'}}} \delta^3(\vv{p} - \vv{p}') a_{\vv{p}ij}\dagger a_{\vv{p}'ij} \left( \omega_{\vv{p}} \omega_{\vv{p}'} + \vv{p} \cdot \vv{p}' + m^2 \right) \\
            &\qquad\: + \frac{\pexp{i\left[\omega_{\vv{p}} + \omega_{\vv{p}'}\right]t}}{2 \sqrt{\omega_{\vv{p}} \omega_{\vv{p}'}}} \delta^3(\vv{p} + \vv{p}') a_{\vv{p}ij}\dagger a_{\vv{p}'ij}\dagger \left( - \omega_{\vv{p}} \omega_{\vv{p}'} - \vv{p} \cdot \vv{p}' + m^2 \right) \bigg] \\
        &=\sum_{i,j=1}^N \int \frac{\dd^3\vv{p}}{2 (2\pi)^3} \bigg[ \frac{\pexp{-2i\omega_{\vv{p}}t}}{2 \omega_{\vv{p}}} a_{\vv{p}ij} a_{-\vv{p}ij} \cancelto{0}{\left( - \omega_{\vv{p}}^2 + \vv{p}^2 + m^2 \right)} \\
            &\qquad\qquad\qquad\qquad + \inv{2 \omega_{\vv{p}}} a_{\vv{p}ij} a_{\vv{p}ij}\dagger \left( \omega_{\vv{p}}^2 + \vv{p}^2 + m^2 \right) \\
            &\qquad\qquad\qquad\qquad + \inv{2 \omega_{\vv{p}}} a_{\vv{p}ij}\dagger a_{\vv{p}ij} \left( \omega_{\vv{p}}^2 + \vv{p}^2 + m^2 \right) \\
            &\qquad\qquad\qquad\qquad + \frac{\pexp{2i\omega_{\vv{p}}t}}{2 \omega_{\vv{p}}} a_{\vv{p}ij}\dagger a_{-\vv{p}ij}\dagger \cancelto{0}{\left( - \omega_{\vv{p}}^2 + \vv{p}^2 + m^2 \right)} \quad\: \bigg] \\
        &=\sum_{i,j=1}^N \int \frac{\dd^3\vv{p}}{2 (2\pi)^3} \bigg[ \frac{2\omega_{\vv{p}}^2}{2 \omega_{\vv{p}}} a_{\vv{p}ij} a_{\vv{p}ij}\dagger + \frac{2\omega_{\vv{p}}^2}{2 \omega_{\vv{p}}} a_{\vv{p}ij}\dagger a_{\vv{p}ij} \bigg] \\
        &=\sum_{i,j=1}^N \int \frac{\dd^3\vv{p}}{2 (2\pi)^3} \omega_{\vv{p}} \bigg[ a_{\vv{p}ij} a_{\vv{p}ij}\dagger + a_{\vv{p}ij}\dagger a_{\vv{p}ij} \bigg] \\
        &=\sum_{i,j=1}^N \int \frac{\dd^3\vv{p}}{2 (2\pi)^3} \omega_{\vv{p}} \bigg[ a_{\vv{p}ij} a_{\vv{p}ij}\dagger - a_{\vv{p}ij}\dagger a_{\vv{p}ij} + a_{\vv{p}ij}\dagger a_{\vv{p}ij} + a_{\vv{p}ij}\dagger a_{\vv{p}ij} \bigg] \\
        &=\sum_{i,j=1}^N \int \frac{\dd^3\vv{p}}{2 (2\pi)^3} \omega_{\vv{p}} \bigg( \commutator{a_{\vv{p}ij}}{\, a_{\vv{p}ij}\dagger} + 2 a_{\vv{p}ij}\dagger a_{\vv{p}ij} \bigg) \: .
\end{split}
\end{align}

Kommutatoren i \cref{eq:Opg2_A10_HamiltonianCalculationPart2} viser sig at være problematisk, da den reelt er uendelig, når denne Hamiltonfunktion benyttes. Vi vil dog negligere ledet, da vi normalt blot kigger på energiforskelle og dermed vil dette led gå ud med det samme led fra det andet energiudtryk. Kommutatoren ses altså blot som værende en ''baggrundsenergi'', som vi ikke betragter. Under denne antagelse bliver Hamiltonfunktionen
\begin{align}
    H &= \sum_{i,j=1}^N \int \frac{\dd^3\vv{p}}{(2\pi)^3} \left[ \omega_{\vv{p}} a_{\vv{p}ij}\dagger a_{\vv{p}ij} \right] \: .
\end{align}
Hermed er det vist, at Hamiltonfunktionen for det kvantiserede felt kan skrives som \cref{eq:Opg2_Q10_HamiltonianWithCreationAndAnnihilationOperators}.


%%%%%%%%%%%%%%%%%%%%%%%%%

\paragraph[11) Feltoperator overholder Heisenbergs bevægelsesligning]{\textbf{11)}}

Det skal vises, at feltoperatoren beskrevet af \cref{eq:Opg2_Q6_ResultingField} overholder Heisenbergs bevægelsesligning, \cref{eq:Opg2_Q11_HeisenbergEquationOfMotion}, med hensyn til Hamiltonfunktionen i \cref{eq:Opg2_Q10_HamiltonianWithCreationAndAnnihilationOperators}. Her er fortsat benyttet notationen med $u_{\vv{p}}(x) = \pexp{-ipx}/\sqrt{2\omega_{\vv{p}}}$, hvorved kommutatoren mellem feltet og Hamiltonfunktionen er
\begin{align} \label{eq:Opg2_A11_FieldOperatorObeyingHeisenbergEquation}
\begin{split}
    \commutator{\phi_{ij}(x,p)}{H(p')} &= \sum_{k,l=1}^N \int \frac{\dd^3\vv{p} \, \dd^3\vv{p}'}{(2\pi)^6} \commutator{a_{\vv{p}ij} u_{\vv{p}} + a_{\vv{p}ij}\dagger u_{\vv{p}}^*}{\, \omega_{\vv{p}'} a_{\vv{p}'kl}\dagger a_{\vv{p}'kl}} \\
        &= \sum_{k,l=1}^N \int \frac{\dd^3\vv{p} \, \dd^3\vv{p}'}{(2\pi)^6} \omega_{\vv{p}'} \bigg( \commutator{a_{\vv{p}ij} u_{\vv{p}}}{\, a_{\vv{p}'kl}\dagger a_{\vv{p}'kl}} \\
            &\qquad\qquad\qquad\qquad\qquad\quad + \commutator{a_{\vv{p}ij}\dagger u_{\vv{p}}^*}{\, a_{\vv{p}'kl}\dagger a_{\vv{p}'kl}} \bigg) \\
        % &= \sum_{k,l=1}^N \int \frac{\dd^3\vv{p} \, \dd^3\vv{p}'}{(2\pi)^6} \omega_{\vv{p}'} \bigg( a_{\vv{p}'kl}\dagger \commutator{a_{\vv{p}ij}}{\, a_{\vv{p}'kl}} u_{\vv{p}} \\
        %     &\qquad\qquad\qquad\qquad\qquad\quad + \commutator{a_{\vv{p}ij}}{\, a_{\vv{p}'kl}\dagger} a_{\vv{p}'kl} u_{\vv{p}} \\
        %     &\qquad\qquad\qquad\qquad\qquad\quad + a_{\vv{p}'kl}\dagger \commutator{a_{\vv{p}ij}\dagger}{\, a_{\vv{p}'kl}} u_{\vv{p}}^* \\
        %     &\qquad\qquad\qquad\qquad\qquad\quad + \commutator{a_{\vv{p}ij}\dagger}{\, a_{\vv{p}'kl}\dagger} a_{\vv{p}'kl} u_{\vv{p}}^* \bigg) \\
        &= \sum_{k,l=1}^N \int \frac{\dd^3\vv{p} \, \dd^3\vv{p}'}{(2\pi)^6} \omega_{\vv{p}'} \bigg( a_{\vv{p}'kl}\dagger \cancelto{0}{\commutator{a_{\vv{p}ij}}{\, a_{\vv{p}'kl}}} u_{\vv{p}} \\
            &\qquad\qquad\qquad\qquad\qquad\quad + \commutator{a_{\vv{p}ij}}{\, a_{\vv{p}'kl}\dagger} a_{\vv{p}'kl} u_{\vv{p}} \\
            &\qquad\qquad\qquad\qquad\qquad\quad + a_{\vv{p}'kl}\dagger \commutator{a_{\vv{p}ij}\dagger}{\, a_{\vv{p}'kl}} u_{\vv{p}}^* \\
            &\qquad\qquad\qquad\qquad\qquad\quad + \cancelto{0}{\commutator{a_{\vv{p}ij}\dagger}{\, a_{\vv{p}'kl}\dagger}} a_{\vv{p}'kl} u_{\vv{p}}^* \bigg) \\
        &= \sum_{k,l=1}^N \int \frac{\dd^3\vv{p} \, \dd^3\vv{p}'}{(2\pi)^6} \omega_{\vv{p}'} \Big( (2\pi)^3 \delta^3(\vv{p}-\vv{p}')\delta_{ik}\delta_{jl} a_{\vv{p}'kl} u_{\vv{p}} \\
            &\qquad\qquad\qquad\qquad\qquad\quad - a_{\vv{p}'kl}\dagger (2\pi)^3 \delta^3(\vv{p}-\vv{p}')\delta_{ik}\delta_{jl} u_{\vv{p}}^* \Big) \\
        &= \int \frac{\dd^3\vv{p} \, \dd^3\vv{p}'}{(2\pi)^3} \omega_{\vv{p}'} \Big( \delta^3(\vv{p}-\vv{p}') a_{\vv{p}'ij} u_{\vv{p}} - a_{\vv{p}'ij}\dagger \delta^3(\vv{p}-\vv{p}') u_{\vv{p}}^* \Big) \\
        &= \int \frac{\dd^3\vv{p}}{(2\pi)^3} \omega_{\vv{p}} \Big( a_{\vv{p}ij} u_{\vv{p}} - a_{\vv{p}ij}\dagger u_{\vv{p}}^* \Big) \\
        &= \id \int \frac{\dd^3\vv{p}}{(2\pi)^3} \omega_{\vv{p}} \Big( a_{\vv{p}ij} u_{\vv{p}} - a_{\vv{p}ij}\dagger u_{\vv{p}}^* \Big) \\
        &= (-i^2) \int \frac{\dd^3\vv{p}}{(2\pi)^3} \omega_{\vv{p}} \Big( a_{\vv{p}ij} u_{\vv{p}} - a_{\vv{p}ij}\dagger u_{\vv{p}}^* \Big) \\
        &= i \int \frac{\dd^3\vv{p}}{(2\pi)^3} \Big( a_{\vv{p}ij} \left\{-i\omega_{\vv{p}}\right\} u_{\vv{p}} + a_{\vv{p}ij}\dagger \left\{i\omega_{\vv{p}}\right\} u_{\vv{p}}^* \Big) \\
        &= i \int \frac{\dd^3\vv{p}}{(2\pi)^3} \Big( a_{\vv{p}ij} \pdif{u_{\vv{p}}}{t} + a_{\vv{p}ij}\dagger \pdif{u_{\vv{p}}^*}{t} \Big) \\
        &= i \pdif{}{t} \left( \int \frac{\dd^3\vv{p}}{(2\pi)^3} \Big( a_{\vv{p}ij} u_{\vv{p}} + a_{\vv{p}ij}\dagger u_{\vv{p}}^* \Big) \right) \\
        &= i \pdif{\phi_{ij}}{t} \: ,
\end{split}
\end{align}
hvor Leibniz's integrationsregel (se \cref{footnote:LeibnitzIntegrationRule} på \cpageref{footnote:LeibnitzIntegrationRule}) igen er brugt for at flytte den partielle differentiation udenfor integralet, og de bosoniske kommutatorrelationer fundet i \textbf{9)}, \cref{eq:Opg2_A9_BosonicCommutatationRelations}, er benyttet.

Af \cref{eq:Opg2_A11_FieldOperatorObeyingHeisenbergEquation} kan det altså ses, at feltudvidelsen i \cref{eq:Opg2_Q6_ResultingField} overholder Heisenbergs bevægelsesligning. Dette betyder, at denne feltoperator er den korrekte at benytte i vekselvirkningsbilledet, da det gør sig gældende, at en sådan operator skal overholde Heisenbergs bevægelsesligning,
\begin{align}
    i \pdif{A_I}{t} &= \commutator{A_I}{\, H_0} \: ,
\end{align}
hvor $A_I$ er en operator i vekselvirkningsbilledet, og $H_0$ er den frie Hamiltonfunktion, \cref{eq:Opg2_Q10_HamiltonianWithCreationAndAnnihilationOperators}, hvilket også er den, som vi har benyttet i \cref{eq:Opg2_A11_FieldOperatorObeyingHeisenbergEquation}, siden vi endnu ikke har medtaget vekselvirkningsled i denne.


%%%%%%%%%%%%%%%%%%%%%%%%%

\paragraph[12) Lagrangedensitet invariant under $O(N)^2$ transformation]{\textbf{12)}}

% By using the property of the orthogonal, real matrices that RTabRbc = RbaRbd = δad, transforming the Lagrangian in the suggested way gives


De ortogonale og reelle matricer $R \in O(N)$ har egenskaben $R^\intercal R = \id$, altså $R_{ji}^\intercal R_{ik} = R_{ij}R_{ik} = \delta_{jk}$, hvor $R_{ji}^\intercal = R_{ij}$ er den $j,i$'te komponent af den transponerede matrix $R^\intercal$.

Benytter vi nu transformationen af $\phi_{ij}$, \cref{eq:Opg2_Q12_TransformationOfField}, til at transformere Lagrangedensiteten fra \cref{eq:Opg2_Lagrangian}, fås
\begin{align}
\begin{split}
    \L' &= \inv{2} \partial_\mu \phi'_{ij} \partial^\mu \phi'_{ji} - \inv{2} m^2 \phi'_{ij} \phi'_{ji} \\
    &= \inv{2} \partial_\mu \big( R_{ik} R_{jl} \phi_{kl} \big) \partial^\mu \big( R_{jm} R_{in} \phi_{mn} \big) - \inv{2} m^2 \big( R_{ik} R_{jl} \phi_{kl} \big) \big( R_{jm} R_{in} \phi_{mn} \big) \\
    &= R_{ik} R_{jl} R_{jm} R_{in} \inv{2} \big( \partial_\mu \phi_{kl} \partial^\mu \phi_{mn} - m^2 \phi_{kl} \phi_{mn} \big) \\
    &= \delta_{kn} \delta_{lm} \inv{2} \big( \partial_\mu \phi_{kl} \partial^\mu \phi_{mn} - m^2 \phi_{kl} \phi_{mn} \big) \\
    &= \inv{2} \partial_\mu \phi_{nm} \partial^\mu \phi_{mn} - \inv{2} m^2 \phi_{nm} \phi_{mn} \\
    &= \L \: ,
\end{split}
\end{align}
hvorved Lagrangedensiteten i \cref{eq:Opg2_Lagrangian} er invariant under denne transformation (\cref{eq:Opg2_Q12_TransformationOfField}) og dermed udviser $O(N)^2$-symmetri.


%%%%%%%%%%%%%%%%%%%%%%%%%

\paragraph[13) Vekselvirkning med og uden $O(N)^2$ symmetri]{\textbf{13)}}

Først findes et vekselvirkningsled, som \textit{bevarer} $O(N)^2$-symmetrien:

Det bemærkes, at når alle indeks fremgår i et lige antal felter og de alle summeres over, da vil de ortogonale $N \times N$ matricer gå ud med hinanden grundet deres egenskab om $R^\intercal R = \id$ og ledet vil dermed være invariant under $O(N)^2$-symmetri. Et eksempel på et $O(N)^2$-invariant led er $\L_I = g \phi_{ab}\phi_{bc}\phi_{ca}$, hvor $g$ er en koblingskonstrant, og dette vekselvirkningsled transformerer som
\begin{align} \label{eq:Opg2_A13_InteractionLagrangianNotBreakingSymmetry}
\begin{split}
    \L_I' &= g \phi'_{ab}\phi'_{bc}\phi'_{ca} \\
        &= g \big( R_{ae} R_{bf} \phi_{ef} \big) \big( R_{bg} R_{ch} \phi_{gh} \big) \big( R_{ci} R_{aj} \phi_{ij} \big) \\
        &= g R_{ae} R_{bf} R_{bg} R_{ch} R_{ci} R_{aj} \phi_{ef} \phi_{gh} \phi_{ij} \\
        &= g \delta_{ej} \delta_{fg} \delta_{hi} \phi_{ef} \phi_{gh} \phi_{ij} \\
        &= g \phi_{jg} \phi_{gi} \phi_{ij} \\
        &= \L_I \: ,
\end{split}
\end{align}
hvor summeringen er implicit. Dette vekselvirkningsled er dermed invariant.
\\

Som det næste findes et vekselvirkningsled, som \textit{bryder} $O(N)^2$-symmetrien:

Hvis ikke alle indeks opstår i et lige antal felter og dermed ikke summeres over, da vil der være nogle ortogonale matricer, der ikke vil gå ud med hinanden, og derved vil ledet ikke være generelt invariant under $O(N)^2$-symmetri. F.eks. kan vi tage eksemplet fra før og skifte sidste felts indeks, så $\L_I = g \phi_{ab}\phi_{bc}\phi_{cd}$ og $g$ stadig er en koblingskonstrant. Dette vekselvirkningsled transformerer nu som
\begin{align} \label{eq:Opg2_A13_InteractionLagrangianBreakingSymmetry}
\begin{split}
    \L_I' &= \sum_a \sum_d g \phi'_{ab}\phi'_{bc}\phi'_{cd} \\
        &= \sum_a \sum_d g \big( R_{ae} R_{bf} \phi_{ef} \big) \big( R_{bg} R_{ch} \phi_{gh} \big) \big( R_{ci} R_{dj} \phi_{ij} \big) \\
        &= \sum_a \sum_d g R_{ae} R_{bf} R_{bg} R_{ch} R_{ci} R_{dj} \phi_{ef} \phi_{gh} \phi_{ij} \\
        &= \sum_a \sum_d g R_{ae} \delta_{fg} \delta_{hi} R_{dj} \phi_{ef} \phi_{gh} \phi_{ij} \\
        &= \sum_a \sum_d g R_{ae} R_{dj} \phi_{eg} \phi_{gi} \phi_{ij} \\
        &\ne \L_I \: .
\end{split}
\end{align}
hvor nogle af summerne (ligesom i \cref{eq:Opg2_A13_InteractionLagrangianNotBreakingSymmetry}) er implicitte, mens to her er eksplicit udskrevet, da de er over de indeks, som gør at de ortogonale matricer ikke går ud med hinanden, men som skal summeres over, idet at matricerne står skrevet på deres elementform. De summeres altså ikke over ''i takt'', men det kan i stedet ses som en summering over ''forskellige rum''. Det overordnede er, at $R_{ae}R_{dj}$ \textit{ikke} kan gå ud med hinanden ved egenskaben $R^\intercal R = 1$, og dermed bevarer denne vekselvirkning \textit{ikke} $O(N)^2$-symmetrien.
\\

Af \cref{eq:Opg2_A13_InteractionLagrangianBreakingSymmetry} kan det tydeligt ses, at de to indekser, som ikke summeres over, er de to indekser, som er i den endelige rang-2-tensor, som transformerer ved to rotationsmatricer. $O(N)^2$-symmetrien kan dermed ikke bevares, hvis ikke alle indeks i alle felterne bliver summeret over, altså må der ikke eksistere et indeks i et felt, som ikke også eksisterer i et andet. Grundet transformationen af feltet (\cref{eq:Opg2_Q12_TransformationOfField}) kan et ikke-invariant led også konstrueres ved at multiplicere $\phi_{ii}$ på et ellers invariant led.


%%%%%%%%%%%%%%%%%%%%%%%%%

\paragraph[14) Generatoren af $O(N)$]{\textbf{14)}}

Vi betragter et element $R \in O(N)$, som er meget tæt på identitetsmatricen, $R = \id + A$, hvor $A$ er en $N \times N$-matrix med elementer $\abs{A_{ij}} \ll 1$. For et sådan $R$ kan det vises, at $A$ er en antisymmetrisk matrix, altså $A^\intercal = -A$:\\
\begin{align}
\begin{split}
    \id &= R^\intercal R \\
        &= (\id + A)^\intercal (\id + A) \\
        &= \left( \id^\intercal + A^\intercal \right) (\id + A) \\
        &= \left( \id + A^\intercal \right) (\id + A) \\
        &= \id \cdot \id + \id A + A^\intercal \id + A^\intercal A \\
        &= \id + A + A^\intercal + A^\intercal A \\
        &\simeq \id + A + A^\intercal \\
    \Rightarrow 0 &= A + A^\intercal \\
    \Rightarrow A^\intercal &= -A \: ,
\end{split}
\end{align}
hvor det er blevet benyttet, at transponering er lineær, altså $(B+C)^\intercal = B^\intercal + C^\intercal$ for to kvadratiske matricer $B$ og $C$, samt at $\abs{A_{ij}} \ll 1 \Rightarrow \abs{A_{ij}A_{ji}} \ll 1$ så $A^\intercal A$ dermed er så lille, at den negligeres.

Det er altså vist, at $A$ er en antisymmetrisk matrix. For en asymmetrisk matrix gør det sig gældende, at den skal være sporløs (eng: traceless), altså at $A_{ii} = 0$ (siden $A_{ij} = -A_{ji} \Rightarrow A_{ii} = -A_{ii}$, hvilket kun $0$ opfylder) og den øvre trekant af elementer (superdiagonalen og over) kan blive unikt bestemt ud fra den nedre trekants elementer (subdiagonalen og under) eller omvendt. Dermed er dimensionen af $A$, hvilket er frihedsgraderne for matricen, være
\begin{align} \label{eq:Opg2_A14_DimensionOfA}
    \mathrm{dim}\big(O(N)\big) &= \mathrm{dim}(A) = \frac{N^2}{2} - \frac{N}{2} = \frac{N(N-1)}{2} \: ,
\end{align}
hvor $N^2/2$ er halvdelen af matricen hvorfra vi så fratrækker den resterende del af diagonalens elementer $N/2$.


%%%%%%%%%%%%%%%%%%%%%%%%%

\paragraph[15) Noetherstrøm associeret med $O(N)^2$ symmetri]{\textbf{15)}}

Da dimensionen af $A$ er fundet til at være $N(N-1)/2$, \cref{eq:Opg2_A14_DimensionOfA} i opgave \textbf{14)}, betragter vi nu $N(N-1)/2$ matricer $A^k$ som udspænder rummet af alle antisymmetriske matricer med dimension $N$. Vi definerer nu $A = \sum_k \epsilon^k A^k = \epsilon^k A^k$ (og lader altså summeringen være implicit), hvorved vi nu kan skrive infinitisimaltransformationen $R$ som
\begin{align}
    R &= \id + A = \id + \epsilon^k A^k \: .
\end{align}
Beregner vi nu variationen af $\phi_{ij}$ får vi
\begin{align}
\begin{split}
    \delta \phi_{ij} &= R_{il} R_{jn} \phi_{ln} - \phi_{ij} \\
        &= \left( \id_{il} + \epsilon^k A_{il}^k \right) \left( \id_{jn} + \epsilon^m A_{jn}^m \right) \phi_{ln} - \phi_{ij} \\
        &\simeq \left( \delta_{il} \delta_{jn} + \epsilon^k A_{il}^k \delta_{jn} + \epsilon^m A_{jn}^m \delta_{il} \right) \phi_{ln} - \phi_{ij} \\
        &= \left( \epsilon^k A_{il}^k \delta_{jn} + \epsilon^m A_{jn}^m \delta_{il} \right) \phi_{ln} \: ,
\end{split}
\end{align}
hvor vi har benyttet, at $\abs{A_{ij}} \ll 1$ samt at identitetsmatricens indgange er givet som $\id_{ij} = \delta_{ij}$. Ved brug af linearitet kan det vises, at
\begin{align}
    \partial_\mu ( \delta \phi_{ij} ) &= \delta ( \partial_\mu \phi_{ij} ) \: .
\end{align}
Varierer vi nu Lagrangedensiteten fås\footnote{
    Her er det brugt at $\dd f = \dif{f}{x} \dd x + \dif{f}{y} \dd y$.
}
\begin{align} \label{eq:Opg2_A15_deltaL}
    \delta \L &= \pdif{\L}{\phi_{ij}} \delta \phi_{ij} + \pdif{\L}{\left( \partial_\mu \phi_{ij} \right)} \delta ( \partial_\mu \phi_{ij} ) \: ,
\end{align}
og sætter vi nu lig $0$, benytter Euler-Lagrangeligningen i første led i \cref{eq:Opg2_A15_deltaL}, samt bruger Lagrangedensitetesns symmetri, så finder vi
\begin{align}
\begin{split}
    0 &= \delta \L \\
        &= \pdif{\L}{\phi_{ij}} \delta \phi_{ij} + \pdif{\L}{\left( \partial_\mu \phi_{ij} \right)} \delta ( \partial_\mu \phi_{ij} ) \\
        &= \partial_\mu \left(\pdif{\L}{(\partial_\mu \phi_{ij})}\right) \delta \phi_{ij} + \pdif{\L}{\left( \partial_\mu \phi_{ij} \right)} \delta ( \partial_\mu \phi_{ij} ) \\
        &= \partial_\mu \left(\pdif{\L}{(\partial_\mu \phi_{ij})}\right) \delta \phi_{ij} + \pdif{\L}{\left( \partial_\mu \phi_{ij} \right)} \partial_\mu (\delta \phi_{ij} ) \\
        &= \partial_\mu \left( \pdif{\L}{\phi_{ij}} \delta \phi_{ij} \right) \: ,
\end{split}
\end{align}
hvor vi har benyttet kædereglen i sidste lighed. Derved kan det ses, at
\begin{align}
    \pdif{\L}{\phi_{ij}} \delta \phi_{ij} &= \pdif{\L}{\phi_{ij}} \left( \epsilon^k A_{il}^k \delta_{jn} + \epsilon^m A_{jn}^m \delta_{il} \right) \phi_{ln}
\end{align}
er bevaret uanset valget af $\epsilon^i$'erne. Af dette kan vi kan vi finde den bevarede strøm som
\begin{align} \label{eq:Opg2_A15_ConservedNoetherCurrent_ComponentForm}
    j_k^\mu &= \pdif{\L}{\phi_{ij}} \left( \epsilon^k A_{il}^k \delta_{jn} + \epsilon^m A_{jn}^m \delta_{il} \right) \phi_{ln}
        = \inv{2} \left( \epsilon^k A_{il}^k \delta_{jn} + \epsilon^m A_{jn}^m \delta_{il} \right) \phi_{ln} \partial^\mu \phi_{ij} \: ,
\end{align}
hvor $\partial_\mu j_k^\mu = 0$ for all $k$, hvor sidste lighed har gjort brug af symmetrien af matrixfeltet, $\phi_{ij} = \phi_{ji}$.

Omskriver vi nu det hele til matricer i stedet for komponenter, får vi fra \cref{eq:Opg2_A15_ConservedNoetherCurrent_ComponentForm} at
\begin{align} \label{eq:Opg2_A15_ConservedNoetherCurrent_MatrixForm}
    j_k^\mu &= \mathrm{tr}\left( \pdif{\L}{\left( \partial_\mu \Phi \right)} \left[ A^k \Phi - \Phi A^k \right] \right)
        = \inv{2} \mathrm{tr}\left( \partial^\mu \Phi \left[ A^k \Phi - \Phi A^k \right] \right)
\end{align}
når vi igen benytter symmetrien af matrixfeltet, samt at $A$ er asymmetrisk, og lader $\Phi$ betegne matricen med komponenterne $\phi_{ij}$.


%%%%%%%%%%%%%%%%%%%%%%%%%

\paragraph[16) Ændringer hvis $\phi_{ij}$ komplekst i stedet for reelt]{\textbf{16)}}

% Hvad ville ændre sig, hvis feltet $\phi_{ij}$ var komplekst i stedet for reelt? Beskriv blot de overordnede forskellige; du skal \emph{ikke} beregne hele opgaven igen med et komplekst felt.

Gør vi feltet komplekst, da vil vi få det dobbelte antal frihedsgrader, da et komplekst felt har én frihedsgrad for dens reelle del og én fra dens imaginære del. For det komplekse felt skal Lagrangedensiteten også ændres, således at
\begin{align}
    \L &= \inv{2} \partial_\mu \phi_{ij}^* \partial^\mu \phi_{ji} - \inv{2} m^2 \phi_{ij}^* \phi_{ji} \: ,
\end{align}
og vi modificerer symmetrien af matrixfeltet til at $\phi_{ij} = \phi_{ji}^*$. Udover bevægelsesligningerne for $\phi_{ij}$ får vi også bevægelsesligninger for $\phi_{ij}^*$ grundet de nye frihedsgrader, så
\begin{align}
    0 &= (\partial_\mu \partial^\mu + m^2) \phi_{ij}^* \quad \forall i,j = 1,\ldots,N \: .
\end{align}
Da feltet er komplekst, da bliver det konjugerede felt modificeret så
\begin{align}
    \pi_{ij} &= \partial^0 \phi_{ij}^*
        \qquad \text{og} \qquad
    \pi_{ij}^* = \partial^0 \phi_{ij} \: ,
\end{align}
hvorved vi også får en ændring i Hamiltonfunktionen, da der deri også skal kompenseres med komplekskonjugeringer
\begin{align}
    H &= \int \dd^3\vv{x} \left[ \inv{2} \pi_{ij}^* \pi_{ji} + \inv{2} \Grad{\phi_{ij}} \cdot \Grad{\phi_{ji}^*} + \inv{2} m^2 \phi_{ij}^* \phi_{ji} \right] \: .
\end{align}
Yderligere vil der ske ændringer i feltudvidelsen for $\phi_{ij}$, nemlig at der vil være forskellige kreations- og annihilationsoperatorer, $a\dagger$ og $c\dagger$ for hhv. partikler og antipartikler, da et komplekst felt betyder, at partiklen ikke længere er sin egen antipartikel
\begin{align}
    \phi_{ij} &= \int \frac{\dd^3\vv{p}}{(2\pi)^3} \invsqrt{2\omega_{\vv{p}}} \left(a_{\vv{p}ij} \bexp{-ipx} + c_{\vv{p}ij}\dagger \bexp{ipx} \right) \: ,
\end{align}
og vi skal dermed også introducere dens Hermitiskkonjugerede, da vi nu har kvantiseret vores felt,
\begin{align}
    \phi_{ij}\dagger &= \int \frac{\dd^3\vv{p}}{(2\pi)^3} \invsqrt{2\omega_{\vv{p}}} \left(a_{\vv{p}ij}\dagger \bexp{ipx} + c_{\vv{p}ij} \bexp{-ipx} \right) \: .
\end{align}
Beregningerne i \textbf{8)} og \textbf{9)} kan gentages for $c_{\vv{p},ij}$, og vi vil komme frem til samme kommutatorrelationer blot med udskiftningen $a_{\vv{p},ij} \rightarrow c_{\vv{p},ij}$, så vores kommutatorrelationer bliver
\begin{subequations}
\begin{align}
    \commutator{a_{\vv{p}'kl}}{\, a_{\vv{p}ij}\dagger} &= (2\pi)^3 \delta^3(\vv{p} - \vv{p}') \delta_{ik} \delta_{jl} \\
    \commutator{c_{\vv{p}'kl}}{\, c_{\vv{p}ij}\dagger} &= (2\pi)^3 \delta^3(\vv{p} - \vv{p}') \delta_{ik} \delta_{jl} \\
    0 &= \commutator{a_{\vv{p}ij}}{\, a_{\vv{p}'kl}}
        = \commutator{c_{\vv{p}ij}}{\, c_{\vv{p}'kl}}
        = \commutator{a_{\vv{p}ij}}{\, c_{\vv{p}'kl}}
        = \commutator{a_{\vv{p}ij}}{\, c_{\vv{p}'kl}\dagger} \: ,
\end{align}
\end{subequations}
og Hermitiskkonjugeringer af disse. Grundet denne tilføjelse af $c_{\vv{p}ij}$ vil vi i den kvantiserede Hamiltonfunktionen også få en tælleoperator for disse antipartikler, så
\begin{align}
    H &= \sum_{i,j=1}^N \int \frac{\dd^3\vv{p}}{(2\pi)^3} \left[ \omega_{\vv{p}} a_{\vv{p}ij}\dagger a_{\vv{p}ij} + c_{\vv{p}ij}\dagger c_{\vv{p}ij} \right] \: .
\end{align}
Når det kommer til spørgsmålet symmetrier med den ortogonale gruppe $O(N)$ i opgave \textbf{12)}, så vil vi i stedet for disse reelle matricer udvide symmetrien til gruppen af $N \times N$ unitære matricer, da disse også kan være komplekse, men stadig samme egenskab, nemlig
\begin{align}
    U(N) &= \{ R \:\vert\: R\dagger R = \id \} \: ,
\end{align}
hvor komplekskonjugeringen er taget i betragtning i den Hermitiske konjugering, da denne er svarende til komplekskonjugering og transponering, og de samme regler for invarians under denne symmetri, som i \textbf{13)}, vil gøre sig gældende, dog når der igen er taget i betragtning, at vi skal have medtaget $\phi_{ij}$ værende kompleks. Noetherstrømmen fundet i \textbf{15)} vil også ændres, idet den blev beregnet til at være afhængig af felterne, hvilket efter kompleksificeringen af felterne vil sige, at den nu kommer til at afhænge af både felterne og deres komplekskonjugeringer, og det giver også mening, at disse strømme vil ændre sig som et resultat af, da det komplekse felt kan have ladning.
\\

De fleste af beregninger i dette tankeeksperiment er ikke specielt meget sværere end dem, som vi har lavet i denne opgave, og i løbet af kurset (problemsæt 2 \cite{problemSet2}) har vi udført beregningerne for det komplekse Klein-Gordonfelt (uden matricer som felter), og vi har i denne opgave set, at det reelle matrixfelts beregninger er analoge til dem for det reelle Klein-Gordonfelt, hvorfor det komplekse matrixfelts beregninger også formodes at være analog med dem for det komplekse Klein-Gordonfelt i \cite{problemSet2}.


%%%%%%%%%%%%%%%%%%%%%%%%%%%%%%%%%%%%%%%%%%%%%%%%%%%%%%%%%%%%%%%%%%%%%%%%%%%%%%%%%%%%%

\end{document}