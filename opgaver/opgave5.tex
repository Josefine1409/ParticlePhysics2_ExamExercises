\documentclass[../main.tex]{subfiles}

\begin{document}

%%%%%%%%%%%%%%%%%%%%%%%%%%%%%%%%%%%%%%%%%%%%%%%%%%%%%%%%%%%%%%%%%%%%%%%%%%%%%%%%%%%%%

\section{En model med to Higgsdubletter}

I denne opgave vil vi betragte en model, hvor der er to Higgsfelter med samme kvantetal som dem, vi normalt introducerer i Standardmodellen. Lagrangedensiteten er
\begin{align} \label{eq:Opg5_Lagrangian}
    \L &= \abs{\left( i\partial_\mu - g \vv{T} \cdot \vv{W}_\mu - g' \frac{1}{2} B_\mu\right) \Phi_1}^2 +
    \abs{\left( i\partial_\mu - g \vv{T} \cdot \vv{W}_\mu - g' \frac{1}{2} B_\mu\right) \Phi_2}^2 - V(\Phi_1,\Phi_2) \: ,
\end{align}
hvor både $\Phi_1$ og $\Phi_2$ er komplekse dubletter med hyperladning $Y=1$ som i Standardmodellen.


%%%%%%%%%%%%%%%%%%%%%%%%%%%%%%%%%%%%%%%%%%%%%%%%%%%%%%%%%%%%%%%%%%%%%%%%%%%%%%%%

\paragraph*{\textbf{1)}}

Antag at vi spontant ødelægger symmetrien for Higgspotentialet på en sådan måde, at vakuumet kan skrives som
\begin{align} \label{eq:Opg5_Q1_Vaccum}
    \Phi_1 &= \TwoRowMat{0}{\frac{v_1}{\sqrtTo}}
        \quad \text{og} \quad
    \Phi_2 = \TwoRowMat{0}{\frac{v_2}{\sqrtTo}} \: .
\end{align}
Find gaugebosonernes masser ved dette valg af vakuum.   


%%%%%%%%%%%%%%%%%%%%%%%%%

\paragraph*{\textbf{2)}}

Antag at Higgspotentialet i denne model er på formen
\begin{align} \label{eq:Opg5_Q2_HiggsPotential}
    V(\Phi_1,\Phi_2) &= -\mu_1^2 \Phi_1\dagger \Phi_1 - \mu_2^2 \Phi_{2}\dagger \Phi_2 + \lambda_1^2 \left( \Phi_{1}\dagger \Phi_1 \right)^2 + \lambda_2^2 \left( \Phi_2\dagger \Phi_2 \right)^2 + \lambda_3 \Phi_1\dagger \Phi_1 \Phi_2\dagger \Phi_2 \: .
\end{align}
Find betingelserne for konstanterne $\mu_i$ og $\lambda_i$ for at der er et lokalt minimum på formen brugt i \textbf{1)}.


%%%%%%%%%%%%%%%%%%%%%%%%%

\paragraph*{\textbf{3)}}

Når vi bruger gaugesymmetri i Standardmodeltilfældet med én Higgsdublet får vi et partikelspektrum med en enkelt ikkeladet Higgsboson. Betragt hvad der vil ske i tilfældet med to Higgsdubletter. Hvor mange frihedsgrader kan vi eliminere med gaugetransformationer og hvor mange er tilbage? Karakterisér alle ''Higgs''-partiklerne ud fra deres kvantetal (ladning).


%%%%%%%%%%%%%%%%%%%%%%%%%%%%%%%%%%%%%%%%%%%%%%%%%%%%%%%%%%%%%%%%%%%%%%%%%%%%%%%%

\subsection{Besvarelse}

%%%%%%%%%%%%%%%%%%%%%%%%%

\paragraph[1) Gaugebosoners masse ved valg af vakuumtilstande]{\textbf{1)}}




%%%%%%%%%%%%%%%%%%%%%%%%%

\paragraph[2) Betingelser for konstanterne $\mu_i$ og $\lambda_i$ i Higgspotential]{\textbf{2)}}




%%%%%%%%%%%%%%%%%%%%%%%%%

\paragraph[3) To Higgsdubletter vs. én Higgsdublet]{\textbf{3)}}

\ldots


%%%%%%%%%%%%%%%%%%%%%%%%%%%%%%%%%%%%%%%%%%%%%%%%%%%%%%%%%%%%%%%%%%%%%%%%%%%%%%%%%%%%%

\end{document}