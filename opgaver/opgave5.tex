\documentclass[../main.tex]{subfiles}

\begin{document}

%%%%%%%%%%%%%%%%%%%%%%%%%%%%%%%%%%%%%%%%%%%%%%%%%%%%%%%%%%%%%%%%%%%%%%%%%%%%%%%%%%%%%

\section{En model med to Higgsdubletter}

I denne opgave vil vi betragte en model, hvor der er to Higgsfelter med samme kvantetal som dem, vi normalt introducerer i Standardmodellen. Lagrangedensiteten er
\begin{align} \label{eq:Opg5_Lagrangian}
    \L &= \abs{\left( i\partial_\mu - g \vv{T} \cdot \vv{W}_\mu - g' \frac{1}{2} B_\mu\right) \Phi_1}^2 +
    \abs{\left( i\partial_\mu - g \vv{T} \cdot \vv{W}_\mu - g' \frac{1}{2} B_\mu\right) \Phi_2}^2 - V(\Phi_1,\Phi_2) \: ,
\end{align}
hvor både $\Phi_1$ og $\Phi_2$ er komplekse dubletter med hyperladning $Y=1$ som i Standardmodellen.


%%%%%%%%%%%%%%%%%%%%%%%%%%%%%%%%%%%%%%%%%%%%%%%%%%%%%%%%%%%%%%%%%%%%%%%%%%%%%%%%

\paragraph*{\textbf{1)}}

Antag at vi spontant ødelægger symmetrien for Higgspotentialet på en sådan måde, at vakuumet kan skrives som
\begin{align} \label{eq:Opg5_Q1_Vaccum}
    \Phi_1 &= \TwoRowMat{0}{\frac{v_1}{\sqrtTo}}
        \quad \text{og} \quad
    \Phi_2 = \TwoRowMat{0}{\frac{v_2}{\sqrtTo}} \: .
\end{align}

Find gaugebosonernes masser ved dette valg af vakuum.   


%%%%%%%%%%%%%%%%%%%%%%%%%

\paragraph*{\textbf{2)}}

Antag at Higgspotentialet i denne model er på formen
\begin{align} \label{eq:Opg5_Q2_HiggsPotential}
    V(\Phi_1,\Phi_2) &= -\mu_1^2 \Phi_1\dagger \Phi_1 - \mu_2^2 \Phi_{2}\dagger \Phi_2 + \lambda_1^2 \left( \Phi_{1}\dagger \Phi_1 \right)^2 + \lambda_2^2 \left( \Phi_2\dagger \Phi_2 \right)^2 + \lambda_3 \Phi_1\dagger \Phi_1 \Phi_2\dagger \Phi_2 \: .
\end{align}

Find betingelserne for konstanterne $\mu_i$ og $\lambda_i$ for at der er et lokalt minimum på formen brugt i \textbf{1)}.


%%%%%%%%%%%%%%%%%%%%%%%%%

\paragraph*{\textbf{3)}}

Når vi bruger gaugesymmetri i Standardmodeltilfældet med én Higgsdublet får vi et partikelspektrum med en enkelt ikke-ladet Higgsboson. Betragt hvad der vil ske i tilfældet med to Higgsdubletter. Hvor mange frihedsgrader kan vi eliminere med gaugetransformationer og hvor mange er tilbage? Karakterisér alle ''Higgs''-partiklerne ud fra deres kvantetal (ladning).


%%%%%%%%%%%%%%%%%%%%%%%%%%%%%%%%%%%%%%%%%%%%%%%%%%%%%%%%%%%%%%%%%%%%%%%%%%%%%%%%

\subsection{Besvarelse}

%%%%%%%%%%%%%%%%%%%%%%%%%

\paragraph[1) Gaugebosoners masse ved valg af vakuumtilstande]{\textbf{1)}}

For at finde gaugebosonernes masser ved valget af vakuum som i \cref{eq:Opg5_Q1_Vaccum}, så indsættes disse komplekse dubletter i Lagrangedensiteten i \cref{eq:Opg5_Lagrangian}, hvor vi kun medtager vakuumdelen, da differentialoperatorledene ikke medvirker til at finde masserne, samt vi udelader potentialet, da det heller ikke indeholder nogle led med bosoner

% To find the masses of the gauge bosons we put the fields into the Lagrangian, however we only need the vacuum part to identify the masses. We also omit the potential part of the Lagrangian since they do not contain any boson terms.

% Using the same procedure as the Standard Model Higgs note, inserting the two expressions for φ1 and φ2 into the Lagrangian and ignoring both the potential and the space-dependent i∂µ terms. Doing this gives

\begin{align} \label{eq:Opg5_A1_CalculationOfMassesPart1}
\begin{split}
    &\abs{\left( - g \vv{T} \cdot \vv{W}_\mu - \frac{g'}{2} B_\mu \right) \TwoRowMat{0}{\frac{v_1}{\sqrtTo}}}^2 + \abs{\left( - g \vv{T} \cdot \vv{W}_\mu - \frac{g'}{2} B_\mu \right) \TwoRowMat{0}{\frac{v_2}{\sqrtTo}}}^2 \\
        &\quad = \abs{\left( - g \frac{\vgv{\sigma}}{2} \cdot \vv{W}_\mu - \frac{g'}{2} B_\mu \right) \TwoRowMat{0}{\frac{v_1}{\sqrtTo}}}^2 + \abs{\left( - g \frac{\vgv{\sigma}}{2} \cdot \vv{W}_\mu - \frac{g'}{2} B_\mu \right) \TwoRowMat{0}{\frac{v_2}{\sqrtTo}}}^2 \\
        &\quad = \inv{8} \left\{ \abs{\left( - g \vgv{\sigma} \cdot \vv{W}_\mu - \frac{g'}{2} B_\mu \right) \TwoRowMat{0}{v_1}}^2 + \abs{\left( - g \vgv{\sigma} \cdot \vv{W}_\mu - \frac{g'}{2} B_\mu \right) \TwoRowMat{0}{v_2}}^2 \right\} \\
        &\quad = \inv{8} \Bigg\{ \abs{\TwoRowMat{g' B_\mu + g W_\mu^3 & g(W_\mu^1 - i W_\mu^2)}{g(W_\mu^1 + i W_\mu^2) & g' B_\mu - g W_\mu^3} \TwoRowMat{0}{v_1}}^2 \\
            &\qquad\qquad\qquad + \abs{\TwoRowMat{g' B_\mu + g W_\mu^3 & g(W_\mu^1 - i W_\mu^2)}{g(W_\mu^1 + i W_\mu^2) & g' B_\mu - g W_\mu^3} \TwoRowMat{0}{v_2}}^2 \Bigg\} \\
        &\quad = \inv{8} \left\{ \abs{\TwoRowMat{g(W_\mu^1 - i W_\mu^2)}{g' B_\mu - g W_\mu^3} v_1}^2 + \abs{\TwoRowMat{g(W_\mu^1 - i W_\mu^2)}{g' B_\mu - g W_\mu^3} v_2}^2 \right\} \\
        &\quad = \frac{v_1^2 + v_2^2}{8} \abs{\TwoRowMat{g(W_\mu^1 - i W_\mu^2)}{g' B_\mu - g W_\mu^3}}^2 \: ,
\end{split}
\end{align}
hvilket præcis er udtrykket fundet i \cite[lign. 27]{problemSetHiggsInStandardModel} med $v^2 \rightarrow v'^2 = v_1^2 + v_2^2$.

Feltet for en massiv spin-$1$-boson har en Lagrangedensitet på formen
\begin{align} \label{eq:Opg5_A1_LagrangianDensityOfMassivSpin1Boson}
    \L_{S=1,B} = - \frac{1}{4} F_{\mu\nu} F^{\mu\nu} + \frac{1}{2} M^2 A_\mu A^\mu \: .
\end{align}
Udskriver vi \cref{eq:Opg5_A1_CalculationOfMassesPart1} og sammenligner med Lagrangedensiteten i \cref{eq:Opg5_A1_LagrangianDensityOfMassivSpin1Boson}, hvor vi husker på, at vi kun har medtaget led, som giver masseledet, altså $\frac{1}{2} m^2 A_\mu A^\mu$, da får vi
\begin{align}
\begin{split}
    &\abs{\left( - g \vv{T} \cdot \vv{W}_\mu - \frac{g'}{2} B_\mu \right) \TwoRowMat{0}{\frac{v_1}{\sqrtTo}}}^2 + \abs{\left( - g \vv{T} \cdot \vv{W}_\mu - \frac{g'}{2} B_\mu \right) \TwoRowMat{0}{\frac{v_2}{\sqrtTo}}}^2 \\
        &\quad = \frac{v_1^2 + v_2^2}{8} \abs{\TwoRowMat{g(W_\mu^1 - i W_\mu^2)}{g' B_\mu - g W_\mu^3}}^2 \\
        &\quad = \frac{v_1^2 + v_2^2}{8} \left\{ g^2 \left[ W_\mu^1 \right]^2 + g^2 \left[ W_\mu^2 \right]^2 + \left[ g' B_\mu - g W_\mu^3 \right]^2 \right\} \\
        &\quad = \inv{2} \left( \frac{g\sqrt{v_1^2 + v_2^2}}{2} \right)^2 \left( \left[ W_\mu^1 \right]^2 + \left[ W_\mu^2 \right]^2 \right) \\
            &\qquad\qquad\qquad + \inv{2} \left( \frac{\sqrt{g^2 + g'^2} \sqrt{v_1^2 + v_2^2}}{2} \right)^2 \left( \frac{g' B_\mu - g W_\mu^3}{\sqrt{g^2 + g'^2}} \right)^2 \\
        &\quad = m_{W^\pm}^2 W_\mu^+ \left(W^-\right)^\mu + \inv{2} m_Z^2 Z_\mu Z^\mu \: ,
\end{split}
\end{align}
hvor vi har benyttet, at $W_\mu^\pm = (W_\mu^1 \mp i W_\mu^2)/\sqrtTo$ (\cite[opg. 4]{problemSetHiggsInStandardModel}) og $Z_\mu = ???$. Dermed er gaugebosonernes masser
\begin{align}
    m_{W^\pm} &= \frac{g\sqrt{v_1^2 + v_2^2}}{2}
        \qquad \text{og} \qquad
    m_Z = \frac{\sqrt{g^2 + g'^2} \sqrt{v_1^2 + v_2^2}}{2} \: ,
\end{align}
hvor 
hvilket er samme masser som fundet i \cite[opg. 17]{problemSetHiggsInStandardModel}, hvor vi blot har ladet $v \rightarrow v' = \sqrt{v_1^2 + v_2^2}$.


%%%%%%%%%%%%%%%%%%%%%%%%%

\paragraph[2) Betingelser for konstanterne $\mu_i$ og $\lambda_i$ i Higgspotential]{\textbf{2)}}

Higgspotentialet indeholder kun felterne i sammensætningen $\Phi_i\dagger \Phi_i$ som er positiv og reel. For at lette notationen defineres derfor $x_i = \abs{\Phi_i}^2$, hvormed potentialet i \cref{eq:Opg5_Q2_HiggsPotential} bliver
\begin{align}
    V(x_1, x_2) &= -\mu_1^2 x_1 - \mu_2^2 x_2 + \lambda_1^2 x_1^2 + \lambda_2^2 x_2^2 + \lambda_3 x_1 x_2 \: .
\end{align}

Først finder vi minimum ved at differentierer potentialet med hensyn til hvert felt og sætter dette lig nul
\begin{align} \label{eq:Opg5_A2_VDifferentiatedOnce}
\begin{split}
    0 &= \pdif{V}{x_1} = - \mu_1^2 + 2 \lambda_1^2 x_1 + \lambda_3 x_2
        \qquad \text{og} \\
    0 &= \pdif{V}{x_2} = - \mu_2^2 + 2 \lambda_2^2 x_2 + \lambda_3 x_1 \: ,
\end{split}
\end{align}
fra hvilket $x_1$ og $x_2$ kan findes som
\begin{align}
\begin{split}
    x_1 &= \frac{2\lambda_2^2 \mu_1^2 - \lambda_3 \mu_2^2}{4 \lambda_1^2 \lambda_2^2 - \lambda_3^2}
        \qquad \text{og} \\
    x_2 &= \frac{2\lambda_1^2 \mu_2^2 - \lambda_3 \mu_1^2}{4 \lambda_1^2 \lambda_2^2 - \lambda_3^2}
\end{split}
\end{align}
under antagelse af at $4 \lambda_1^2 \lambda_2^2 - \lambda_3^2 \ne 0$ og $\lambda_2 \ne 0$.

\ldots
\begin{align} \label{eq:Opg5_A2_VDifferentiatedTwice}
\begin{split}
    0 &< \pdif[2]{V}{x_1} = 2 \lambda_1^2
        \qquad \text{og} \\
    0 &< \pdif[2]{V}{x_2} = 2 \lambda_2^2 \: ,
\end{split}
\end{align}

\ldots !!! DISKUTER DENNE OPGAVE MED ANDRE !!! \ldots


%%%%%%%%%%%%%%%%%%%%%%%%%

\paragraph[3) To Higgsdubletter vs. én Higgsdublet]{\textbf{3)}}

Den mest generelle form at de to Higgsdoubletter, når de er korrekt normaliserede (se \cite[lign. 25]{problemSetHiggsInStandardModel}), er
\begin{align}
    \Phi_1 &= \TwoRowMat{\phi_1^R + i \phi_1^I}{\phi_2^R + i \phi_2^I}
        \qquad \text{og} \qquad
    \Phi_2 = \TwoRowMat{\phi_3^R + i \phi_3^I}{\phi_4^R + i \phi_4^I} \: ,
\end{align}
altså er der 8 frihedsgrader tilsammen i de to Higgsdubletter, hvilket giver god mening, da en dublet består af to komplekse skalarfelter, som hver har to frihedsgrader, da de er komplekse, så vi har i alt $2 \cdot 2 \cdot 2 = 8$ frihedsgrader.

Den generelle gaugesymmetri for Higgsfeltet er $\exp(i[\vv{T}\cdot\vgv{\alpha}(x) + \beta])$, hvilken har fire frihedsgrader da $\vgv{\alpha}(x)$ har tre frihedsgrader og $\beta$ én. Vi kan, som i \cite[lign. 26]{problemSetHiggsInStandardModel}, skrive Higgsdubletterne ved denne gaugesymmetri
\begin{align} \label{eq:Opg5_A3_HiggsDoubletsWithTheirGaugeSymmetry}
    \Phi_1 &= \pexp{i\vv{T}\cdot\vgv{\alpha}_1(x)} \TwoRowMat{0}{\frac{v_1 + H_1(x)}{\sqrtTo}}
        \qquad \text{og} \qquad
    \Phi_2 = \pexp{i\vv{T}\cdot\vgv{\alpha}_2(x)} \TwoRowMat{0}{\frac{v_2 + H_2(x)}{\sqrtTo}} \: ,
\end{align}
idet vi stadig ønsker at efterlade én af frihedsgraderne i gaugesymmetrien ubrudt for det elektromagnetiske felt. Dog skal de to Higgsdubletter være i samme gauge, hvorfor vi kun kan gaugetransformere tre frihedsgrader væk, så
\begin{align} \label{eq:Opg5_A3_HiggsDoubletsAfterGaugeTransformation}
    \Phi_1 &= \TwoRowMat{0}{\frac{v_1 + H_1(x)}{\sqrtTo}}
        \qquad \text{og} \qquad
    \Phi_2 = \invsqrtTo \TwoRowMat{H_2(x) + i H_3(x)}{H_4(x) + i H_5(x)} \: .
\end{align}
Der er altså fem Higgsfelter, og dermed fem frihedsgrader, tilbage.
\\

Som det næste skal Higgsfelterne karakteriseres ud fra deres kvantetal. Vi ved at begge Higgsdubletter har hyperladning $Y=1$


%%%%%%%%%%%%%%%%%%%%%%%%%%%%%%%%%%%%%%%%%%%%%%%%%%%%%%%%%%%%%%%%%%%%%%%%%%%%%%%%%%%%%

\end{document}