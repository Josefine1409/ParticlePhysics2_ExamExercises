\documentclass[../main.tex]{subfiles}

\begin{document}

%%%%%%%%%%%%%%%%%%%%%%%%%%%%%%%%%%%%%%%%%%%%%%%%%%%%%%%%%%%%%%%%%%%%%%%%%%%%%%%%%%%%%

\section{Den lineære sigmamodel}

Betragt den følgende Lagrangedensitet
\begin{align}
    \L &= \inv{2} \left( \partial_\mu \sigma \right) \left( \partial^\mu \sigma \right) + \inv{2} \left( \partial_\mu \pi \right) \left( \partial^\mu \pi \right) - \inv{2} m^2 \sigma^2 - \lambda v \sigma^3 - \lambda v \sigma \pi^2 - \inv{4} \lambda \left( \sigma^2 + \pi^2 \right)^2 \: ,
\end{align}
hvor $\sigma$ og $\pi$ er skalarfelter og $v^2 = m^2/(2\lambda)$.


%%%%%%%%%%%%%%%%%%%%%%%%%%%%%%%%%%%%%%%%%%%%%%%%%%%%%%%%%%%%%%%%%%%%%%%%%%%%%%%%

\paragraph*{\textbf{1)}}

Find dimensionen af alle felter og koblingskonstanter i $\L$ i enheder, når $\hbar = c = 1$.


%%%%%%%%%%%%%%%%%%%%%%%%%

\paragraph*{\textbf{2)}}

Ved at kigge på $\L$, hvad vil du så mene, at massen af $\sigma$ og $\pi$ er?


%%%%%%%%%%%%%%%%%%%%%%%%%

\paragraph*{\textbf{3)}}

Tegn alle de lovlige vekselvirkningsvertices for denne Lagrangedensitet og bestem deres tilsvarende koblingskonstanter.


%%%%%%%%%%%%%%%%%%%%%%%%%

\paragraph*{\textbf{4)}}

Tegn alle loopkorrigeringerne til $\pi$-propagatoren. Du skal finde fem forskellige slags korrektioner.


%%%%%%%%%%%%%%%%%%%%%%%%%

\paragraph*{\textbf{5)}}

Betragt diagrammet hvor der er én $\pi$ og én $\sigma$, som løber i loopet (eng: running 
around the loop). Dette tillades af $\pi\pi\sigma$-ledet for andenordensperturbation. Vis at diagrammets amplitude er
\begin{align}
    4 \lambda^2 v^2 \int \frac{\dd^4 k}{(2\pi)^4} \inv{k^2 - m^2} \inv{(k + p)^2} \: .
\end{align}
Er amplituden endelig eller uendelig, og hvis den er uendelig, hvordan divergerer den så for store $k$?\\
(Hint: Faktoren af $4$ er den såkaldte symmetrifaktor. For at udlede den, tænk da på det mulige antal måder, som man kan parre de eksterne $\pi_\textrm{in}$-  og $\pi_\textrm{out}$-felter med operatorerne i matrixelementet $\pi\pi\sigma\pi\pi\sigma$. Du kan også gøre dette ved at brute force gennem udvidelsen af Dysonserien, men at tænke sig til udledningen er noget mere elegant.)


%%%%%%%%%%%%%%%%%%%%%%%%%

\paragraph*{\textbf{6)}}

Nedskriv amplituderne for resten af loopkorrigeringerne til $\pi$-propagatoren. Husk symmetrifaktorerne; de er vigtige!


%%%%%%%%%%%%%%%%%%%%%%%%%

\paragraph*{\textbf{7)}}

Betragt nu grænsen for nul-firimpuls for $\pi$ ($p_\mu = 0$) i propagatoren. Vis at når du adderer alle korrektionerne for amplituden for $p_\mu = 0$, så får man $0$! (Hint: Der er en god idé at nedbryde (eng: decompose) integranten i amplituden i \textbf{5)} til partielle brøker (eng: partial fractions).)


%%%%%%%%%%%%%%%%%%%%%%%%%

\paragraph*{\textbf{8)}}

I de ovenstående opgaver er loopkorrektionerne blevet betragtet til laveste orden. Forestil dig at vi kunne inkludere højereordenskorrektioner i propagatoren (flere loops osv.). Hvordan forestiller du dig, at $\pi$-propagatoren kan modificeres for højere ordener?


%%%%%%%%%%%%%%%%%%%%%%%%%%%%%%%%%%%%%%%%%%%%%%%%%%%%%%%%%%%%%%%%%%%%%%%%%%%%%%%%

\subsection{Besvarelse}

%%%%%%%%%%%%%%%%%%%%%%%%%

\paragraph[1) Dimension af felter og koblingskostanter i $\L$]{\textbf{1)}}




%%%%%%%%%%%%%%%%%%%%%%%%%

\paragraph[2) Massen af skalarfelterne $\pi$ og $\sigma$]{\textbf{2)}}




%%%%%%%%%%%%%%%%%%%%%%%%%

\paragraph[3) Vekselvirkningsvertices og tilsvarende koblingskostanter for Lagrangedensiteten $\L$]{\textbf{3)}}




%%%%%%%%%%%%%%%%%%%%%%%%%

\paragraph[4) Loopkorrigeringer til $\pi$-propagatoren]{\textbf{4)}}




%%%%%%%%%%%%%%%%%%%%%%%%%
\paragraph[5) Amplitude af specifik loopkorrigering]{\textbf{5)}}




%%%%%%%%%%%%%%%%%%%%%%%%%

\paragraph[6) Amplitude for resterende loopkorrigeringer]{\textbf{6)}}




%%%%%%%%%%%%%%%%%%%%%%%%%

\paragraph[7) For $p_\mu = 0$ for $\pi$ er addition af alle loopkorrektioners amplituder $0$]{\textbf{7)}}




%%%%%%%%%%%%%%%%%%%%%%%%%

\paragraph[8) Højereordens loopkorrektioner og $\pi$-propagatoren]{\textbf{8)}}




%%%%%%%%%%%%%%%%%%%%%%%%%%%%%%%%%%%%%%%%%%%%%%%%%%%%%%%%%%%%%%%%%%%%%%%%%%%%%%%%%%%%%

\end{document}