\documentclass[../main.tex]{subfiles}

\begin{document}

%%%%%%%%%%%%%%%%%%%%%%%%%%%%%%%%%%%%%%%%%%%%%%%%%%%%%%%%%%%%%%%%%%%%%%%%%%%%%%%%%%%%%

\section{Den lineære sigmamodel}

Betragt den følgende Lagrangedensitet
\begin{align} \label{eq:Opg6_Lagrangian}
    \L &= \inv{2} \left( \partial_\mu \sigma \right) \left( \partial^\mu \sigma \right) + \inv{2} \left( \partial_\mu \pi \right) \left( \partial^\mu \pi \right) - \inv{2} m^2 \sigma^2 - \lambda v \sigma^3 - \lambda v \sigma \pi^2 - \inv{4} \lambda \left( \sigma^2 + \pi^2 \right)^2 \: ,
\end{align}
hvor $\sigma$ og $\pi$ er skalarfelter og $v^2 = m^2/(2\lambda)$.


%%%%%%%%%%%%%%%%%%%%%%%%%%%%%%%%%%%%%%%%%%%%%%%%%%%%%%%%%%%%%%%%%%%%%%%%%%%%%%%%

\paragraph*{\textbf{1)}}

Find dimensionen af alle felter og koblingskonstanter i $\L$ i enheder, når $\hbar = c = 1$.


%%%%%%%%%%%%%%%%%%%%%%%%%

\paragraph*{\textbf{2)}}

Ved at kigge på $\L$, hvad vil du så mene, at massen af $\sigma$ og $\pi$ er?


%%%%%%%%%%%%%%%%%%%%%%%%%

\paragraph*{\textbf{3)}}

Tegn alle de lovlige vekselvirkningsknudepunkter (eng: interaction vertices) for denne Lagrangedensitet og bestem deres tilsvarende koblingskonstanter.


%%%%%%%%%%%%%%%%%%%%%%%%%

\paragraph*{\textbf{4)}}

Tegn alle loopkorrigeringerne til $\pi$-propagatoren. Du skal finde fem forskellige slags korrektioner.


%%%%%%%%%%%%%%%%%%%%%%%%%

\paragraph*{\textbf{5)}}

Betragt diagrammet hvor der er én $\pi$ og én $\sigma$, som løber i loopet (eng: running around the loop). Dette tillades af $\pi\pi\sigma$-ledet for andenordensperturbation. Vis at diagrammets amplitude er
\begin{align} \label{eq:Opg6_Q5_AmplitudeOfLoopCorrection}
    4 \lambda^2 v^2 \int \frac{\dd^4 k}{(2\pi)^4} \inv{k^2 - m^2} \inv{(k + p)^2} \: .
\end{align}
Er amplituden endelig eller uendelig, og hvis den er uendelig, hvordan divergerer den så for store $k$?\\
(Hint: Faktoren af $4$ er den såkaldte symmetrifaktor. For at udlede den, tænk da på det mulige antal måder, som man kan parre de eksterne $\pi_\textrm{in}$-  og $\pi_\textrm{out}$-felter med operatorerne i matrixelementet $\pi\pi\sigma\pi\pi\sigma$. Du kan også gøre dette ved at brute force gennem udvidelsen af Dysonserien, men at tænke sig til udledningen er noget mere elegant.)


%%%%%%%%%%%%%%%%%%%%%%%%%

\paragraph*{\textbf{6)}}

Nedskriv amplituderne for resten af loopkorrigeringerne til $\pi$-propagatoren. Husk symmetrifaktorerne; de er vigtige!


%%%%%%%%%%%%%%%%%%%%%%%%%

\paragraph*{\textbf{7)}}

Betragt nu grænsen for nul-firimpuls for $\pi$ ($p_\mu = 0$) i propagatoren. Vis at når du adderer alle korrektionerne for amplituden for $p_\mu = 0$, så får man $0$! (Hint: Der er en god idé at nedbryde (eng: decompose) integranten i amplituden i \textbf{5)} til partielle brøker (eng: partial fractions).)


%%%%%%%%%%%%%%%%%%%%%%%%%

\paragraph*{\textbf{8)}}

I de ovenstående opgaver er loopkorrektionerne blevet betragtet til laveste orden. Forestil dig at vi kunne inkludere højereordenskorrektioner i propagatoren (flere loops osv.). Hvordan forestiller du dig, at $\pi$-propagatoren kan modificeres for højere ordener?


%%%%%%%%%%%%%%%%%%%%%%%%%%%%%%%%%%%%%%%%%%%%%%%%%%%%%%%%%%%%%%%%%%%%%%%%%%%%%%%%

\subsection{Besvarelse}

%%%%%%%%%%%%%%%%%%%%%%%%%

\paragraph[1) Dimension af felter og koblingskostanter i $\L$]{\textbf{1)}}

Først betragter vi effekten af naturlige enheder $\hbar = c = 1$: Idet at $c = 1$, da bliver tid og sted sat på lige fod $1 = c = [l]/[t]$, hvorfor $[\text{længde}] = [\text{tid}]$. Ligeså gør det sig gældende, at når $c = 1$ så giver energi-impulsrelationen ($E^2 = p^2 + m^2$) at $[E] = [m]$. Sammenhængen mellem disse to sæt af sammenhænge findes ved $\hbar = 1$, hvilket gør at $E = \omega$, altså $[E] = [t]$. Ergo betyder naturlige enheder, at
\begin{align}
    [\text{længde}] = [\text{tid}] = [\text{energi}]^{-1} = [\text{masse}]^{-1} \: .
\end{align}

Dimensionen af alle felter og koblingskonstanter i $\L$ vil nu blive udtrykt i enheder af energi. Først bemærkes det, at Lagrangedensiteten må have dimension $[\L] = [E]$, idet at Lagrangefunktionen er dimensionsløs $[1] = [\int \dd^4 x] [\L] = ([l]^3[t]^1) [\L] = [E]^{-4} [\L] \Rightarrow [\L] = [E]^4$. Dermed skal alle led i Lagrangedensiteten altså have dimensionen $[E]^4$.

Først betragtes de kinetiske led. Siden $[\partial_\mu] = [E]$, da må $[\sigma] = [\pi] = [E]$. Dette stemmer også for tredje led i \cref{eq:Opg6_Lagrangian} med $[m]^2 [\sigma]^2 = [E]^2 [E]^2 = [E]^4$. For det fjerde led i Lagrangedensiteten (det femte led kunne her også have været brugt), hvor $[E]^4 = [\lambda v \sigma^3] = [\lambda v] [E]^3 \Rightarrow [\lambda v] = [E]$. Fra sidste led i Lagrangedensiteten har vi dog, at $[\lambda] = [1]$ idet $[(\sigma^2 + \pi^2)^2] = [E]^4$, hvormed $[v] = [E]$. Dermed har vi tilsammen at
\begin{align}
\begin{split}
    [E] &= [\sigma] = [\pi] = [v] \quad \text{og} \\
    [1] &= [\lambda] \: .
\end{split}
\end{align}


%%%%%%%%%%%%%%%%%%%%%%%%%

\paragraph[2) Massen af skalarfelterne $\pi$ og $\sigma$]{\textbf{2)}}

Betragter vi Lagrangedensiteten i \cref{eq:Opg6_Lagrangian} ses det, at den kan opdeles i et frit Lagrangedensitetsled og et vekselvirknings-Lagrangedensitetsled, hvor sidstnævnte er de sidste tre led. Dermed er det frie Lagrangedensitetsled
\begin{align} \label{eq:Opg6_A2_FreeLagrangian}
    \L_0 &= \inv{2} \left( \partial_\mu \sigma \right) \left( \partial^\mu \sigma \right) + \inv{2} \left( \partial_\mu \pi \right) \left( \partial^\mu \pi \right) - \inv{2} m^2 \sigma^2 \: .
\end{align}
Den generelle Lagrangedensitet for et frit skalarfelt $\phi$ er
\begin{align} \label{eq:Opg6_A2_FreeLagrangianGeneral}
    \L_{0,\textrm{gen}} &= \inv{2} \left( \partial_\mu \phi \right) \left( \partial^\mu \phi \right) - \inv{2} m^2 \phi^2 \: ,
\end{align}
så sammenlignes \cref{eq:Opg6_A2_FreeLagrangian} med \cref{eq:Opg6_A2_FreeLagrangianGeneral} for begge felter, $\sigma$ og $\pi$, ses det, at massen af $\sigma$-feltet er $m_\sigma = m$, mens massen af $\pi$-feltet er $m_\pi = 0$. Altså
\begin{align}
    m_\sigma &= m \qquad \text{og} \qquad m_\pi = 0 \: .
\end{align}


%%%%%%%%%%%%%%%%%%%%%%%%%

\paragraph[3) Vekselvirkningsknudepunkter og tilsvarende koblingskostanter for Lagrangedensiteten $\L$]{\textbf{3)}}

Siden der er 5 led i vekselvirkningsledet for Lagrangedensiteten i \cref{eq:Opg6_Lagrangian} forventes også 5 Feynmandiagrammer med vekselvirkning. Koblingskonstanterne tilhørende vekselvirkningsknudepunkterne er angivet på Feynmandiagrammerne, og for en Lagrangedensitet på formen $\L = \L_0 + \L_I$ med vekselvirkningen $\L_I = -k\phi_j^n$ er koblingskonstanten givet som $-i k$.

\begin{tikzpicture}[scale=.85, transform shape]
    \begin{feynman}
        \vertex [label={right:$-i \lambda v$}] (vertex);
        \vertex [above left = of vertex] (sigmaUpLeft) {\(\sigma\)};
        \vertex [above right = of vertex] (sigmaUpRight) {\(\sigma\)};
        \vertex [below = 2.6em of vertex] (sigmaDown) {\(\sigma\)};
        \diagram*{
            (sigmaDown) -- [scalar] (vertex) -- [scalar] (sigmaUpLeft),
            (vertex) -- [scalar] (sigmaUpRight),
        };
    \end{feynman}
\end{tikzpicture}
\hfill
\begin{tikzpicture}[scale=.85, transform shape]
    \begin{feynman}
        \vertex [label={right:$-i \lambda v$}] (vertex);
        \vertex [above left = of vertex] (sigmaUpLeft) {\(\sigma\)};
        \vertex [above right = of vertex] (piUpRight) {\(\pi\)};
        \vertex [below = 2.6em of vertex] (piDown) {\(\pi\)};
        \diagram*{
            (piDown) -- [boson] (vertex) -- [scalar] (sigmaUpLeft),
            (vertex) -- [boson] (piUpRight),
        };
    \end{feynman}
\end{tikzpicture}
\hfill
\begin{tikzpicture}[scale=.85, transform shape]
    \begin{feynman}
        \vertex [label={right:$-i \frac{1}{4} \lambda$}] (vertex);
        \vertex [above left = of vertex] (sigmaUpLeft) {\(\sigma\)};
        \vertex [above right = of vertex] (sigmaUpRight) {\(\sigma\)};
        \vertex [below left = of vertex] (sigmaDownLeft) {\(\sigma\)};
        \vertex [below right = of vertex] (sigmaDownRight) {\(\sigma\)};
        \diagram*{
            (sigmaDownLeft) -- [scalar] (vertex) -- [scalar] (sigmaUpLeft),
            (sigmaDownRight) -- [scalar] (vertex) -- [scalar] (sigmaUpRight),
        };
    \end{feynman}
\end{tikzpicture}
\hfill
\begin{tikzpicture}[scale=.85, transform shape]
    \begin{feynman}
        \vertex [label={right:$-i \frac{1}{2} \lambda$}] (vertex);
        \vertex [above left = of vertex] (sigmaUpLeft) {\(\sigma\)};
        \vertex [above right = of vertex] (piUpRight) {\(\pi\)};
        \vertex [below left = of vertex] (piDownLeft) {\(\pi\)};
        \vertex [below right = of vertex] (sigmaDownRight) {\(\sigma\)};
        \diagram*{
            (piDownLeft) -- [boson] (vertex) -- [scalar] (sigmaUpLeft),
            (sigmaDownRight) -- [scalar] (vertex) -- [boson] (piUpRight),
        };
    \end{feynman}
\end{tikzpicture}
\hfill
\begin{tikzpicture}[scale=.85, transform shape]
    \begin{feynman}
        \vertex [label={right:$-i \frac{1}{4} \lambda$}] (vertex);
        \vertex [above left = of vertex] (piUpLeft) {\(\pi\)};
        \vertex [above right = of vertex] (piUpRight) {\(\pi\)};
        \vertex [below left = of vertex] (piDownLeft) {\(\pi\)};
        \vertex [below right = of vertex] (piDownRight) {\(\pi\)};
        \diagram*{
            (piDownLeft) -- [boson] (vertex) -- [boson] (piUpLeft),
            (piDownRight) -- [boson] (vertex) -- [boson] (piUpRight),
        };
    \end{feynman}
\end{tikzpicture}


%%%%%%%%%%%%%%%%%%%%%%%%%

\paragraph[4) Loopkorrigeringer til $\pi$-propagatoren]{\textbf{4)}}

Loopkorrigeringerne for Lagrangedensiteten i \cref{eq:Opg6_Lagrangian} er vist herunder.

\begin{tikzpicture}
    \begin{feynman}
        \vertex (vertexUp);
        \vertex [below = 2.6em of vertexUp] (vertexDown);
        \vertex [above = 2.6em of vertexUp] (piUp) {\(\pi\)};
        \vertex [below = 2.6em of vertexDown] (piDown) {\(\pi\)};
        \diagram*{
            (piDown) -- [boson] (vertexDown) -- [scalar, half left, edge label = $\sigma$] (vertexUp),
            (vertexDown) -- [boson, half right, edge label = $\pi$, swap] (vertexUp) -- [boson] (piUp),
            % swap is for swapping the side of the edge label placement
        };
    \end{feynman}
\end{tikzpicture}
\hfill
\begin{tikzpicture}
    \begin{feynman}
        \vertex (vertex);
        \vertex [left = 3em of vertex] (loopOutermostPoint);
        \vertex [above = 3.9em of vertex] (piUp) {\(\pi\)};
        \vertex [below = 3.9em of vertex] (piDown) {\(\pi\)};
        \diagram*{
            (piDown) -- [boson] (vertex) -- [boson] (piUp),
            (vertex) -- [boson, half right, edge label' = $\pi$, looseness = 1.6] (loopOutermostPoint) -- [boson, half right, looseness = 1.6] (vertex),
        };
    \end{feynman}
\end{tikzpicture}
\hfill
\begin{tikzpicture}
    \begin{feynman}
        \vertex (vertex);
        \vertex [left = 3em of vertex] (loopOutermostPoint);
        \vertex [above = 3.9em of vertex] (piUp) {\(\pi\)};
        \vertex [below = 3.9em of vertex] (piDown) {\(\pi\)};
        \diagram*{
            (piDown) -- [boson] (vertex) -- [boson] (piUp),
            (vertex) -- [scalar, half right, edge label' = $\sigma$, looseness = 1.6] (loopOutermostPoint) -- [scalar, half right, looseness = 1.6] (vertex),
            % swap is for swapping the side of the edge label placement
        };
    \end{feynman}
\end{tikzpicture}
\hfill
\begin{tikzpicture}
    \begin{feynman}
        \vertex (vertex);
        \vertex [left = 3em of vertex] (loopInnermostPoint);
        \vertex [left = 3em of loopInnermostPoint] (loopOutermostPoint);
        \vertex [above = 3.9em of vertex] (piUp) {\(\pi\)};
        \vertex [below = 3.9em of vertex] (piDown) {\(\pi\)};
        \diagram*{
            (piDown) -- [boson] (vertex) -- [boson] (piUp),
            (vertex) -- [scalar, edge label = $\sigma$, swap] (loopInnermostPoint),
            (loopInnermostPoint) -- [boson, half right, edge label' = $\pi$, looseness = 1.6] (loopOutermostPoint) -- [boson, half right, looseness = 1.6] (loopInnermostPoint),
        };
    \end{feynman}
\end{tikzpicture}
\hfill
\begin{tikzpicture}
    \begin{feynman}
        \vertex (vertex);
        \vertex [left = 3em of vertex] (loopInnermostPoint);
        \vertex [left = 3em of loopInnermostPoint] (loopOutermostPoint);
        \vertex [above = 3.9em of vertex] (piUp) {\(\pi\)};
        \vertex [below = 3.9em of vertex] (piDown) {\(\pi\)};
        \diagram*{
            (piDown) -- [boson] (vertex) -- [boson] (piUp),
            (vertex) -- [scalar, edge label = $\sigma$, swap] (loopInnermostPoint),
            (loopInnermostPoint) -- [scalar, half right, edge label' = $\sigma$, looseness = 1.6] (loopOutermostPoint) -- [scalar, half right, looseness = 1.6] (loopInnermostPoint),
        };
    \end{feynman}
\end{tikzpicture}


%%%%%%%%%%%%%%%%%%%%%%%%%
\paragraph[5) Amplitude af $\pi\pi\sigma$-ledets loopkorrigering]{\textbf{5)}}

Lad os først tegne Feynmandiagrammet igen og denne gang notere impulserne på diagrammet.

\begin{center}
\begin{tikzpicture}
    \begin{feynman}
        \vertex (vertexUp);
        \vertex [below = 4em of vertexUp] (vertexDown);
        \vertex [above = 2.6em of vertexUp] (piUp) {\(\pi\)};
        \vertex [below = 2.6em of vertexDown] (piDown) {\(\pi\)};
        \diagram* [layered layout] {
            (piDown) -- [boson, momentum' = $p$] (vertexDown) -- [scalar, half left, looseness=1.75, edge label' = $\sigma$, momentum = $k$] (vertexUp),
            (vertexDown) -- [boson, half right, looseness=1.75, edge label = $\pi$, momentum' = $p-k$] (vertexUp) -- [boson, momentum' = $p$] (piUp),
        };
    \end{feynman}
\end{tikzpicture}
\end{center}

Nu benyttes Feynmanreglerne for skalarfelter, \cite[lign./opg. ??]{}. Først vides det, at indkomne og udgående partikler (initial/final) bidrager med $1$, og at hvert knudepunkt bidrager med koblingskonstanten $g = - i \lambda v$. Ydermere bidrage de virtuelle partikler hver med en propagator på formen $-i/(p'^2 - m^2)$, hvor $p'$ er impulsen, så for de to virtuelle partikler får vi propagatorerne
\begin{align}
    \Delta_\sigma &= \frac{-i}{k^2 - m^2} \qquad \text{og} \qquad
    \Delta_\pi = \frac{-i}{(p - k)^2} \: ,
\end{align}
da $m_\pi = 0$. Sidst skal symmetrifaktoren findes, hvilket gøres ved at betragte matrixelementet, som opstår i spredningsamplituden til anden orden,
\begin{align}
\begin{split}
    &\mel**{\pi,\, p}{\pi(x) \pi(x) \sigma(x) \pi(0) \pi(0) \sigma(0)}{\pi,\, p} \\
        &\qquad\quad = \mel**{\pi,\, p}{\pi(x) \pi(x) \pi(0) \pi(0)}{\pi,\, p} \mel**{0}{\sigma(x) \sigma(0)}{0} \: .
\end{split}
\end{align}
$\sigma$-matrixelementet kan kun udføres på én måde, så dette led bidrager ikke med nogle symmetrifaktorer. $\pi$-matrixelementet derimod bidrager med symmetrifaktoren $8$. Dette kan ses, da hvert $\pi$-felt indeholder en kreations- og en annihilationsoperator for $\pi$-partikler, og symmetrifaktoren er antallet af måder, hvorpå en $\pi$ partikel kan annihileres og en virtuel $\pi$-partikel skabes, hvorefter denne annihileres og en $\pi$-partikel skabes. Den første $\pi$-partikel kan skabes ved enhver af de fire operatorer, hvilket bidrager en symmetrifaktorer af $4$, og den udgående partikel kan da kun skabes ved to af de fire operatorer, da den ikke må skabes samme sted som den indgående partikel blev annihileret, hvilket bidrager en symmetrifaktor på $2$. Ergo bliver den overordnede symmetrifaktor $S_{\text{sym}} = 1 \cdot 4 \cdot 2 = 8$.

Samles alt det ovenstående, og huskes faktoren $1/2$ fra Dysonserien til anden orden, fås amplituden
\begin{align} \label{eq:Opg6_A5_AmplitudeOfLoopCorrection1}
\begin{split}
    M_1 &= \inv{2} S_{\text{sym}}\, g^2 \int \frac{\dd^4 k}{(2\pi)^4} \Delta_\sigma \Delta_\pi \\
        &= \inv{2} 8 (-i)^2 \lambda^2 v^2 \int \frac{\dd^4 k}{(2\pi)^4}\, \frac{-i}{k^2 - m^2}\, \frac{-i}{(p - k)} \\
        &= 4 (-i)^4 \lambda^2 v^2 \int \frac{\dd^4 k}{(2\pi)^4}\, \inv{k^2 - m^2}\, \inv{(p - k)^2} \\
        &= 4 (-1)^2 \lambda^2 v^2 \int \frac{\dd^4 k}{(2\pi)^4}\, \inv{(-k)^2 - m^2}\, \inv{\big(p - [-k]\big)^2} \\
        &= 4 \lambda^2 v^2 \int \frac{\dd^4 k}{(2\pi)^4}\, \inv{k^2 - m^2}\, \inv{(p + k)^2} \: ,
\end{split}
\end{align}
hvor vi ved anden sidste lighed har ladet $k \rightarrow - k$, hvilket vi har lov til, da vi integrerer over hele impulsrummet. Dette er amplituden, \cref{eq:Opg6_Q5_AmplitudeOfLoopCorrection}, som skulle vises.
\\

Denne amplitude er divergerende for store $k$, hvilket kan ses ved
\begin{align}
    4 \lambda^2 v^2 \int \frac{\dd^4 k}{(2\pi)^4}\, \inv{k^2 - m^2}\, \inv{(p + k)^2}
        \overset{\text{stor $k$}}{\simeq} 4 \lambda^2 v^2 \int \frac{\dd^4 k}{(2\pi)^4}\, \inv{k^2}\, \inv{k^2}
        = 4 (2\pi)^4 \lambda^2 v^2 \int \frac{\dd^4 k}{k^4} \: .
\end{align}
Altså divergerer amplituden logaritmisk for store $k$.


%%%%%%%%%%%%%%%%%%%%%%%%%

\paragraph[6) Amplitude for resterende loopkorrigeringer]{\textbf{6)}}

Besvarelsen af denne opgave følger den præcis samme metode som \textbf{5)}, hvorfor der i denne opgave ikke vil blive skrevet alting ud i ord, da forklaring af, hvordan man finder frem til ledene, er beskrevet i \textbf{5)}.
\\

For den anden loopkorrektion i \textbf{4)} har vi, at Feynmandiagrammet med impulser er som følger nedenfor.

\vspace{-.5em}
\begin{center}
\begin{tikzpicture}
    \begin{feynman}
        \vertex (vertex);
        \vertex [left = 3em of vertex] (loopOutermostPoint);
        \vertex [above = 3.9em of vertex] (piUp) {\(\pi\)};
        \vertex [below = 3.9em of vertex] (piDown) {\(\pi\)};
        \diagram*{
            (piDown) -- [boson, momentum' = $p$] (vertex) -- [boson, momentum' = $p$] (piUp),
            (vertex) -- [boson, half left, momentum' = $k$, looseness = 1.6] (loopOutermostPoint) -- [boson, half left, edge label = $\pi$, momentum', looseness = 1.6] (vertex),
        };
    \end{feynman}
\end{tikzpicture}
\end{center}
\vspace{-1em}
%
Symmetrifaktoren på $12$ kommer af, at knudepunktet har ''fire'' indgående $\pi$-partikler, hvorfor der er fire mulige valg af $\pi$ for den indkomne $\pi$-partikel og dermed kun tre valg for den udgående, og de sidste to er den samme og er i loopet, hvorfor denne ikke giver et bidrag. Dermed bliver amplituden
\begin{align} \label{eq:Opg6_A5_AmplitudeOfLoopCorrection2}
\begin{split}
    M_2 &= S_{\text{sym}}\, g \int \frac{\dd^4 k}{(2\pi)^4}\, \Delta_\pi \\
        &= 12 \left(-\frac{i\lambda}{4} \right) \int \frac{\dd^4 k}{(2\pi)^4} \frac{i}{k^2} \\
        &= 3 \lambda \int \frac{\dd^4 k}{(2\pi)^4}\, \inv{k^2} \: .
\end{split}
\end{align}
\\

For den tredje loopkorrektion i \textbf{4)} har vi, at Feynmandiagrammet med impulser er som følger nedenfor.

\vspace{-.5em}
\begin{center}
\begin{tikzpicture}
    \begin{feynman}
        \vertex (vertex);
        \vertex [left = 3em of vertex] (loopOutermostPoint);
        \vertex [above = 3.9em of vertex] (piUp) {\(\pi\)};
        \vertex [below = 3.9em of vertex] (piDown) {\(\pi\)};
        \diagram*{
            (piDown) -- [boson, momentum' = $p$] (vertex) -- [boson, momentum' = $p$] (piUp),
            (vertex) -- [scalar, half left, momentum' = $k$, looseness = 1.6] (loopOutermostPoint) -- [scalar, half left, edge label = $\sigma$, momentum', looseness = 1.6] (vertex),
        };
    \end{feynman}
\end{tikzpicture}
\end{center}
\vspace{-1em}
%
Her er symmetrifaktoren $2$, da der er to mulige valg for den indkomne $\pi$, mens der kun er én mulighed for den udgående, og $\sigma$-partiklen ikke bidrager, da der ikke er andre steder, at denne kan være. Dermed bliver amplituden
\begin{align} \label{eq:Opg6_A5_AmplitudeOfLoopCorrection3}
\begin{split}
    M_3 &= S_\text{sym}\, g \int \frac{\dd^4 k}{(2\pi)^4}\, \Delta_\sigma \\
        &= 2 \frac{-i\lambda}{2} \int \frac{\dd^4 k}{(2\pi)^4}\, \frac{i}{k^2 - m^2} \\
        &= \lambda \int \frac{\dd^4 k}{(2\pi)^4}\, \inv{k^2 - m^2} \: .
\end{split}
\end{align}
\\

For den fjerde loopkorrektion i \textbf{4)} har vi, at Feynmandiagrammet med impulser er som følger nedenfor.

\vspace{-.5em}
\begin{center}
\begin{tikzpicture}
    \begin{feynman}
        \vertex (vertex);
        \vertex [left = 3em of vertex] (loopInnermostPoint);
        \vertex [left = 3em of loopInnermostPoint] (loopOutermostPoint);
        \vertex [above = 3.9em of vertex] (piUp) {\(\pi\)};
        \vertex [below = 3.9em of vertex] (piDown) {\(\pi\)};
        \diagram*{
            (piDown) -- [boson, momentum' = $p$] (vertex) -- [boson, momentum' = $p$] (piUp),
            (vertex) -- [scalar, edge label' = $\sigma$, momentum = $0$] (loopInnermostPoint),
            (loopInnermostPoint) -- [boson, half left, looseness = 1.6, momentum' = $k$] (loopOutermostPoint) -- [boson, half left, edge label = $\pi$, looseness = 1.6, momentum'] (loopInnermostPoint),
        };
    \end{feynman}
\end{tikzpicture}
\end{center}
\vspace{-1em}
%
I dette diagram er symmetrifaktoren $4$, der er to knudepunkter og to $\pi$-felter, som kan være indgående og udgående partikler i det højre knudepunkt. Her skal vi huske at inkludere en faktor af $1/2$ fra Dysonserien, da der er tale om en andenordenskorrektion (to knudepunkter), og vi skal medtage propagatoren for begge virtuelle partikler. Af dette bliver amplituden
\begin{align} \label{eq:Opg6_A5_AmplitudeOfLoopCorrection4}
\begin{split}
    M_4 &= \inv{2} S_\text{sym}\, g \int \frac{\dd^4 k}{(2\pi)^4}\, \Delta_\sigma \Delta_\pi \\
        &= \inv{2} 4 (-i)^2 \lambda^2 v^2 \int \frac{\dd^4 k}{(2\pi)^4}\, \frac{i}{-m^2}\, \frac{i}{k^2} \\
        &= - 2 \frac{\lambda^2 v^2}{m^2} \int \frac{\dd^4 k}{(2\pi)^4}\, \inv{k^2} \: .
\end{split}
\end{align}
\\

For den sidste loopkorrektion i \textbf{4)} har vi, at Feynmandiagrammet med impulser er som følger nedenfor.

\vspace{-.5em}
\begin{center}
\begin{tikzpicture}
    \begin{feynman}
        \vertex (vertex);
        \vertex [left = 3em of vertex] (loopInnermostPoint);
        \vertex [left = 3em of loopInnermostPoint] (loopOutermostPoint);
        \vertex [above = 3.9em of vertex] (piUp) {\(\pi\)};
        \vertex [below = 3.9em of vertex] (piDown) {\(\pi\)};
        \diagram*{
            (piDown) -- [boson, momentum' = $p$] (vertex) -- [boson, momentum' = $p$] (piUp),
            (vertex) -- [scalar, edge label' = $\sigma$, momentum = $0$] (loopInnermostPoint),
            (loopInnermostPoint) -- [scalar, half left, looseness = 1.6, momentum' = $k$] (loopOutermostPoint) -- [scalar, half left, edge label = $\sigma$, looseness = 1.6, momentum'] (loopInnermostPoint),
        };
    \end{feynman}
\end{tikzpicture}
\end{center}
\vspace{-1em}
%
Symmetrifaktoren i dette diagram er $12$, da der er to knudepunkter, samt to $\pi$-felter og tre $\sigma$-felter at vælge mellem. Igen skal vi huske faktoren $1/2$ fra Dysonserien og begge propagatorer. Derved fås amplituden
\begin{align} \label{eq:Opg6_A5_AmplitudeOfLoopCorrection5}
\begin{split}
    M_5 &= \inv{2} S_\text{sym}\, g \int \frac{\dd^4 k}{(2\pi)^4}\, \Delta_\sigma \Delta_{\sigma}' \\
        &= \inv{2} 12 (-i)^2 \lambda^2 v^2 \int \frac{\dd^4 k}{(2\pi)^4}\, \frac{i}{-m^2}\, \frac{i}{k^2 - m^2} \\
        &= - 6 \frac{\lambda^2 v^2}{m^2} \int \frac{\dd^4 k}{(2\pi)^4}\, \inv{k^2 - m^2} \: .
\end{split}
\end{align}


%%%%%%%%%%%%%%%%%%%%%%%%%

\paragraph[7) For $p_\mu = 0$ for $\pi$ er addition af alle loopkorrektioners amplituder $0$]{\textbf{7)}}

Først kigger vi på \cref{eq:Opg6_Q5_AmplitudeOfLoopCorrection}, hvor $p = p^\mu = 0$
\begin{align}
    M_1 &= 4 \lambda^2 v^2 \int \frac{\dd^4 k}{(2\pi)^4}\, \inv{k^2 - m^2}\, \inv{(p + k)^2}
        = 4 \lambda^2 v^2 \int \frac{\dd^4 k}{(2\pi)^4}\, \inv{k^2 - m^2}\, \inv{k^2}
\end{align}
og opdeler vi delene, som kommer fra propagatorerne, til partielle brøker, altså
\begin{align}
    \inv{k^2} \inv{k^2 - m^2} &= \frac{m^4}{m^4 k^2 (k^2 - m^2)}
        = \frac{m^2 k^2 - m^2(k^2 - m^2)}{m^4 k^2 (k^2 - m^2)}
        = \inv{m^2 (k^2 - m^2)} - \inv{m^2 k^2} \: .
\end{align}
Summeres nu over alle loopkorrektionernes amplituder, \cref{eq:Opg6_A5_AmplitudeOfLoopCorrection1,eq:Opg6_A5_AmplitudeOfLoopCorrection2,eq:Opg6_A5_AmplitudeOfLoopCorrection3,eq:Opg6_A5_AmplitudeOfLoopCorrection4,eq:Opg6_A5_AmplitudeOfLoopCorrection5}, hvor $p = p^\mu = 0$, og benyttes det at $v^2 = m^2 / (2\lambda)$, fås
\begin{align}
\begin{split}
    \sum_i M_i &= \int \frac{\dd^4 k}{(2\pi)^4} \bigg( 4 \lambda^2 v^2 \inv{m^2 (k^2 - m^2)} - 4 \lambda^2 v^2 \inv{m^2 k^2} + 3 \lambda \inv{k^2} \\
            &\qquad\qquad\qquad + \lambda \inv{k^2 - m^2} - \frac{2 \lambda^2 v^2}{m^2} \inv{k^2} - \frac{6 \lambda^2 v^2}{m^2} \inv{k^2 - m^2} \bigg) \\
        &= \int \frac{\dd^4 k}{(2\pi)^4} \bigg( 4 \frac{\lambda^2 m^2}{2\lambda} \inv{m^2 (k^2 - m^2)} - 4 \frac{\lambda^2 m^2}{2\lambda} \inv{m^2 k^2} + 3 \lambda \inv{k^2} \\
            &\qquad\qquad\qquad + \lambda \inv{k^2 - m^2} - \frac{2 \lambda^2 m^2}{2 m^2 \lambda} \inv{k^2} - \frac{6 \lambda^2 m^2}{2 m^2 \lambda} \inv{k^2 - m^2} \bigg) \\
        &= \int \frac{\dd^4 k}{(2\pi)^4} \bigg( \frac{2 \lambda}{k^2 - m^2} - \frac{2 \lambda}{k^2} + \frac{3 \lambda}{k^2} + \frac{\lambda}{k^2 - m^2} - \frac{\lambda}{k^2} - \frac{3 \lambda}{k^2 - m^2} \bigg) \\
        &= \int \frac{\dd^4 k}{(2\pi)^4} \cdot 0 \\
        &= 0 \: .
\end{split}
\end{align}


%%%%%%%%%%%%%%%%%%%%%%%%%

\paragraph[8) Højereordens loopkorrektioner og $\pi$-propagatoren]{\textbf{8)}}

\ldots

\begin{center}
\begin{tikzpicture}
    \begin{feynman}
        \vertex (vertexUp);
        \vertex [below = 4em of vertexUp] (vertexDown);
        \vertex [above = 2.6em of vertexUp] (piUp) {\(\pi\)};
        \vertex [below = 2.6em of vertexDown] (piDown) {\(\pi\)};
        \diagram* [layered layout] {
            (piDown) -- [boson] (vertexDown) -- [boson, half left, looseness=1.75, edge label = $\pi$] (vertexUp),
            (vertexDown) -- [boson, half right, looseness=1.75, edge label' = $\pi$] (vertexUp) -- [boson] (piUp),
            (vertexDown) -- [boson, edge label' = $\pi$] (vertexUp),
        };
    \end{feynman}
\end{tikzpicture}
\end{center}

% I de ovenstående opgaver er loopkorrektionerne blevet betragtet til laveste orden. Forestil dig at vi kunne inkludere højereordenskorrektioner i propagatoren (flere loops osv.). Hvordan forestiller du dig, at $\pi$-propagatoren kan modificeres for højere ordener?


%%%%%%%%%%%%%%%%%%%%%%%%%%%%%%%%%%%%%%%%%%%%%%%%%%%%%%%%%%%%%%%%%%%%%%%%%%%%%%%%%%%%%

\end{document}